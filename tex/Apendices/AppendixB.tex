\chapter{Teoría de Gráficas}

\begin{dfn} \cite{Yo} \\
Una \textbf{gráfica} es una pareja $G = (V, E)$ donde $V$ es un conjunto finito no vacío
cuyos elementos se llaman \textbf{vértices} y $E$ es un conjunto cuyos elementos son subconjuntos de
cardinalidad 2 de V y estos son llamados \textbf{aristas}.
\end{dfn}

\begin{dfn} \cite{Yo} \\ 
El \textbf{grado} de un vértice $v$ es el número de aristas que inciden en $v$, se denota $d(v)$.
\end{dfn}

\begin{dfn} \cite{Ramon}
Se dice que una gráfica es \textbf{bipartita}, si su conjunto de vertices puede ser partido en dos subconjuntos ajenos $X$ y $Y$, de tal forma que cada arista tiene un extremo en $X$ y otro en $Y$. Esta partición tiene el nombre de $\textbf{bipartición}$ de la gráfica.
\end{dfn}

