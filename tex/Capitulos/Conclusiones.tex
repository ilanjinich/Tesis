\chapter{Conclusiones}

En esta tesis se construyó el camino para explicar los problemas de admisión a universidades con cotas inferiores o con cotas comunes. Estos a pesar de su facilidad al momento de explicarlos tienen la propiedad de ser NP-Completos y por lo tanto no son fáciles de resolver. Al mismo tiempo se vio que si se hace una reducción al problema de cotas comunes, éste es fácil de resolver y simultáneamente notamos que cuenta con una estructura algebraica muy poderosa. 

Al principio de la tesis se habló de como estos problemas se pueden plantear usando restricciones enteras, esto fue con el fin de plantear todo en el mundo de investigación de operaciones en el que estamos todos acostumbrados. Además, se aprovechó para dar una muy breve introducción a las clases de complejidad por si el lector no se encontraba familiarizado con el tema. 

Posteriormente, se dio una introducción a dos problemas realmente simples: el del matrimonio estable y el de admisión de universidades. Estos a pesar de su simplicidad son el primer paso para entender la situación completa de la tesis y nos dieron herramientas para estudiar el resto de los resultados. 

Aprovechando los resultados anteriores se plantearon tres problemas y se demostró que dos de ellos son NP-Completos (el tercero también cumple esta propiedad, pero se decidió que la demostración no iba con los objetivos de la tesis). Estos problemas resultan interesantes porque con solo incluir una restricción a los problemas originales estos pasan de ser muy fáciles de resolver a muy difíciles. 

Al final se vio que uno de los tres problemas de los planteados (el de cotas comunes) si le agregamos la propiedad de que éstas son anidadas, el problema pasa de regreso a ser muy simple de resolver. Además, usando una estructura algebraica llamada matroides tuvimos la oportunidad de reevaluar y generalizar muchos de los resultados de la tesis en un contexto diferente. 
