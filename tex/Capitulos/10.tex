\chapter{Funciones de elección y matroides}

En este capítulo redefiniremos el problema de admisión a universidades con cotas comunes anidadas por completo, vamos a suponer que tenemos una gráfica bipartita $G=(A\cup C,E)$ donde los vértices en $A$ representan los aspirantes, $C$ representa las universidades y las aristas de la gráfica $E$ representan las asignaciones aceptables entre los dos. Para definir el problema de esta manera es necesario definir dos funciones de elección. 

\begin{dfn}
Decimos que $Q$ es una \textbf{función de elección} para un conjunto $E$ si para todo subconjunto $X$ de $E$ tenemos que $Q(X)$ es un subconjunto de $X$. En particular, decimos que un conjunto $X$ es \textbf{$Q$-independiente} si $Q(X)=X$.
\end{dfn}

Definimos dos funciones de elección una para los solicitantes denotado como $Q_A$ y otra para las universidades denotada como $Q_C$. Si un conjunto es $Q_A$-independiente decimos que es \textbf{aspirante independiente} y si un conjunto es $Q_C$-independiente decimos que es \textbf{universidad independiente}. Las dos funciones se definen de la siguiente manera:

\begin{dfn}
Definimos la \textbf{función de elección para los aspirantes} $Q_A$ como una función de elección sobre las aristas que cumple que para todo aspirante $a$, $Q_A(X)$ contiene a las aristas en $X$ que el solicitante escogería si pudiera elegir libremente sin importar las restricciones. En particular, si $X$ contiene varias aristas que inciden sobre $a$ éste elige la que incide en la universidad más alta en su lista. 
\end{dfn} 

\begin{dfn}
Definimos la \textbf{función de elección para las universidades} $Q_C$ como una función de elección sobre las aristas que cumple que para toda universidad $c$, $Q_C(X)$ contiene a las aristas $X$ que inciden en los solicitantes que $c$ elegiría si pudiera elegir libremente. En particular se sigue el siguiente algoritmo, supongamos que la familia de conjuntos $\mathcal{C}=\{C_1,C_2,\dots,C_m\}$ son los conjuntos de universidades que son parte de una restricción y supongamos que si $C_i$ es un subconjunto de $C_j$ es porque $i$ es menor igual que $j$:

\IncMargin{1em}
\begin{Algoritmo}[H]
%\SetKwData{Left}{left}\SetKwData{This}{this}\SetKwData{Up}{up}
%\SetKwFunction{Union}{Union}\SetKwFunction{FindCompress}{FindCompress}
\SetKwInOut{Input}{input}\SetKwInOut{Output}{output}

\Input{Un conjunto $X$ de aristas en $E$, la familia de conjuntos con restricciones $\mathcal{C}$ con sus cotas y las listas de preferencias para las universidades y los aspirantes}
\Output{$Q(X)$}
\BlankLine
\emph{Sea $X_0 = X$;}

\For{i = 1, \dots, m}{

\emph{$X_i$ es el conjunto obtenido por aplicarle la cota superior de $C_i$ a $X_{i-1}$;}

\emph{Sea $X_{i-1}(C_i)$ las aristas en $X_{i-1}$ que inciden en $C_i$;}

\emph{Sean $X'_{i}$ las aristas en $X_{i-1}(C_i)$ que no se encuentran entre las $q(C_i)$ mejores;}

\emph{$X_i = X_{i-1}\setminus X'_i$;}
}

\emph{$Q_C(X) = X_m$;}
\caption{Algoritmo para calcular la función de elección para las universidades}
\end{Algoritmo}
\DecMargin{1em}

Para la línea 5 del algoritmo es necesario explicar cuál es el orden de preferencias de $X_{i-1}(C_i)$, este depende de dos casos: 
\begin{enumerate}
\item Dadas dos aristas que inciden en dos solicitantes distintos escogemos de acuerdo con el orden de preferencias de $C_i$.
\item Dadas dos aristas que inciden en el mismo aspirante $a$ escogemos de acuerdo con el orden de preferencias de $a$.
\end{enumerate}

\end{dfn}

A partir de esto realizamos las siguientes observaciones sobre el comportamiento de estas funciones:

\begin{obs}
$Q_A(X)=X$ si y solo si cada aspirante tiene a lo más una arista incidente en $X$.
\end{obs}
\begin{obs}
$Q_C(X)=X$ si y solo si $X$ no viola ninguna cota de $\mathcal{C}$.
\end{obs}

Usando esas dos observaciones concluimos que:

\begin{obs}
$M$ es un emparejamiento si y solo si $M$ es aspirante independiente y universidad independiente. 
\end{obs}

A partir de estos resultados podemos demostrar el siguiente teorema que habla de la relación entre nuestro problema y las matroides. 

\begin{teo}
\label{matroid}
Sea $I_C$ el subconjunto del conjunto potencia de $E$ que contiene a todos los conjuntos universidad independiente, entonces $M_C = (E,I_C)$ es una matroide. 
\end{teo}

\begin{proof}
Usamos inducción sobre el número $m$ de conjuntos acotados (la cardinalidad de $\mathcal{C}$). Si $m=1$ entonces solo tenemos una universidad y $M_C$ es una matroide uniforme de rango $q(C_i)$. 
Supongamos que el resultado es válido si la cantidad de conjuntos es estrictamente menor que $m$ y supongamos que tenemos $m$ conjuntos acotados. Sean $C^1,C^2,\dots, C^k$ los grupos en $\mathcal{C}$, entonces tenemos dos casos:
\begin{enumerate}
\item Si $k$ es mayor estricto que $1$ entonces cada $I_{C^i}$ forma una matroide y como $I_C$ es la suma directa de los $I_{C^i}$ entonces es una matroide. 
\item Si $k=1$ entonces $I'$ es el conjunto de conjuntos universidad independientes de acuerdo con $C_1,\dots,C_{m-1}$ es parte de una matroide. Podemos ver que al aplicar la restricción de $C_m$ tenemos un truncamiento sobre la misma y no se pierde el hecho de que es una matroide. 
\end{enumerate}
Por lo tanto, si $I_C$ el subconjunto del conjunto potencia de $E$ que contiene a todos los conjuntos universidad independiente, entonces $M_C = (E,I_C)$ es una matroide.
\end{proof}

Un resultado inmediato del teorema es que lo mismo sucede para los aplicantes. 

\begin{cor}
Sea $I_A$ el subconjunto del conjunto potencia de $E$ que contiene a todos los conjuntos universidad independiente, entonces $M_A = (E,I_A)$ es una matroide. 
\end{cor}
\begin{proof}
Como los aspirantes pueden ser vistos como universidades en donde cada cota superior es 1 y no existen cotas comunes, de acuerdo con el teorema \ref{matroid} $M_A = (E,I_A)$ es una matroide. 
\end{proof}

A partir de esto, podemos encontrar varios resultados bastante interesantes sobre el problema. 

\begin{cor}
La función de elección para las universidades para $X$ encuentra el subconjunto maximal universidad independiente de $X$.
\end{cor}

\begin{proof}
Para cada elemento $x$ en $X\setminus Q_c(X)$ tenemos que el conjunto $\{x\}\cup Q_c(X)$ es no universidad independiente, por definición existe un conjunto acotado $C_i$ tal que $Q_c(X)$ contiene $q(C_i)$ aristas incidentes en $C_i$ que todas son preferidas a $x$.
\end{proof}

A partir de esto podemos ver dos observaciones.

\begin{obs}
La cardinalidad de $Q_c(X)$ es igual al rango de $X$ respecto a la matroide de las universidades $M_c$. 
\end{obs}

\begin{obs}
La función $Q_c$ es creciente. Es decir, si $X$ es un subconjunto de $Y$ entonces la cardinalidad de $Q_c(X)$ es menor o igual que la cardinalidad de $Q_c(Y)$. \footnote{La demostración es directo del 9.8 } %Me falta referencia
\end{obs} 

Estas dos observaciones implican que la función $Q_c$ puede ser construida a partir del algoritmo glotón de la siguiente manera:

\IncMargin{1em}
\begin{Algoritmo}[H]
%\SetKwData{Left}{left}\SetKwData{This}{this}\SetKwData{Up}{up}
%\SetKwFunction{Union}{Union}\SetKwFunction{FindCompress}{FindCompress}
\SetKwInOut{Input}{input}\SetKwInOut{Output}{output}

\Input{Un conjunto $X$ de aristas en $E$, la familia de conjuntos con restricciones $\mathcal{C}$ con sus cotas y las listas de preferencias para las universidades y los solicitantes}
\Output{$Q(X)$}
\BlankLine
\emph{Acomodamos las aristas de $E=e_1,e_2,\dots,e_n$ de cierta forma que si dos aristas $e_i$ y $e_j$ pertenecen al mismo conjunto acotado entonces si $i$ es menor que $j$ implica que $e_i$ es preferido a $e_j$;}

\emph{Sea $E_i$ las aristas $e_j$ con $j$ menor o igual que $i$;}

\emph{Sea $X_i=X \cap E_i$;}

\emph{Sea $X_0 = \emptyset$;}

\For{$i = 1, \dots, n$}{
\If{$Q_c(X_i)\cup\{e_{i+1}\}$ es universidad independiente}{
\emph{$Q_c(X_i+1)=Q_c(X_i)\cup\{e_{i+1}\}$}

\Else{$Q_c(X_{i+1})=Q_c(X_i)$;}
}
}

\caption{Algoritmo alternativo para calcular la función de elección para las universidades}
\end{Algoritmo}
\DecMargin{1em}

El algoritmo es igual porque por el teorema (\ref{glot}) sabemos que el algoritmo glotón siempre converge al subconjunto maximal universidad independiente de $X$ y por lo tanto aplicar uno u otro algoritmo debe de llevar al mismo resultado. Además de esto si agregamos la siguiente definición podemos ver que la función es comonotona, es decir:

\begin{dfn}
Decimos que una función $Q$ en $E$ es \textbf{comonotona} si para cualquier par de subconjuntos $X,Y$ de $E$ con $X$ siendo un subconjunto de $Y$ tenemos que $X\setminus Q(X)$ es un subconjunto de $Y\setminus Q(Y)$.
\end{dfn}

\begin{cor}
La función $Q_c$ es comonotona sobre $E$
\end{cor}

\begin{proof}
Para probar comonotonicidad tenemos que ver que si $x$ en $X$ no pertenece a $Q_c(X)$ entonces esta no pertenece a $Q_c(Y)$ cuando $X$ es un subconjunto de $Y$. Esto sale directo del algoritmo glotón porque si durante la construcción de $Q_c(X)$ no se eligió $x$ entonces es claro que durante la construcción de $Q_c(Y)$ tampoco se eligió a $x$ por las mismas razones. 
\end{proof}

Para relacionar todo esto con la idea de estabilidad igual que en los demás capítulos es necesario enunciar algunas definiciones.

\begin{dfn}
Para un conjunto $X$ en $E$ y una arista $e$ en $E$, decimos que $X$ \textbf{domina} a $e$ en el lado del aspirante si $e$ no pertenece a $Q_A(X \cup \{e\})$ y decimos que $X$ domina a $e$ en el lado de las universidades si $e$ no pertenece a $Q_C(X \cup \{e\})$.

En particular, para los dos tipos de dominancia si tenemos dos subconjuntos $X,Y$ de $E$. Decimos que $X$ domina a $Y$ si $X$ domina a todos los elementos $Y \setminus X$.
\end{dfn}

A partir de esta definición podemos realizar la siguiente observación:

\begin{obs}
Un subconjunto $M$ de $E$ es estable si y solo si domina a $E\setminus M$ (bajo cualquiera de los dos esquemas).
\end{obs}

\section{Resultados estructurales}

A partir de todos los resultados anteriores, en lo que sigue del capítulo mostraremos varios resultados importantes sobre el problema que se relacionan de forma general con el resto de la tesis. 

\begin{teo}
\label{union de estables}
Si $M_1,M_2,\dots,M_k$ son asignaciones estables entonces $Q_C( \cup_{i=1}^k M_i)$ y $Q_A( \cup_{i=1}^k M_i)$ son también asignaciones estables. 
\end{teo}
\begin{proof}
Dado un conjunto $X$ en $E$ sabemos que $Q_C(X)$ es el subconjunto maximal universidad independiente de $X$, como $X$ es la unión de varias asignaciones podemos ver qué $Q_C(X)$ es también una asignación y además podemos ver por propiedades del algoritmo glotón que este domina a las demás asignaciones en el sentido de las universidades, por lo tanto $Q_C( \cup_{i=1}^k M_i)$ es una asignación estable. En particular, el argumento es idéntico para demostrar que $Q_A( \cup_{i=1}^k M_i)$ es una asignación estable. 
\end{proof}

\begin{lem}
\label{desigualdad}
Si tenemos dos asignaciones estables $M_1,M_2$ si $M_1$ domina a $M_2$ en el lado de las universidades esto se cumple si y solo si $M_2$ domina a $M_1$ en el lado de los solicitantes.
\end{lem}
\begin{proof}
Si $M_1$ y $M_2$ son emparejamientos estables entonces cada uno de ellos domina al otro. Si $M_1$ domina a cada elemento de $M_2 \setminus M_1$ en el lado de la universidad entonces $M_2$ debe de dominar a cada elemento de $M_1 \setminus M_2$ en el lado de los aspirantes. A partir de esta observación podemos ver que el lema es verdadero.
\end{proof}

\begin{teo}
Si $M_1,M_2,\dots,M_n$ son asignaciones estables y $k$ es un entero arbitrario entre $1$ y $n$. Si cada aspirante escoge su $k$ mejor opción entre todas las asignaciones entonces el conjunto $M$ con esas asignaciones es estable. 
\end{teo}
\begin{proof}
Sea $a$ un solicitante arbitrario y sean $M_a^1,M_a^2,\dots,M_a^n$ los $n$ emparejamientos respecto al orden de preferencias de $a$. Además, para facilitar la notación sean $M_a = Q_C(\cup_{i=1}^kM_a^i)$ y $M=Q_A( \cup_{a \in A} M_a)$.

Por el teorema \ref{union de estables} sabemos que todos los $M_a$ y que $M$ son asignaciones estables. Observamos que para todo aspirante $a$ tenemos que $M_a$ domina por las universidades a $M_1,M_2,\dots,M_k$ y por lo tanto $M_1,M_2,\dots,M_k$ dominan a $M_a$ por los aspirantes (lema \ref{desigualdad}). Esto implica que en $M_a$ cada solicitante $a'$ en $A$ recibe su peor asignación dentro de $M_a^1,M_a^1,\dots,M_a^k$ y por lo tanto $a$ recibe su $k$ mejor opción en $M_a$ y los demás aspirantes reciben una asignación que no es más preferida que su mejor $k$ opción. Ahora, al construir $M$ cada aspirante escoge su mejor opción entre todos los $M_a$ lo que implicaría que cada aspirante escoge su mejor $k$ opción entre $M_1,M_2,\dots,M_n$. Como $M$ es estable por el teorema \ref{union de estables} entonces podemos concluir que si cada solicitante escoge su $k$ mejor opción entre todas las asignaciones entonces el conjunto $M$ con esas asignaciones es estable.
\end{proof}

Para la siguiente parte necesitamos agregar una definición sobre las matroides que nos va a ayudar a generar muchos resultados sobre este problema.

\begin{dfn}
Si $M=(\mathcal{F},S)$ es una matroide, decimos que un elemento $e$ en $S$ es \textbf{abarcado} por un subconjunto $E$ de $S$ si $e$ pertenece a $E$ o si existe un subconjunto independiente $E'$ de $E$ con la propiedad de que todo subconjunto propio de $E' \cup \{e\}$ es independiente. Esto es equivalente a decir que existe $E'$ tal que el rango de $E'$ independiente es igual al rango de $E' \cup \{e\}$. 
\end{dfn}

En nuestra matroide sobre las universidades esta definición es equivalente a decir que si un emparejamiento estable $M$ domina a $e$ entonces $e$ es abarcado por $M$. El siguiente resultado está altamente relacionado con los mencionados anteriormente. 

\begin{teo}
\label{help}
Para todo par de asignaciones estables $M,M'$ se tiene que los elementos abarcados por $M$ y $M'$ en la matroide de las universidades son exactamente los mismos. El resultado además es análogo en la matroide de los aspirantes. 
\end{teo}

A partir de este resultado podemos ver varios resultados bastante interesantes, primero necesitamos definir qué significa que una universidad este débilmente libre. 

\begin{dfn}
Decimos que un conjunto acotado $C_i$ es \textbf{débilmente libre} relativo a un conjunto de aristas $E'$ si ningún conjunto acotado que contiene propiamente a $C_i$ está lleno relativo a $E'$ (no existe ningún emparejamiento en $E'$ con la propiedad de que acote a $C_i$). Además, decimos que $C_i$ está \textbf{esencialmente libre} relativo a $E'$ si $C_i$ está débilmente libre relativo a $E'$ y ningún subconjunto de $C_i$ está lleno relativo a $E'$. 
\end{dfn}

Con esta definición podemos generalizar el teorema de los hospitales rurales. 

\begin{teo}{Generalización del teorema de los hospitales rurales}

Sea $M$ un emparejamiento. Si un conjunto acotado $C_i$ está débilmente libre relativo a $M$ para todo emparejamiento estable $M'$. Entonces, tenemos que la cantidad de personas asignadas a $C_i$ es idéntica en $M'$ y en $M$. Además, si $C_i$ es esencialmente libre relativo a $M$ entonces las personas asignadas a $C_i$ en $M$ y $M'$ son las mismas. 
\end{teo}

\begin{proof}
La demostración se hace por inducción sobre $i$ el número de conjuntos acotados. Supongamos que la familia de conjuntos $\mathcal{C}=\{C_1,C_2,\dots,C_m\}$ son los conjuntos de universidades que son parte de una restricción y supongamos que si $C_i$ es un subconjunto de $C_j$ es porque $i$ es menor igual que $j$. 

Supongamos que $i=1$ entonces hay tres opciones:
\begin{enumerate}
\item $C_1$ no es débilmente libre entonces por vacuidad el teorema es verdadero.
\item $C_1$ es libre y por lo tanto es débilmente libre. Además, como $C_1$ no contiene ningún conjunto acotado entonces también es relativamente libre. Esto quiere decir que si $A$ son los elementos abarcados por $M$ y $B$ son las aristas que inciden en $C_1$ entonces la intersección entre $A$y $B$ es igual a los elementos asignados a $C_1$ en $M$. Por el teorema \ref{help} tenemos que los elementos abarcados por $M$ intersectados con las aristas incidentes en $C_1$ son iguales a los elementos abarcados por $M'$ intersectados con las aristas incidentes en $C_1$ y, por lo tanto, las personas asignadas a $C_i$ en $M$ y $M'$ son las mismas. Por lo tanto, el teorema es cierto. 
\item Supongamos que $C_1$ está lleno relativo a $M$ y supongamos que la cantidad de personas asignadas a $C_1$ en $M$ es distinta a las asignadas en $M'$. Como $C_1$ está llena en $M$ tenemos que la cantidad de personas asignadas en $M'$ es estrictamente menor que su cota superior y por lo tanto $C_1$ esta subsuscrito en $M'$, lo que implica que $C_1$ no puede estar lleno en ningún emparejamiento estable y llegamos a una contradicción porque $C_1$ está llena en $M$. Por lo tanto, el teorema en este caso es cierto. 
\end{enumerate}
Ahora, tomamos una $i$ fija y supongamos que el teorema es cierto para cada $C_j$ con $j$ estrictamente menor que $i$. 

Si $C_i$ no contiene a ningún otra $C_k$ entonces la demostración es idéntica al caso de $i=1$. De otra forma, sea $C_i = C^1 \cup C^2 \cup \cdots \cup C^k$, donde $C^1,C^2,\dots,C^k$ son los conjuntos acotados más grandes con la propiedad de que son ajenos. Podemos suponer que $C_i$ es débilmente libre porque esa es una de las hipótesis del teorema. 

Supongamos que $C_i$ está subsuscrito relativo a $M$, entonces tenemos que cada uno de los $C^1,C^2,\dots,C^k$ son débilmente libres. Por hipótesis de inducción el teorema es cierto para cada $C^j$ entonces tenemos que para $C^j$ se cumple que la cantidad de personas asignadas a $M$ es idéntica a la cantidad de personas asignadas al conjunto en $M'$. Si sumamos sobre todos los conjuntos obtenemos que la cantidad de personas asignadas a $C_i$ en $M$ es igual a la cantidad de personas asignadas a $C_i$ en $M'$. 

Simultáneamente si suponemos que $C_i$ es relativamente libre respecto a $M$ tenemos que $C^1,C^2,\dots,C^k$ también son relativamente libres respecto a $M$. Para cada $C^j$ tenemos que las personas asignadas a ella son igual en los dos emparejamientos. Si unimos todos los conjuntos tenemos que las personas asignadas a $C_i$ son las mismas en $M$ y en $M'$. 

Si $C_i$ está lleno se puede usar el mismo argumento que en el caso de $i=1$ para mostrar que está lleno en los emparejamientos y por lo tanto en este caso el teorema es cierto. 

Por lo tanto, si un conjunto acotado $C_i$ está débilmente libre relativo a $M$ para todo emparejamiento estable $M'$. Entonces, tenemos que la cantidad de personas asignadas a $C_i$ es idéntica en $M'$ y en $M$. Además, si $C_i$ es esencialmente libre relativo a $M$ entonces las personas asignadas a $C_i$ en $M$ y $M'$ son las mismas. 

\end{proof}

Esta demostración sirve para demostrar también el caso planteado en el capítulo 4. Un resultado directo de este teorema y de vital importancia es el siguiente que muestra como los grupos en el caso de cotas anidadas se comportan igual al caso sin cotas comunes como si fueran universidades por sí solos.

\begin{cor}
Si $C_i$ es un grupo entonces cada emparejamiento estable asigna la misma cantidad de personas a $C_i$.
\end{cor}

\begin{proof}
Un grupo por definición es débilmente libre y por lo tanto cada emparejamiento estable asigna la misma cantidad de personas a $C_i$.
\end{proof}

Notamos que la gran mayoría de estos resultados se cumplen para varios de los subproblemas que planteamos en la tesis y que problemas como el del matrimonio estable cumplen la misma estructura de matroide y por lo tanto todos los resultados de este capítulo.






