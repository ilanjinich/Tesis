\chapter{Funciones de elección y matroides}

En este capítulo redefineremos el problema de admisión a universidades con cotas comunes anidadas por completo, vamos a suponer que tenemos una gráfica bipartita $G=(A\cup C,E)$ donde los vertices en $A$ representan los aplicantes, $C$ representa las universidades y las aristas de la gráifca $E$ representan las asignaciónes aceptables entre los dos. Para definir el problema de esta manera es necesario definir dos funciones de elección. 

\begin{dfn}
Decimos que $Q$ es una \textbf{función de elección} para un conjunto $E$ si para todo subconjunto $X$ de $E$ tenemos que $Q(X)$ es un subconjunto de $X$.  En particular, decimos que un conjunto $X$ es \textbf{$Q$-independiente} si $Q(X)=X$.
\end{dfn}

Definimos dos funciones de elección (las dos sobre ) una para los aplicantes denotado como $Q_A$ y otra para las universidades denotada como $Q_C$. Si un conjunto es $Q_A$-independiente decimos que es \textbf{aplicante independiente} y si un conjunto es $Q_C$-independiente decimos que es \textbf{universidad independiente}. Las dos funciones se definen de la siguiente manera:

\begin{dfn}
Definimos la \textbf{función de elección para los aplicantes} $Q_A$ como una función de elección sobre las aristas que cumple que para todo aplicante $a$, $Q_A(X)$ contiene a las aristas en $X$ que el aplicante escogería si podria elegir libremente sin importar las restricciones. En particular, si $X$ contiene  varias aristas que inciden sobre $a$ este elige la que incide en la universidad más alta en su lista. 
\end{dfn} 

\begin{dfn}
Definimos la \textbf{función de elección para las universidades} $Q_C$ como una función de elección sobre las aristas que cumple que para toda universidad $c$, $Q_C(X)$ contiene a las aristas $X$ que inciden en los aplicantes que a $c$ elegiria si podria elegir libremente. En particular se sigue el siguiente algoritmo, supongamos que la familia de conjuntos $\mathcal{C}=\{C_1,C_2,\dots,C_m\}$ son los conjuntos de universidades que son parte de una restricción y supongamos que si $C_i$ es un subconjunto de $C_j$ es porque $i$ es menor igual que $j$:

\IncMargin{1em}
\begin{Algoritmo}[H]
%\SetKwData{Left}{left}\SetKwData{This}{this}\SetKwData{Up}{up}
%\SetKwFunction{Union}{Union}\SetKwFunction{FindCompress}{FindCompress}
\SetKwInOut{Input}{input}\SetKwInOut{Output}{output}

\Input{Un conjunto $X$ de aristas en $E$, la familia de conjuntos con restricciones $\mathcal{C}$ con sus cotas y las listas de preferencias para las universidades y los aplicantes}
\Output{$Q(X)$}
\BlankLine
\emph{Sea $X_0 = X$;}

\For{i = 1, \dots, m}{

\emph{$X_i$ es el conjunto obtenido por aplicarle la cota superior de $C_i$ a $X_{i-1}$;}

\emph{Sea $X_{i-1}(C_i)$ las aristas en $X_{i-1}$ que inciden en $C_i$;}

\emph{Sean $X'_{i}$ las aristas en $X_{i-1}(C_i)$ que no se encuentran entre las $q(C_i)$ mejores;}

\emph{$X_i = X_{i-1}\setminus X'_i$;}
}

\emph{$Q_C(X) = X_m$;}
\caption{Algoritmo para calcular la función de elección para las universidades}
\end{Algoritmo}
\DecMargin{1em}

Para la linea 5 del algoritmo es necesario explicar cuál es el orden de preferencias de $X_{i-1}(C_i)$, este depende de dos casos: 
\begin{enumerate}
\item Dadas dos aristas que inciden en dos aplicantes distintos escogemos de acuerdo al orden de preferencias de $C_i$.
\item Dadas dos aristas que inciden en el mismo aplicante $a$ escogemos de acuerdo al orden de preferencias de $a$.
\end{enumerate}

\end{dfn}

Apartir de esto realizamos las siguientes observaciones sobre el comportamiento de estas funciones:

\begin{obs}
$Q_A(X)=X$ si y solo si cada aplicante tiene a lo más una arista incidente en $X$.
\end{obs}
\begin{obs}
$Q_C(X)=X$ si y solo si $X$ no viola ninguna cota de $\mathcal{C}$.
\end{obs}

Usando esas dos observaciones concluimos que:

\begin{obs}
$M$ es un emparejamiento si y solo si $M$ es aplicante independiente y universidad independiente. 
\end{obs}

\begin{teo}
\label{matroid}
Sea $I_C$ el subconjunto del conjunto potencia de $E$ que contiene a todos los conjuntos universidad independiente, entonces $M_C = (E,I_C)$ es una matroide. 
\end{teo}

\begin{proof}
Usamos inducción sobre el número $m$ de conjuntos acotados (la cardinalidad de $\mathcal{C}$). Si $m=1$ entonces solo tenemos una universidad y $M_C$ es una matroide uniforme de rango $q(C_i)$. 
Supongamos que el resultado es valido si la cantidad de conjuntos es estrictamente menor que $m$ y supongamos que tenemos $m$ conjuntos acotados. Sean $C^1,C^2,\dots, C^k$ los grupos en $\mathcal{C}$, entonces tenemos dos casos:
\begin{enumerate}
\item Si $k$ es mayor estricto que $1$ entonces cada $I_{C^i}$ forma una matroide y como $I_C$ es la suma directa de los $I_{C^i}$ entonces es una matroide. 
\item Si $k=1$ entonces $I'$ es el conjunto de conjuntos universidad independientes de acuerdo a $C_1,\dots,C_{m-1}$ es parte de una matroide. Podemos ver que al aplicar la restricción de $C_m$ tenemos un truncamiento sobre la misma y no se pierde el hecho de que es una matroide. 
\end{enumerate}
Por lo tanto, si $I_C$ el subconjunto del conjunto potencia de $E$ que contiene a todos los conjuntos universidad independiente, entonces $M_C = (E,I_C)$ es una matroide.
\end{proof}

\begin{cor}
Sea $I_A$ el subconjunto del conjunto potencia de $E$ que contiene a todos los conjuntos universidad independiente, entonces $M_A = (E,I_A)$ es una matroide. 
\end{cor}
\begin{proof}
Como los aplicantes pueden ser vistos como universidades en donde cada cota superior es 1 y no existen cotas comunes, de acuerdo al teorema \ref{matroid}  $M_A = (E,I_A)$ es una matroide. 
\end{proof}

\begin{cor}
La función de elección para las universidades para $X$ encuentra el subconjunto maximal universidad independiente de $X$.
\end{cor}

\begin{proof}
Para cada elemento $x$ en $X\setminus Q_c(X)$ tenemos que el conjunto $\{x\}\cup Q_c(X)$ es no universidad independiente, por definición existe un conjunto acotado $C_i$ tal que $Q_c(X)$ contiene $q(C_i)$ aristas incidentes en $C_i$ que todas son preferidas a $x$.
\end{proof}

A partir de esto podemos ver dos observaciónes.

\begin{obs}
La cardinalidad de $Q_c(X)$ es igual al rango de $X$ respecto a la matroide de las universidades $M_c$. 
\end{obs}

\begin{obs}
La función $Q_c$ es creciente. Es decir, si $X$ es un subconjunto de $Y$ entonces la cardinalidad de $Q_c(X)$ es menor o igual que la cardinalidad de $Q_c(Y)$. \footnote{La demostración es directo del 9.8 } %Me falta referencia
\end{obs} 

Estas dos observaciones implican que la función $Q_c$ puede ser construida a partir del algoritmo gloton de la siguiente manera:

\IncMargin{1em}
\begin{Algoritmo}[H]
%\SetKwData{Left}{left}\SetKwData{This}{this}\SetKwData{Up}{up}
%\SetKwFunction{Union}{Union}\SetKwFunction{FindCompress}{FindCompress}
\SetKwInOut{Input}{input}\SetKwInOut{Output}{output}

\Input{Un conjunto $X$ de aristas en $E$, la familia de conjuntos con restricciones $\mathcal{C}$ con sus cotas y las listas de preferencias para las universidades y los aplicantes}
\Output{$Q(X)$}
\BlankLine
\emph{Acomodamos las aristas de $E=e_1,e_2,\dots,e_n$ de cierta forma que si dos aristas $e_i$ y $e_j$ pertenecen al mismo conjunto acotado entonces si $i$ es menor que $j$ implica que $e_i$ es preferido a $e_j$;}

\emph{Sea $E_i$ las aristas $e_j$ con $j$ menor o igual que $i$;}

\emph{Sea $X_i=X \cap E_i$;}

\emph{Sea $X_0 = \emptyset$;}

\For{$i = 1, \dots, n$}{
\If{$Q_c(X_i)\cup\{e_{i+1}\}$  es universidad independiente}{
\emph{$Q_c(X_i+1)=Q_c(X_i)\cup\{e_{i+1}\}$}

\Else{$Q_c(X_{i+1})=Q_c(X_i)$;}
}
}

\caption{Algoritmo alternativo para calcular la función de elección para las universidades}
\end{Algoritmo}
\DecMargin{1em}

El algoritmo es igual porque por el teorema () sabemos que el algoritmo gloton siempre converge al subconjunto maximal universidad independiente de $X$ y por lo tanto aplicar uno u otro algoritmo debe de llevar al mismo resultado. Además de esto si agregamos podemos ver que la función es comonotona, es decir:

\begin{dfn}
Decimos que una función $Q$ en $E$ es \textbf{comonotona} si para cualquier par de subconjuntos $X,Y$ de $E$ con $X$ siendo un subconjunto de $Y$ tenemos que $X\setminus Q(X)$ es un subconjunto de $Y\setminus Q(Y)$.
\end{dfn}

\begin{cor}
La función $Q_c$ es comonotona sobre $E$
\end{cor}

\begin{proof}
Para probar comonotonicidad tenemos que ver que si $x$ en $X$ no pertenece a $Q_c(X)$ entonces esta no pertenece a $Q_c(Y)$ cuando $X$ es un subconjunto de $Y$. Esto sale directo del algoritmo gloton porque durante la construcción de $Q_c(X)$ no se eligio $x$ entonces es claro que durante la construcción de $Q_c(Y)$ tampoco se eligio a $x$ por las mismas razones. 
\end{proof}

Para relacionar todo esto con la idea de de estabilidad igual que en los demas capitulos es necesario enunciar algunas definiciones.

\begin{dfn}
Para un conjunto $X$ en $E$ y una arista $e$ en $E$, decimos que $X$ \textbf{domina} a $e$ en el lado del aplicante si $e$ no pertenece a $Q_A(X \cup \{e\})$ y decimos que $X$ domina a $e$ en el lado de las universidades si $e$ no pertenece a $Q_C(X \cup \{e\})$.

En particular, para los dos tipos de dominancia si tenemos dos subconjuntos $X,Y$ de $E$. Decimos que $X$ domina a $Y$ si $X$ domina a todos los elementos $Y \setminus X$.
\end{dfn}

A partir de este definición podemos realizar la siguiente observación:

\begin{obs}
Un subconjunto $M$ de $E$ es estable si y solo si domina a $E\setminus M$ (bajo cualquiera de los dos esquemas).
\end{obs}

\section{Resultados estructurales}

A partir de todos los resultados en lo que sigue del capítulo mostraremos varios resultados importantes sobre el problema que se relacionan de forma general con el resto de la tesis. 

\begin{teo}
\label{union de estables}
Si $M_1,M_2,\dots,M_k$ son asignaciones estables entonces $Q_C( \cup_{i=1}^k M_i)$ y $Q_A( \cup_{i=1}^k M_i)$ son tambien asignaciones estables. 
\end{teo}
\begin{proof}
Dado un conjunto $X$ en $E$ sabemos que $Q_C(X)$ es el subconjunto maximal universidad independiente de $X$, como $X$ es la union de varias asignaciones podemos ver que $Q_C(X)$ es tambien una asignacón y además podemos ver por propiedades del algoritmo glotón que este domina a las demás asignaciones en el sentido de las universidades, por lo tanto  $Q_C( \cup_{i=1}^k M_i)$ es una asignación estable. En particular, el argumento es identico para demostrar que $Q_A( \cup_{i=1}^k M_i)$ es una asignación estable. 
\end{proof}

\begin{lem}
\label{desigualdad}
Si tenemos dos asignaciones estables $M_1,M_2$ si $M_1$ domina a $M_2$ en el lado de las universidades esto se cumple si y solo si $M_2$ domina a $M_1$ en el lado de los aplicantes.
\end{lem}
\begin{proof}
Si $M_1$ y $M_2$ son emperajemientos estables entonces cado uno de ellos domina al otro. Si $M_1$ domina a cada elemento de $M_2 \setminus M_1$ en el lado de la universidad entonces $M_2$ debe de dominar a cada elemento de $M_1 \setminus M_2$ en el lado de los aplicantes. A partir de esta observación podemos ver que el lema es verdadero.
\end{proof}

\begin{teo}
Si $M_1,M_2,\dots,M_n$ son asignaciones estables y $k$ es un entero arbitrario entre $1$ y $n$. Si cada aplicante escoge su $k$ mejor opción entre todas las asignaciónes entonces el conjunto $M$ con esas asignaciones es estable. 
\end{teo}
\begin{proof}
Sea $a$ un aplicante arbitrario y sean $M_a^1,M_a^2,\dots,M_a^n$ los $n$ emparejamientos respecto al orden de preferencias de $a$. Además para facilitar la notación sean $M_a = Q_C(\cup_{i=1}^kM_a^i)$ y  $M=Q_A( \cup_{a \in A} M_a)$.

Por el teorema \ref{union de estables} sabemos que todos los $M_a$ y que $M$ son asignaciones estables. Observamos quer para todo aplicante $a$ tenemos que $M_a$ domina por las universidades a $M_1,M_2,\dots,M_k$ y por lo tanto $M_1,M_2,\dots,M_k$ dominan a $M_a$ por los aplicantes (lema \ref{desigualdad}). Esto implica que en $M_a$ cada aplicante $a'$ en $A$ recibe su peor asignación dentro de $M_a^1,M_a^1,\dots,M_a^k$ y por lo tanto $a$ recibre su $k$ mejor opción en $M_a$ y los demás aplicantes reciben una asignación que no es más preferida que su mejor $k$ opción. Ahora, al construir $M$ cada aplicante escoge su mejor opción entre todos los $M_a$ lo que implicaria que cada aplicante escoge su mejor $k$ opción entre $M_1,M_2,\dots,M_n$. Como $M$ es estable por el teorema \ref{union de estables} entonces podemos concluir que si cada aplicante escoge su $k$ mejor opción entre todas las asignaciónes entonces el conjunto $M$ con esas asignaciones es estable.
\end{proof}

%Aquí me falta ver que pedo con spans



