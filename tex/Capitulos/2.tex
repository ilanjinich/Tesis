\chapter{Un poco sobre computación}

Introducción.... (hablar un poco sobre la importancia de la computación en IDO)

\begin{dfn}
Qué es un algoritmo
\end{dfn}

\begin{dfn}
Orden de un algoritmo
\end{dfn}

Intro a clases de complejidad


\section{Clases de complejidad}
Algo que nos interesa mucho cuando atacamos un problema de computación es que tan dificíl es de resolver y si existe una mejor manera de hacerlo. La complejidad computacional nos permite estudiar esto. Para encarrilarnos hay que empezar con algunas definiciones y ejemplos.
\begin{dfn}
Se dice que un problema es un problema de decisón si admite dos posibles respuestas ``Si'' o ``No''. Resolverlo es encontrar todas las soluciones en las que la respuesta es ``Si''.
\end{dfn}
Existen dos clases de problemas de decisión que nos interesan, los que son faciles de verificar (NP) y los que son faciles de resolver (P)-
\begin{dfn}
Un problema esta en la clase NP si existe un algoritmo polinomial para verificar que la respuesta de una solución dada es ``Si''.
\end{dfn}

\begin{eje}[Problema del clima]
Supongamos que Mariana le dice a Adrián que esta lloviendo, si Adrián decide verificar si esto es cierto solo tiene que realizar tres pasos:
\begin{enumerate}
\item Caminar a la ventana.
\item Abrir la ventana.
\item Ver el cielo.
\end{enumerate}
Por lo tanto a Adrián le toma un tiempo constante verificar si esta lloviendo lo que implica que este problem esta en NP.
\end{eje}
\begin{dfn}
Se dice que un problema de decisión esta en P si existe un algoritmo polinomial que lo resuelve.
\end{dfn}
\begin{obs}[$P  \subseteq NP$]
Si un problema de decisión pertenece a P entonces pertenece a NP.
\end{obs}
%\begin{proof}
%Supongamos que $A$ es un problema de decision en P y $b$ es una solución dada de $A$, podemos verificar si la respuesta de la solución es ``Si'' resolviendo el problema en tiempo polinomial.
%\end{proof}
\begin{obs}
\label{pp eq}
Si A es un problema de decisión en P y existe un algoritmo polinomial que reduce resolver el problema de decisión B a resolver el problema A entonces B esta en P.
\end{obs}

Vale la pena mencionar algunos ejemplos de problemas que estan en P:
\begin{enumerate}
\item El problema de admisión a universidades.
\item Verificar si un numero es primo.
\item Encontrar la ruta más corta entre dos vertices en una gráfica.
\item El problema de la ruta critica. 
\end{enumerate}

Una conjetura famosa que vale la pena mencionar es ¿si $P=NP$?, es decir si todo problema en NP tambien se enecuentra en P. La persona que lo demuestra además de ganar fama y trabajo va a ser acredor a un permio de un millon de dolares. 

\section{El $3-SAT$}
Un problema de gran importancia para entender las clases de complejidad es el $3-SAT$, la raíz de todo el estudio necesario para estudiar el problema ¿$P=NP?$ se remonta a enteder este problema. 
\begin{dfn} [conjunctive normal form.]
\end{dfn}


\section{Problemas NP-Completos}
\begin{flushright}
\textit{``Chaos is a ladder'}
\end{flushright}
\begin{dfn}
Qué es un problema NP-completo. 
\end{dfn}
\begin{teo}[Teorema de Cook-Levin]
El 3-SAT es NP-Completo. Chance no
\end{teo}
Donde esta la demostración. 

\begin{eje}
Ejemplos con y sin demostración. 
\end{eje}
