\chapter{Un poco sobre computación}

El objetivo de este capítulo es introducir la idea de complejidad algorítmica. Se habla de qué es un algoritmo, de cuál es el orden de complejidad de un algoritmo y de algunas clases de complejidad de algoritmos. Para empezar a introducir estas ideas son necesarias varias definiciones. 

\begin{dfn} 
Definimos un \textbf{algoritmo} como una serie de pasos para resolver un problema. El \textbf{tamaño} de un problema es la cantidad de datos necesarios para resolver el problema (las entradas del algoritmo). La \textbf{complejidad} de un problema es una función $f$, donde $f(n)$ es la cantidad de operaciones necesarias para que el algoritmo termine. Además, decimos que $f(n)$ es de orden a lo más $g(n)$ si para alguna $K>0$ se cumple que
$$f(n) \leq Kg(n)$$
para todo $n$ natural.
\end{dfn}

A partir de esto podemos definir la clase de algoritmos polinomiales, los cuales son muy importantes para entender el contenido de esta tesis. 

\begin{dfn}
Decimos que un algoritmo es \textbf{polinomial} si el algoritmo es de orden a lo más $n^k$ para alguna $k$ en los enteros. 
\end{dfn}


\section{Clases de complejidad}
Algo que nos interesa mucho cuando atacamos un problema de computación es que tan difícil es de resolver y si existe una mejor manera de hacerlo. La complejidad computacional nos permite estudiar esto. Para encarrilarnos hay que empezar con algunas definiciones y ejemplos.
\begin{dfn}
Se dice que un problema es un problema de decisión si admite dos posibles respuestas ``Si'' o ``No''. 
\end{dfn}
Existen dos clases de problemas de decisión que nos interesan, los que son fáciles de verificar (NP) y los que son fáciles de resolver (P).
\begin{dfn}
Un problema está en la clase NP si existe un algoritmo polinomial para verificar que la respuesta de una solución dada es ``Si''.
\end{dfn}

\begin{eje}[Problema del clima]
Supongamos que Mariana le dice a Adrián que está lloviendo, si Adrián decide verificar si esto es cierto solo tiene que realizar tres pasos:
\begin{enumerate}
\item Caminar a la ventana.
\item Abrir la ventana.
\item Ver el cielo.
\end{enumerate}
Por lo tanto, a Adrián le toma un tiempo constante verificar si está lloviendo lo que implica que este problema está en NP.
 \fin
\end{eje}
\begin{dfn}
Se dice que un problema de decisión está en P si existe un algoritmo polinomial que lo resuelve.
\end{dfn}
\begin{obs}[$P \subseteq NP$]
Si un problema de decisión pertenece a P entonces pertenece a NP.
\end{obs}
%\begin{proof}
%Supongamos que $A$ es un problema de decisión en P y $b$ es una solución dada de $A$, podemos verificar si la respuesta de la solución es ``Si'' resolviendo el problema en tiempo polinomial.
%\end{proof}
\begin{obs}
\label{pp eq}
Si A es un problema de decisión en P y existe un algoritmo polinomial que reduce resolver el problema de decisión B a resolver el problema A entonces B está en P.
\end{obs}

Vale la pena mencionar algunos ejemplos de problemas que están en P:
\begin{enumerate}
\item Verificar si un número es primo (ver \cite{agrawal2004primes}).
\item Encontrar la ruta más corta entre dos vértices en una gráfica.
\item El problema de la ruta crítica. 
\end{enumerate}

Una conjetura famosa que vale la pena mencionar es si $P=NP$, es decir si todo problema en NP también se encuentra en P. La persona que lo demuestre además de ganar fama y trabajo va a ser acreedor a un premio de un millón de dólares. 


\section{Problemas NP-Completos}

\begin{dfn}
Decimos que $A$ un problema es \textbf{NP-Completo} si para todo problema $B$ en NP existe una reducción polinomial que reduce el resolver $B$ a resolver $A$.
\end{dfn}

Una conclusión inmediata de este problema es la siguiente: 

\begin{obs}
Si existe algún algoritmo polinomial para resolver un problema NP-Completo entonces $P=NP$.
\end{obs}

Algunos ejemplos de problemas NP-Completos son: 

\begin{eje}
\begin{itemize}
\item El problema del agente viajero. 
\item El problema de 3-coloración. 
\item El problema de las galerías de arte.
\end{itemize}
\fin
\end{eje}


En el siguiente capítulo se reduce el problema planteado en el capítulo 1 a su caso más simple, conocido como el problema del matrimonio estable.
