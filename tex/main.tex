%Plantilla basada en "Template for Masters / Doctoral Thesis" (plantilla disponible en writeLaTex) que subió LaTeXTemplates.com

\documentclass[11pt]{book}
\usepackage[paperwidth=17cm, paperheight=22.5cm, bottom=2.5cm, right=2.5cm]{geometry}
\usepackage{amssymb,amsmath,amsthm} %paquete para símbolo matemáticos
\usepackage[spanish]{babel}
\usepackage[utf8]{inputenc} %Paquete para escribir acentos y otros símbolos directamente
\usepackage{enumerate}
%\usepackage{graphicx}
\usepackage[pdftex]{color,graphicx}   
\usepackage{tikz}
\usepackage{float}
\usepackage{url}
%\usepackage[]{algorithm2e}
\usepackage[linesnumbered,ruled,vlined]{algorithm2e}



\newenvironment{Algoritmo}[1][htb]
  {\renewcommand{\algorithmcfname}{Algoritmo}% Update algorithm name
   \begin{algorithm}[H]%
  }{\end{algorithm}}
  
 \usepackage{algcompatible}

\usetikzlibrary{arrows}
\usetikzlibrary{quotes}
\usetikzlibrary{shapes,snakes}
\tikzset{cross/.style={cross out, draw=black, minimum size=2*(#1-\pgflinewidth), inner sep=0pt, outer sep=0pt},
%default radius will be 1pt. 
cross/.default={10pt}}
\newcommand{\tcancel}[2][black]{%
\begin{tikzpicture}
\node[draw=#1,cross out,inner sep=1pt] (a){#2};
\end{tikzpicture}%
}


% Gian Carlo Diluvi's preamble ---------
\usepackage{amsthm}
\usepackage{amsmath}
\usepackage[nottoc]{tocbibind}
\usepackage{apacite} % Citar con APA
%\usepackage[numbers]{natbib}
\usepackage{natbib}
%\usepackage{biblatex}
%\bibliography{bibliografia}
\usepackage{here}
\usepackage{float}
\usepackage{etoolbox}
\usepackage{url}
\usepackage{mathrsfs}
\usepackage{commath}
\usepackage{color}
\usepackage{multirow}
%\usepackage{subcaption}



%\usepackage{subfig} %para poner subfiguras
\graphicspath{{Img/}} %En qué carpeta están las imágenes
\usepackage[nottoc]{tocbibind}

\usepackage{listings}
	\lstset{frame = single,
    		literate={á}{{\'a}}1
        			 {ã}{{\~a}}1
        			 {é}{{\'e}}1
        			 {ó}{{\'o}}1
        			 {í}{{\'i}}1
        			 {ñ}{{\~n}}1
        			 {¡}{{!`}}1
        			 {¿}{{?`}}1
        			 {ú}{{\'u}}1
                     {Á}{{\'A}}1
                     {É}{{\'E}}1
                     {Í}{{\'I}}1
                     {Ó}{{\'O}}1
                     {Ú}{{\'U}}1
}
\renewcommand{\lstlistingname}{C\'odigo}
\lstdefinestyle{R}{%
%\captionsetup{labelformat=algocaption,labelsep=colon}
    mathescape=false,
    breaklines=true,
    frame=single,
    numbers=left, 
    numberstyle=\small,
    language=R,
    basicstyle=\scriptsize\ttfamily, 
    keywordstyle=\color{black}\bf,
    xleftmargin=.04\textwidth,
}

\usepackage[pdftex,
            pdfauthor={Ilan Jinich Fainsod},
            pdftitle={El problema de admisión a universidades con cotas inferiores},
            pdfsubject={ÁREA DE LA TESIS},
            pdfkeywords={PALABRAS CLAVE},
            pdfproducer={Latex con hyperref},
            pdfcreator={pdflatex}]{hyperref}



\begin{document}

%----------------------------------------------------------------------------------------
%	COMANDOS PERSONALIZADOS
%----------------------------------------------------------------------------------------

%SI TU TESIS TIENE TEOREMAS Y DEMOSTRACIONES, PUEDES DESCOMENTAR Y USAR LOS SIGUIENTES COMANDOS

\renewcommand{\proofname}{Demostración}
%\providecommand{\norm}[1]{\lVert#1\rVert} %Provee el comando para producir una norma.
%\providecommand{\innp}[1]{\langle#1\rangle} 
%\newcommand{\seno}{\mathrm{sen}}
%\newcommand{\diff}{\mathrm{d}}


\newtheorem{teo}{Teorema}[chapter] 
\newtheorem{cor}[teo]{Corolario}
\newtheorem{lem}[teo]{Lema}
%\renewcommand{\qedsymbol}{$\blacksquare$}
%\renewcommand{\qedsymbol}{$\dag$}}
\newcommand{\fin}{\hfill$\triangle$}

\theoremstyle{definition}
\newtheorem{dfn}[teo]{Definición}

\theoremstyle{remark}
\newtheorem{obs}[teo]{Observación}

\newtheorem{eje}{Ejemplo}[chapter]

\allowdisplaybreaks


%----------------------------------------------------------------------------------------
%	PORTADA
%----------------------------------------------------------------------------------------

\title{TÍTULO DE LA TESIS} %Con este nombre se guardará el proyecto en writeLaTex

\begin{titlepage}
\begin{center}

\textsc{\Large Instituto Tecnológico Autónomo de México}\\[4em]

%Figura
\begin{figure}[h]
\begin{center}
\includegraphics{logo-ITAM_ch.jpg}
\end{center}
\end{figure}

\vspace{4em}

\textsc{\huge \textbf{El problema de admisión a universidades con cotas inferiores y comunes}}\\[4em]

\textsc{\large Tesis}\\[1em]

\textsc{que para obtener el título de}\\[1em]

\textsc{LICENCIADO EN MATEMÁTICAS APLICADAS}\\[1em]

\textsc{presenta}\\[1em]

\textsc{\Large Ilan Jinich Fainsod}\\[1em]

\textsc{\large Asesor: DR RAMÓN ESPINOSA ARMENTA }

\end{center}

\vspace*{\fill}
\textsc{México, D.F. \hspace*{\fill} 2018}

\end{titlepage}


%----------------------------------------------------------------------------------------
%	DECLARACIÓN
%----------------------------------------------------------------------------------------

\thispagestyle{empty}
\vspace*{\fill}
\begingroup
``Con fundamento en los artículos 21 y 27 de la Ley Federal del Derecho de Autor y como titular de los derechos moral y patrimonial de la obra titulada ``\textbf{El problema de admisión a universidades con cotas inferiores y comunes}'', otorgo de manera gratuita y permanente al Instituto Tecnológico Autónomo de México y a la Biblioteca Raúl Bailléres Jr., la autorización para que fijen la obra en cualquier medio, incluido el electrónico, y la divulguen entre sus usuarios, profesores, estudiantes o terceras personas, sin que pueda percibir por tal divulgación una contraprestación''.

\centering

\hspace{3em}

\textsc{AUTOR}

\vspace{5em}

\rule[1em]{20em}{0.5pt} % Línea para la fecha

\textsc{Fecha}
 
\vspace{8em}

\rule[1em]{20em}{0.5pt} % Línea para la firma

\textsc{Firma}

\endgroup
\vspace*{\fill}


%----------------------------------------------------------------------------------------
%	DEDICATORIA
%----------------------------------------------------------------------------------------

\pagestyle{empty}
\frontmatter

\chapter*{}
\begin{flushright}
%\textit{``To life, to life, L'Chaim!''}
\textit{$6\times 10^6$}
%\textit{זאָג ניט קיין מאָל}
\end{flushright}


%----------------------------------------------------------------------------------------
%	AGRADECIMIENTOS
%----------------------------------------------------------------------------------------

\chapter*{Agradecimientos}
%\markboth{AGRADECIMIENTOS23}{AGRADECIMIENTOS} % encabezado 
%
%Es difícil hacer una lista agradecimientos porque le debo mucho a demasiadas personas, intentare ser lo más exhaustivo posible. \\
%A mis papas, les agradezco por toda la ayuda, apoyo y amor que me han dado desde chiquito. Todos mis logros son gracias a ustedes. \\
%A mis abuelos y a mi tío Nico, siempre cuidándome desde el cielo, inspirándome. Los extrañó mucho y me habría gustado compartir este logro con ustedes. \\
%A mis hermanos, por todo el apoyo. Agradezco el interés continuo que siempre tuvieron en mí, por todas las experiencias que tengo con ustedes. \\
%A mis abuelas, por todo el amor incondicional. Por siempre preocuparse por mí, desde la cosa más superficial. \\
%A mis tíos, por preocuparse siempre de mi futuro. Por apoyarme a conseguir trabajo y por enseñarme como aprovechar todas las oportunidades. \\
%Al resto de mi familia, por siempre estar ahí. \\ 
%A mis profesores de la universidad, por todo lo que aprendí. Especialmente a Ramon Espinosa, por aceptar ser mi asesor de tesis y por siempre estar ahí para lo que necesite; a Claudia Gómez por siempre tener su puerta abierta para mí; a Manuel Mendoza por guiarme, por toda la paciencia que tuvo sobre mis miles de dudas y por enseñarme mis materias favoritas de la carrera; a Víctor Guerrero por darme la oportunidad de ser su asistente y por enseñarme que siempre hay que seguir el camino correcto; a Javier Alfaro por todos los cursos que me dio y por el apoyo constante durante toda mi carrera; a Jorge de la Vega por el esfuerzo constante y por inspirarme a ser lo mejor de mí siempre; a Guillermo Grabinsky por mostrarme lo divertidas e increíbles que pueden ser las matemáticas; a Carlos Bosch por toda la paciencia que me tuvo y por siempre escucharme; a Begoña Albizuri por enseñarme a programar; a Fernando Esponda por las buenas platicas que tuve con él; a Alejandro de los Santos por darme uno de mis cursos favoritos de la carrera; a Pablo Castañeda por siempre molestarme y por dejarme molestarlo; a Ale Flores por enseñarme lo padre que es la programación; a Vladimir Caetano por guiarme al principio de mi carrera; a Luis Enrique Nieto por inspírame en el camino de la estadística; a Ana Ortiz por todo lo que aprendí y a Alfredo Villafranca por enseñarme que hay más cosas en la vida a la escuela o el trabajo. \\
%A mis profesores no de la universidad, por hacerme llegar a donde tuve que llegar. A Enrique Camarena, porque gracias a él estudié matemáticas; a Loren Astorga, por sacarme adelante; a Ángel Peláez y a Yuraldi Díaz por todos los valores que aprendí y por mostrarme el amor al deporte. \\
%A mis amigos, por todos los tiempos divertidos que tuvimos, porque son y fueron la mejor parte de este proceso. A los de facu, los de $\epsilon \delta$, los de Laberintos e infinitos, los de la escuela y todos ellos que no estoy seguro de donde conozco. A Tame, Mariana, Patito, Roberto, Jaco, Natán, Diego, Copca, Manu, Majito, Daniel, Jimena, Gloria, Santi, Reggaetony, Shuk, Giank, Term, Silva, Lemus, José Carlos, Mau, Fredo, Regis, Rafa, David, Alice, Delpa y Rome. \\
%A la familia XHash por apoyarme para que acabe este trabajo y por darme el trabajo más divertido que pude haber pedido. Especialmente a Natán y a Bernardo por los increíbles jefes que son.


%----------------------------------------------------------------------------------------
%	PREFACIO
%----------------------------------------------------------------------------------------

\chapter*{Prefacio}

\pagestyle{plain}
%\markboth{PREFACIO23}{PREFACIO} % encabezado 

PUEDEN QUITAR ESTA PARTE


%----------------------------------------------------------------------------------------
%	TABLA DE CONTENIDOS
%---------------------------------------------------------------------------------------

\tableofcontents
\newpage


%----------------------------------------------------------------------------------------
%	TESIS
%----------------------------------------------------------------------------------------
\mainmatter %empieza la numeración de las páginas
\pagestyle{headings}

%  Incluye los capítulos en el folder de capítulos

\chapter{El modelo}
\begin{flushright}
\textit{``Matchmaker, matchmaker,
Make me a match, Find me a find,
Catch me a catch''}
\end{flushright}
\section{Descripción del problema}
Imaginemos que en una ciudad tenemos $m$ universidades y $n$ personas que desean ir a una de estas escuelas, cada universidad tiene un límite alumnos que puede admitir y por esto no todas las personas pueden asistir a su universidad
 ``favorita'', este tipo de cota puede ser común si dos o más escuelas comparten recursos o personal como es el caso de las universidades publicas que dependen en conjunto del presupuesto del gobierno. 
Además, podemos agregar el supuesto de que las escuelas tienen cotas inferiores, usualmente por razones de dinero o de personal a una escuela no le conviene abrir o aceptar a menos de que entren una cantidad considerable de alumnos a esta. Adicionalmente, tenemos el supuesto de que todos los alumnos tienen una lista de preferencias no exhaustiva (en muchos casos si es exhaustiva) de qué universidad prefieren y de forma análoga las universidades tienen una lista de preferencias sobre los alumnos, cada escuela y persona tienen necesidades diferentes a los demás, algunos podrían preferir la escuela que está más cerca de sus casas, otros la que te prepara mejor académicamente y otros entre muchas razones podrían preferir la que tenga el mejor ambiente social. 
\\ En notación matemática lo que hacemos es suponer que tenemos $n$ alumnos $\alpha_1, \alpha_2, \ldots, \alpha_n$ y $m$ universidades $a_1, a_2, \ldots, a_m$, con las restricciones: 

\begin{enumerate}
\item \begin{equation} \label{r1}
x_{i,j}= 
\begin{cases}
1 & \qquad \text{si $i$ asiste a la universidad $j$.} \\
0 &\qquad\text{en otro caso.}\ \\ 
\end{cases} \end{equation}
\item \begin{equation} y_{j}= 
\begin{cases}
1 & \qquad \text{si la universidad $j$ abre.} \\
0 &\qquad\text{en otro caso.} \\ 
\end{cases} \end{equation}
\item \begin{equation} \label{r2}
\sum_{j=1}^{m}x_{i,j} \leq1 \ \text{ para toda $i=1,2,\ldots,n$. }
\end{equation} Cada estudiante es admitido en a lo más una universidad. 
\item \begin{equation} \label{r3}
\sum_{i=1}^{n} x_{i,j} \leq M_j\ \text{ para toda $j=1,2,\dots,m$.} 
\end{equation}
Cada universidad tiene un límite de alumnos que puede admitir.
\item \begin{equation} \label{r4}
\sum_{i=1}^{n} x_{i,j} \geq\ m_j\times y_j 
\end{equation}
Cada universidad necesita admitir a una cantidad mínima de alumnos para abrir.
%\item \begin{equation} \label{r5}
%\sum_{i=1}^{n} (x_{i,j}+x_{i,j'}) 
%\begin{cases}
%\geq m_{j,j'} \times y_j \\
%\geq m_{j,j'} \times y_{j'} \\
%\end{cases}
%\end{equation} \\
%para toda $(j,j')$, con $j$ distinto de $j'$. \\
%La cota inferior de las universidades puede ser común.

\item \begin{equation}
\sum_{i=1}^{n} \sum_{j=1}^m w_{j,k} \cdot x_{i,j} \leq N_k\cdot %z_k
\end{equation} 
donde \begin{equation} w_{j,k} \footnote{\text{$w_j,k$ es constante}}= 
\begin{cases}
1 & \qquad \text{si la universidad $j$ es parte de la restricción $k$.} \\
0 &\qquad\text{en otro caso.} \\ 
\end{cases} \end{equation} \\   para toda $k=1,2,\dots,p$   \footnote{donde $p$ es el número de restricciones comunes.}. \\
La cota superior de las universidades puede ser común.

%\item  \begin{equation} 
%z_{k}= 
%\begin{cases}
%1 & \qquad \text{si las universidades en la restricción $k$ abren.} \\
%0 &\qquad\text{en otro caso.}\ \\ 
%\end{cases} \end{equation}

%\item \begin{equation}
%\sum_{j=1}^{m} w_{j,k} y_{j} \leq z_{k} \text{  para toda $k=1,2,\dots,p$. }
%\end{equation}    \\ Las universidades solo pueden abrir si cumplen su cota común. 

\item \begin{equation} \label{r6}
x_{i,j} \leq y_j \ \text{ para toda $i=1,2,\ldots,n$ y para toda $j=1,2,\ldots,m$.}
\end{equation}
Los solicitantes únicamente pueden asistir a universidades abiertas.

\end{enumerate}

Es importante mencionar que no hay garantía de que todos los alumnos sean admitidos en alguna universidad, algunos se pueden quedar sin estudiar. No todas las universidades necesariamente tienen cotas superiores, inferiores o comunes; en ciertas situaciones se puede considerar que para cierta universidad $j$, $M_{j}$ puede ser igual a infinito o $m_j$ puede ser igual a cero.

Para definir las preferencias de una forma un poco más formal, hacemos uso de la matriz de preferencias

\begin{dfn}
\label{matpref}
Definimos la \textbf{matriz de preferencias} para un problema con $n$ estudiantes y con $m$ universidades como una matriz con $n$ filas y $m$ columnas donde la primera entrada de la celda $(i,j)$ representa el orden de preferencia que le asigna el solicitante $i$ a la universidad $j$ y análogamente la segunda entrada representa el orden de preferencia que le da la universidad $j$ a la persona $i$. La matriz podría tener lugares vacíos en caso de listas no exhaustivas. 
\end{dfn}

Para que dejar más clara la definición es necesario mostrar un ejemplo. 

\begin{eje}
\label{ejemplo matrimonio 1}
Supongamos que contamos con tres solicitantes $\alpha$, $\beta$ y $\gamma$ y con tres universidades $A$, $B$ y $C$ y la matriz de preferencias
$$\begin{pmatrix}
& A & B & C \\
\alpha & 1,3 & 2,2 & 3,1 \\
\beta & 3,1 & 1,3 & 2,2 \\
\gamma & 2,2 & 3,1 & 1,3 
\end{pmatrix}$$
aquí por ejemplo el orden preferencias de $\alpha$ es $(A,B,C)$ y el orden de preferencias de $C$ es $(\gamma, \beta, \alpha)$.
\fin
\end{eje}


Para continuar es necesario introducir un criterio de estabilidad y otro de optimalidad, para ello enunciamos dos definiciones. 

\begin{dfn}{\cite{GaleShapley}}
\label{Estable}
Decimos que una asignación de solicitantes a universidades es inestable si existen dos solicitantes $\alpha$ y $\beta$ asignados a universidades $A$ y $B$ respectivamente, con la propiedad de que $\alpha$ prefiere estar en $B$ que en $A$ y $B$ prefiere tener a $\alpha$ que a $B$. Es decir, existen una universidad y un solicitante que se prefieren entre ellos a sus respectivas asignaciones. 
Además a esto, decimos que una asignación es \textbf{inestable} si existe una universidad cerrada $A$ y $m_j$ solicitantes $\alpha_1,\alpha,2,\dots,\alpha_{m_j}$, con la propiedad de que los $m_j$ solicitantes prefieren estar en $A$ a su situación actual, en este caso se le llama a $A$ \textbf{universidad bloqueadora}.\\
Alternativamente una asignación de solicitantes a universidades es \textbf{estable} si no es inestable.
\end{dfn}

\begin{dfn}{\cite{GaleShapley}}
\label{optima}
Una asignación es considera \textbf{óptima} si cada solicitante está mejor o igual respecto a sus preferencias que en cualquier otra asignación estable. 
\end{dfn}

A continuación mostramos un par de ejemplos de cómo funcionan las definiciones para dejar todo más claro. 

\begin{eje}
Retomando el ejemplo \ref{ejemplo matrimonio 1}. \\
El conjunto de parejas $(\alpha, A), \; (\beta, B)\; y\; (\gamma, C)$ es estable porque a pesar de las preferencias de las universidades cada persona esta con su primera opción, es decir, los solicitantes prefieren a su universidad más que a cualquier otra. Esta además es óptima porque cada solicitante está en su primera opción.
\begin{figure}[H]\centering

\begin{tikzpicture}[ scale=0.8]
\tikzset{vertex/.style = {shape=circle,draw,minimum size=1.5em}}
\tikzset{edge/.style = {->,> = latex}}
\filldraw[color=blue!60, fill=blue!5, very thick](0,3) ellipse (.8 and 1.5);
\filldraw[color=green!60, fill=green!5, very thick](4,3) ellipse (.8 and 1.5);


% vertices
% 


\node[vertex] (a) at (0,4) {$\alpha$};
\node[vertex] (b) at (0,3) {$\beta$};
\node[vertex] (c) at (0,2) {$\gamma$};


\node[vertex] (e) at (4,4) {$A$};
\node[vertex] (f) at (4,3) {$B$};
\node[vertex] (g) at (4,2) {$C$};


\node (i) at (0,5) {Solicitantes};
\node (j) at (4,5) {Universidades};

\path[-stealth] (a) edge (e);
\path[-stealth] (b) edge (f);
\path[-stealth] (c) edge (g);



%\draw (0.2,8)--(3.8,8);



\end{tikzpicture}

\caption{Emparejamiento estable.}
\end{figure}

\fin
\end{eje}
\begin{eje}
Retomando el ejemplo \ref{ejemplo matrimonio 1}. \\
El conjunto de parejas $(\alpha, A)$, $(\gamma, B)$ y $(\beta, C)$ es inestable porque $\gamma$ prefiere a $A$ que a su universidad actual ($B$) y A prefiere a $\gamma$ que a su universidad actual $(\alpha)$.

\begin{figure}[H]\centering

\begin{tikzpicture}[ scale=0.8]
\tikzset{vertex/.style = {shape=circle,draw,minimum size=1.5em}}
\tikzset{edge/.style = {->,> = latex}}
\filldraw[color=blue!60, fill=blue!5, very thick](0,3) ellipse (.8 and 1.5);
\filldraw[color=green!60, fill=green!5, very thick](4,3) ellipse (.8 and 1.5);


% vertices
% 


\node[vertex] (a) at (0,4) {$\alpha$};
\node[vertex] (b) at (0,3) {$\beta$};
\node[vertex] (c) at (0,2) {$\gamma$};


\node[vertex] (e) at (4,4) {$A$};
\node[vertex] (f) at (4,3) {$B$};
\node[vertex] (g) at (4,2) {$C$};


\node (i) at (0,5) {Solicitantes};
\node (j) at (4,5) {Universidades};

\path[-stealth] (a) edge (e);
\path[-stealth] (c) edge (f);
\path[-stealth] (b) edge (g);



%\draw (0.2,8)--(3.8,8);



\end{tikzpicture}

\caption{Emparejamiento inestable.}
\end{figure}
\fin
\end{eje}


Surgen algunas preguntas de aquí ¿siempre existe un emparejamiento estable? ¿cómo se encuentra? Para resolverlas explicaremos un poco sobre algoritmos y su complejidad. Luego, consideraremos el caso en el que $M_j=1$ para toda universidad $j$ y donde además no hay cotas inferiores o comunes, que es mejor conocido como el problema del matrimonio estable. 

\thispagestyle{empty}

\chapter{Un poco sobre computación}

La idea de este capítulo es introducir la idea de complejidad algorítmica. Se habla de qué es un algoritmo, de cuál es el orden de complejidad de un algoritmo y de algunas clases de complejidad de algoritmos. Para empezar a introducir estas ideas son necesarias varias definiciones. 

\begin{dfn} \cite{Ramon}
Definimos un \textbf{algoritmo} como una serie de pasos para resolver un problema. El \textbf{tamaño} de un problema es la cantidad de datos necesarios para resolver el problema (las entradas del algoritmo). La \textbf{complejidad} de un problema es una función $f$, donde $f(n)$ es la cantidad de operaciones necesarias para que el algoritmo termine. Además, decimos que $f(n)$ es orden a lo más $g(n)$ si para alguna $K>0$ se cumple que
$$f(n) \leq Kg(n)$$
para todo $n$ natural.
\end{dfn}

A partir de esto podemos definir la clase de algoritmos polinomiales, los cuales son muy importantes para entender el contenido de esta tesis. 

\begin{dfn}
Decimos que un algoritmo es \textbf{polinomial} si el algoritmo es de orden a lo más $n^k$ para alguna $k$ en los enteros. 
\end{dfn}


\section{Clases de complejidad}
Algo que nos interesa mucho cuando atacamos un problema de computación es que tan difícil es de resolver y si existe una mejor manera de hacerlo. La complejidad computacional nos permite estudiar esto. Para encarrilarnos hay que empezar con algunas definiciones y ejemplos.
\begin{dfn}
Se dice que un problema es un problema de decisión si admite dos posibles respuestas ``Si'' o ``No''. 
\end{dfn}
Existen dos clases de problemas de decisión que nos interesan, los que son fáciles de verificar (NP) y los que son fáciles de resolver (P).
\begin{dfn}
Un problema está en la clase NP si existe un algoritmo polinomial para verificar que la respuesta de una solución dada es ``Si''.
\end{dfn}

\begin{eje}[Problema del clima]
Supongamos que Mariana le dice a Adrián que está lloviendo, si Adrián decide verificar si esto es cierto solo tiene que realizar tres pasos:
\begin{enumerate}
\item Caminar a la ventana.
\item Abrir la ventana.
\item Ver el cielo.
\end{enumerate}
Por lo tanto, a Adrián le toma un tiempo constante verificar si está lloviendo lo que implica que este problema está en NP.
 \fin
\end{eje}
\begin{dfn}
Se dice que un problema de decisión está en P si existe un algoritmo polinomial que lo resuelve.
\end{dfn}
\begin{obs}[$P \subseteq NP$]
Si un problema de decisión pertenece a P entonces pertenece a NP.
\end{obs}
%\begin{proof}
%Supongamos que $A$ es un problema de decisión en P y $b$ es una solución dada de $A$, podemos verificar si la respuesta de la solución es ``Si'' resolviendo el problema en tiempo polinomial.
%\end{proof}
\begin{obs}
\label{pp eq}
Si A es un problema de decisión en P y existe un algoritmo polinomial que reduce resolver el problema de decisión B a resolver el problema A entonces B está en P.
\end{obs}

Vale la pena mencionar algunos ejemplos de problemas que están en P:
\begin{enumerate}
\item Verificar si un número es primo.
\item Encontrar la ruta más corta entre dos vértices en una gráfica.
\item El problema de la ruta crítica. 
\end{enumerate}

Una conjetura famosa que vale la pena mencionar es ¿si $P=NP$?, es decir si todo problema en NP también se encuentra en P. La persona que lo demuestra además de ganar fama y trabajo va a ser acreedor a un premio de un millón de dólares. 


\section{Problemas NP-Completos}

\begin{dfn}
Decimos que $A$ un problema es \textbf{NP-Completo} si para todo problema $B$ en NP existe una reducción polinomial que reduce el resolver $B$ a resolver $A$.
\end{dfn}

Una conclusión inmediata de este problema es la siguiente: 

\begin{obs}
Si existe algún algoritmo polinomial para resolver un problema NP-Completo entonces $P=NP$.
\end{obs}

Algunos ejemplos de problemas NP-Completos son: 

\begin{eje}
\begin{itemize}
\item El problema del agente viajero. 
\item El problema de 3-coloración. 
\item El problema de las galerías de arte.
\end{itemize}
\fin
\end{eje}


En el siguiente capítulo se reduce el problema planteado en el capítulo 1 a su caso más simple, conocido como el problema del matrimonio estable.

\thispagestyle{empty}

\chapter{El problema del matrimonio estable} \label{me}
%\begin{flushright}
%\textit{``Even the worst husband, God forbid, is better than no husband, God forbid.''}
%\end{flushright}
En el pequeño Shtetl de Anatevka, había muchos hombres y muchas mujeres jóvenes con el deseo de casarse. Yenta, la casamentera del pueblo, tenía el trabajo y la obligación de emparejar a todos estos jóvenes. Esta tarea sonará fácil pero no lo es, cada joven tenía distintas preferencias y cada uno de ellos necesitaba un trato especial, ``el matrimonio es para toda la vida y no es algo que hay que tomar como si fuera cualquier cosa" decía Yenta. Retomando los conceptos de la última sección, se desea que el emparejamiento de estos jóvenes sea estable y óptimo si es que es posible.\footnote{Este problema hace el supuesto de que las relaciones solo pueden ser entre un hombre y una mujer y que esto es reflejado por sus preferencias. Esto no manifiesta la realidad o las opiniones del autor y se hace únicamente con fines matemáticos.}
En términos matemáticos, supongamos que tenemos $n$ hombres $\alpha_1,\alpha_2,\ldots,\alpha_n$ y $m$ mujeres $a_1, a_2,\ldots,a_m$ con las restricciones:
\begin{enumerate}
\item \begin{equation} \label{1r1}
x_{i,j}= 
\begin{cases}
1 & \qquad \text{si $i$ se casa con $j$.} \\
0 &\qquad\text{en otro caso.}\ \\ 
\end{cases} \end{equation}
\item \begin{equation} \label{1r2}
\sum_{j=1}^{m}x_{i,j} \leq1 \ \text{ para toda $i=1,2,\ldots,n$. }
\end{equation} Cada hombre se casa con a lo más una mujer. 
\item \begin{equation} \label{1r3}
\sum_{i=1}^{n} x_{i,j} \leq 1\ \text{ para toda $j=1,2,\dots,m$.} 
\end{equation}
Cada mujer se casa con a lo más un hombre. 
\end{enumerate}

Es claro, a partir de la definición del problema que éste es un caso particular del problema original. Aprovechando lo que se hizo en la sección anterior la definición de la matriz de preferencias, un emparejamiento estable y de un emparejamiento óptimo son análogas a las definiciones \ref{matpref}, \ref{Estable} y \ref{optima}. Existen diversos resultados sobre el problema del matrimonio estable, uno de los más famosos es por Gale y Shapley que encontraron un algoritmo que siempre encuentra un emparejamiento estable y además lo hace sin muchas complicaciones. 

Un supuesto adicional que haremos es que el número de hombres es exactamente igual al número de mujeres. Esto se verá después que es relativamente sencillo de generalizar al caso en el que el número de hombres y de mujeres no es igual. 

%Definición del problema
%Definiciones alternativas
%

%Ejemplos

\section{Algoritmo de Gale Shapley}
En 1962 David Gale y Lloyd Stowell Shapley demostraron que para el problema del matrimonio estable siempre se puede encontrar un emparejamiento estable óptimo para los hombres, esto se hizo enunciando un algoritmo y mostrando su convergencia. El algoritmo queda representado por el siguiente código y es conocido como el algoritmo de Gale Shapley.

%\begin{lstlisting}[style=R, escapeinside={(*}{*)},caption={Algoritmo de Gale Shapley}, captionpos=b, label=c1]
%Input : Una matriz de preferencias para (*$n$*) hombres y (*$n$*) mujeres 
%Output: Un emparejamiento. 
%Cada hombre le propone a la primera mujer de su lista. Cada mujer que recibe más de una propuesta acepta la que este más arriba en su lista y rechaza al resto. 
%repeat{ #hasta que todos los hombres tengan pareja
%Los hombres no emparejados le proponen a la siguiente mujer en su lista. 
%Cada mujer que recibe una propuesta escoge la que está más arriba en su lista entre las propuestas que recibió y su pareja actual.
%Las mujeres rechazan las propuestas que no aceptaron.
%}
%\end{lstlisting}

\IncMargin{1em}
\begin{Algoritmo}[H]
%\SetKwData{Left}{left}\SetKwData{This}{this}\SetKwData{Up}{up}
%\SetKwFunction{Union}{Union}\SetKwFunction{FindCompress}{FindCompress}
\SetKwInOut{Input}{input}\SetKwInOut{Output}{output}
\Input{Una matriz de preferencias para $n$ hombres y $n$ mujeres}
\Output{Un emparejamiento. }
\BlankLine
\emph{Cada hombre le propone a la primera mujer de su lista\; Cada mujer que recibe más de una propuesta acepta la que este más arriba en su lista y rechaza al resto \; }
\Repeat{hasta que todos los hombres tengan pareja}{
	\emph{Los hombres no emparejados le proponen a la siguiente mujer en su lista\; } 
	\emph{Cada mujer que recibe una propuesta escoge la que está más arriba en su lista entre las propuestas que recibió y su pareja actual\;} 
	\emph{Las mujeres rechazan las propuestas que no aceptaron\;} 
}
\caption{Gale Shapley}
\end{Algoritmo}
\DecMargin{1em}
Para dejar la explicación un poco más clara es importante mostrar un ejemplo de cómo funciona el algoritmo y cuál es su resultado final.
\begin{eje}{\cite{GaleShapley}}
\label{ejemploGS}
Supongamos que para 4 hombres y 4 mujeres tenemos la matriz de preferencias 
$$\begin{pmatrix}
& A & B & C & D \\
\alpha & 1,3 & 2,2 & 3,1 & 4,3 \\
\beta & 1,4 & 2,3 & 3,2 & 4,4 \\
\gamma & 3,1 & 1,4 & 2,3 & 4,2 \\
\delta & 2,2&3,1 &1,4 & 4,1 
\end{pmatrix}.$$
En la primera iteración, $\alpha$ le propone matrimonio a $A$, $\beta$ le propone matrimonio a $A$, $\gamma$ le propone matrimonio a $B$ y $\delta$ le propone matrimonio a $C$.

\begin{figure}[H]\centering

\begin{tikzpicture}[ scale=0.8]
\tikzset{vertex/.style = {shape=circle,draw,minimum size=1.5em}}
\tikzset{edge/.style = {->,> = latex}}
\filldraw[color=blue!60, fill=blue!5, very thick](0,2.5) ellipse (.8 and 2);
\filldraw[color=red!60, fill=red!5, very thick](4,2.5) ellipse (.8 and 2);


% vertices
% 


\node[vertex] (a) at (0,4) {$\alpha$};
\node[vertex] (b) at (0,3) {$\beta$};
\node[vertex] (c) at (0,2) {$\gamma$};
\node[vertex] (d) at (0,1) {$\delta$};

\node[vertex] (e) at (4,4) {$A$};
\node[vertex] (f) at (4,3) {$B$};
\node[vertex] (g) at (4,2) {$C$};
\node [vertex] (h) at (4,1) {$D$};

\node (i) at (0,5) {Hombres};
\node (j) at (4,5) {Mujeres};

\path[-stealth] (a) edge (e);
\path[-stealth] (b) edge (e);
\path[-stealth] (c) edge (f);
\path[-stealth] (d) edge (g);



%\draw (0.2,8)--(3.8,8);



\end{tikzpicture}

\caption{Primera iteración.}
\end{figure}

En la segunda iteración, $A$ acepta la propuesta de $\alpha$ y $\beta$ le propone matrimonio a $B$.

\begin{figure}[H]
\centering

\begin{tikzpicture}[ scale=0.8]
\tikzset{vertex/.style = {shape=circle,draw,minimum size=1.5em}}
\tikzset{edge/.style = {->,> = latex}}
\filldraw[color=blue!60, fill=blue!5, very thick](0,2.5) ellipse (.8 and 2);
\filldraw[color=red!60, fill=red!5, very thick](4,2.5) ellipse (.8 and 2);


% vertices
% 


\node[vertex] (a) at (0,4) {$\alpha$};
\node[vertex] (b) at (0,3) {$\beta$};
\node[vertex] (c) at (0,2) {$\gamma$};
\node[vertex] (d) at (0,1) {$\delta$};

\node[vertex] (e) at (4,4) {$A$};
\node[vertex] (f) at (4,3) {$B$};
\node[vertex] (g) at (4,2) {$C$};
\node [vertex] (h) at (4,1) {$D$};

%\draw (0,.5) node[cross,red] {};

\node (i) at (0,5) {Hombres};
\node (j) at (4,5) {Mujeres};

\path[-stealth] (a) edge (e);
\path[-stealth] (b) edge (f);
\path[-stealth] (c) edge (f);
\path[-stealth] (d) edge (g);



%\draw (0.2,8)--(3.8,8);



\end{tikzpicture}

\caption{Segunda iteración.}
\end{figure}


En la tercera iteración, $B$ acepta la propuesta de $\beta$ y $\gamma$ le propone matrimonio a $C$.

\begin{figure}[H]\centering

\begin{tikzpicture}[ scale=0.8]
\tikzset{vertex/.style = {shape=circle,draw,minimum size=1.5em}}
\tikzset{edge/.style = {->,> = latex}}
\filldraw[color=blue!60, fill=blue!5, very thick](0,2.5) ellipse (.8 and 2);
\filldraw[color=red!60, fill=red!5, very thick](4,2.5) ellipse (.8 and 2);


% vertices
% 


\node[vertex] (a) at (0,4) {$\alpha$};
\node[vertex] (b) at (0,3) {$\beta$};
\node[vertex] (c) at (0,2) {$\gamma$};
\node[vertex] (d) at (0,1) {$\delta$};

\node[vertex] (e) at (4,4) {$A$};
\node[vertex] (f) at (4,3) {$B$};
\node[vertex] (g) at (4,2) {$C$};
\node [vertex] (h) at (4,1) {$D$};

\node (i) at (0,5) {Hombres};
\node (j) at (4,5) {Mujeres};

\path[-stealth] (a) edge (e);
\path[-stealth] (b) edge (f);
\path[-stealth] (c) edge (g);
\path[-stealth] (d) edge (g);



%\draw (0.2,8)--(3.8,8);



\end{tikzpicture}

\caption{Tercera iteración.}
\end{figure}

En la cuarta iteración, $C$ acepta la propuesta de $\gamma$ y $\beta$ le propone matrimonio a $A$.

\begin{figure}[H]\centering

\begin{tikzpicture}[ scale=0.8]
\tikzset{vertex/.style = {shape=circle,draw,minimum size=1.5em}}
\tikzset{edge/.style = {->,> = latex}}
\filldraw[color=blue!60, fill=blue!5, very thick](0,2.5) ellipse (.8 and 2);
\filldraw[color=red!60, fill=red!5, very thick](4,2.5) ellipse (.8 and 2);


% vertices
% 


\node[vertex] (a) at (0,4) {$\alpha$};
\node[vertex] (b) at (0,3) {$\beta$};
\node[vertex] (c) at (0,2) {$\gamma$};
\node[vertex] (d) at (0,1) {$\delta$};

\node[vertex] (e) at (4,4) {$A$};
\node[vertex] (f) at (4,3) {$B$};
\node[vertex] (g) at (4,2) {$C$};
\node [vertex] (h) at (4,1) {$D$};

\node (i) at (0,5) {Hombres};
\node (j) at (4,5) {Mujeres};

\path[-stealth] (a) edge (e);
\path[-stealth] (b) edge (f);
\path[-stealth] (c) edge (g);
\path[-stealth] (d) edge (e);



%\draw (0.2,8)--(3.8,8);



\end{tikzpicture}

\caption{Cuarta iteración.}
\end{figure}


En la quinta iteración, $A$ acepta la propuesta de $\delta$ y $\alpha$ le propone matrimonio a $B$.

\begin{figure}[H]\centering

\begin{tikzpicture}[ scale=0.8]
\tikzset{vertex/.style = {shape=circle,draw,minimum size=1.5em}}
\tikzset{edge/.style = {->,> = latex}}
\filldraw[color=blue!60, fill=blue!5, very thick](0,2.5) ellipse (.8 and 2);
\filldraw[color=red!60, fill=red!5, very thick](4,2.5) ellipse (.8 and 2);


% vertices
% 


\node[vertex] (a) at (0,4) {$\alpha$};
\node[vertex] (b) at (0,3) {$\beta$};
\node[vertex] (c) at (0,2) {$\gamma$};
\node[vertex] (d) at (0,1) {$\delta$};

\node[vertex] (e) at (4,4) {$A$};
\node[vertex] (f) at (4,3) {$B$};
\node[vertex] (g) at (4,2) {$C$};
\node [vertex] (h) at (4,1) {$D$};

\node (i) at (0,5) {Hombres};
\node (j) at (4,5) {Mujeres};

\path[-stealth] (a) edge (f);
\path[-stealth] (b) edge (f);
\path[-stealth] (c) edge (g);
\path[-stealth] (d) edge (e);



%\draw (0.2,8)--(3.8,8);



\end{tikzpicture}

\caption{Quinta iteración.}
\end{figure}

En la sexta iteración, $B$ acepta la propuesta de $\alpha$ y $\beta$ le propone matrimonio a $C$.

\begin{figure}[H]\centering

\begin{tikzpicture}[ scale=0.8]
\tikzset{vertex/.style = {shape=circle,draw,minimum size=1.5em}}
\tikzset{edge/.style = {->,> = latex}}
\filldraw[color=blue!60, fill=blue!5, very thick](0,2.5) ellipse (.8 and 2);
\filldraw[color=red!60, fill=red!5, very thick](4,2.5) ellipse (.8 and 2);


% vertices
% 


\node[vertex] (a) at (0,4) {$\alpha$};
\node[vertex] (b) at (0,3) {$\beta$};
\node[vertex] (c) at (0,2) {$\gamma$};
\node[vertex] (d) at (0,1) {$\delta$};

\node[vertex] (e) at (4,4) {$A$};
\node[vertex] (f) at (4,3) {$B$};
\node[vertex] (g) at (4,2) {$C$};
\node [vertex] (h) at (4,1) {$D$};

\node (i) at (0,5) {Hombres};
\node (j) at (4,5) {Mujeres};

\path[-stealth] (a) edge (f);
\path[-stealth] (b) edge (g);
\path[-stealth] (c) edge (g);
\path[-stealth] (d) edge (e);



%\draw (0.2,8)--(3.8,8);



\end{tikzpicture}

\caption{Sexta iteración.}
\end{figure}

En la séptima iteración, $C$ acepta la propuesta de $\beta$ y $\gamma$ le propone matrimonio a $A$.

\begin{figure}[H]\centering

\begin{tikzpicture}[ scale=0.8]
\tikzset{vertex/.style = {shape=circle,draw,minimum size=1.5em}}
\tikzset{edge/.style = {->,> = latex}}
\filldraw[color=blue!60, fill=blue!5, very thick](0,2.5) ellipse (.8 and 2);
\filldraw[color=red!60, fill=red!5, very thick](4,2.5) ellipse (.8 and 2);


% vertices
% 


\node[vertex] (a) at (0,4) {$\alpha$};
\node[vertex] (b) at (0,3) {$\beta$};
\node[vertex] (c) at (0,2) {$\gamma$};
\node[vertex] (d) at (0,1) {$\delta$};

\node[vertex] (e) at (4,4) {$A$};
\node[vertex] (f) at (4,3) {$B$};
\node[vertex] (g) at (4,2) {$C$};
\node [vertex] (h) at (4,1) {$D$};

\node (i) at (0,5) {Hombres};
\node (j) at (4,5) {Mujeres};

\path[-stealth] (a) edge (f);
\path[-stealth] (b) edge (g);
\path[-stealth] (c) edge (e);
\path[-stealth] (d) edge (e);



%\draw (0.2,8)--(3.8,8);



\end{tikzpicture}

\caption{Séptima iteración.}
\end{figure}


En la octava iteración, $A$ acepta la propuesta de $\gamma$ y $\delta$ le propone matrimonio a $B$.

\begin{figure}[H]\centering

\begin{tikzpicture}[ scale=0.8]
\tikzset{vertex/.style = {shape=circle,draw,minimum size=1.5em}}
\tikzset{edge/.style = {->,> = latex}}
\filldraw[color=blue!60, fill=blue!5, very thick](0,2.5) ellipse (.8 and 2);
\filldraw[color=red!60, fill=red!5, very thick](4,2.5) ellipse (.8 and 2);


% vertices
% 


\node[vertex] (a) at (0,4) {$\alpha$};
\node[vertex] (b) at (0,3) {$\beta$};
\node[vertex] (c) at (0,2) {$\gamma$};
\node[vertex] (d) at (0,1) {$\delta$};

\node[vertex] (e) at (4,4) {$A$};
\node[vertex] (f) at (4,3) {$B$};
\node[vertex] (g) at (4,2) {$C$};
\node [vertex] (h) at (4,1) {$D$};

\node (i) at (0,5) {Hombres};
\node (j) at (4,5) {Mujeres};

\path[-stealth] (a) edge (f);
\path[-stealth] (b) edge (g);
\path[-stealth] (c) edge (e);
\path[-stealth] (d) edge (f);



%\draw (0.2,8)--(3.8,8);



\end{tikzpicture}

\caption{Octava iteración.}
\end{figure}

En la novena iteración, $B$ acepta la propuesta de $\delta$ y $\alpha$ le propone matrimonio a $C$.

\begin{figure}[H]\centering

\begin{tikzpicture}[ scale=0.8]
\tikzset{vertex/.style = {shape=circle,draw,minimum size=1.5em}}
\tikzset{edge/.style = {->,> = latex}}
\filldraw[color=blue!60, fill=blue!5, very thick](0,2.5) ellipse (.8 and 2);
\filldraw[color=red!60, fill=red!5, very thick](4,2.5) ellipse (.8 and 2);


% vertices
% 


\node[vertex] (a) at (0,4) {$\alpha$};
\node[vertex] (b) at (0,3) {$\beta$};
\node[vertex] (c) at (0,2) {$\gamma$};
\node[vertex] (d) at (0,1) {$\delta$};

\node[vertex] (e) at (4,4) {$A$};
\node[vertex] (f) at (4,3) {$B$};
\node[vertex] (g) at (4,2) {$C$};
\node [vertex] (h) at (4,1) {$D$};

\node (i) at (0,5) {Hombres};
\node (j) at (4,5) {Mujeres};

\path[-stealth] (a) edge (g);
\path[-stealth] (b) edge (g);
\path[-stealth] (c) edge (e);
\path[-stealth] (d) edge (f);



%\draw (0.2,8)--(3.8,8);



\end{tikzpicture}

\caption{Novena iteración.}
\end{figure}


En la décima iteración, $C$ acepta la propuesta de $\alpha$, $\beta$ le propone matrimonio a $D$ y $D$ acepta la propuesta de $\beta$. Después de 10 pasos en el algoritmo, todos los hombres ya están emparejados, es decir, el algoritmo convergió en un emparejamiento. Cabe mencionar que el número total de propuestas fue 13. 

\begin{figure}[H]\centering

\begin{tikzpicture}[ scale=0.8]
\tikzset{vertex/.style = {shape=circle,draw,minimum size=1.5em}}
\tikzset{edge/.style = {->,> = latex}}
\filldraw[color=blue!60, fill=blue!5, very thick](0,2.5) ellipse (.8 and 2);
\filldraw[color=red!60, fill=red!5, very thick](4,2.5) ellipse (.8 and 2);


% vertices
% 


\node[vertex] (a) at (0,4) {$\alpha$};
\node[vertex] (b) at (0,3) {$\beta$};
\node[vertex] (c) at (0,2) {$\gamma$};
\node[vertex] (d) at (0,1) {$\delta$};

\node[vertex] (e) at (4,4) {$A$};
\node[vertex] (f) at (4,3) {$B$};
\node[vertex] (g) at (4,2) {$C$};
\node [vertex] (h) at (4,1) {$D$};

\node (i) at (0,5) {Hombres};
\node (j) at (4,5) {Mujeres};

\path[-stealth] (a) edge (g);
\path[-stealth] (b) edge (h);
\path[-stealth] (c) edge (e);
\path[-stealth] (d) edge (f);



%\draw (0.2,8)--(3.8,8);



\end{tikzpicture}

\caption{Décima iteración.}
\end{figure}
\fin
\end{eje}

Lo importante de este algoritmo no es solo que produce un emparejamiento, adicional a esto como lo muestra el siguiente teorema, el emparejamiento resultante es estable. 
\begin{teo}[Teorema de Gale Shaley] \cite{GaleShapley} \\
\label{teorema de Gale Shapley}
El algoritmo de Gale Shapley termina en un emparejamiento estable.
\end{teo}
\begin{proof}
Supongamos que el emparejamiento producido por el algoritmo no es estable. Esto es, un hombre $\alpha$ y una mujer $A$ se prefieren entre ellos que a sus respectivas parejas. 

Como $\alpha$ prefiere a $A$ más que a su esposa entonces, $\alpha$ le propuso matrimonio primero a $A$ que a su propia esposa. 
Además, como $A$ prefiere a $\alpha$ que a su esposo entonces, $A$ hubiera rechazado a su esposo y se hubiera quedado casada con $\alpha$ lo cual es una contradicción. 

Por lo tanto, el algoritmo de Gale Shapley termina siempre en un emparejamiento estable. 
\end{proof}
Un resultado inmediato de esto es que siempre, sin importar como sean las preferencias, existe un emparejamiento estable. 
\begin{cor}
\label{corexiste}
Dada una matriz de preferencias arbitraria existe un emparejamiento estable. 
\end{cor}
\begin{proof}
Si aplicamos el algoritmo de Gale Shapley, sabemos por el teorema \ref{teorema de Gale Shapley} que este siempre acaba en un emparejamiento estable. Por lo tanto, siempre existe un emparejamiento estable.
\end{proof}

Una vez que sabemos que el algoritmo siempre converge, nos interesa conocer que tan rápido lo hace y si tiene algunas ventajas el emparejamiento resultante. El lema \ref{lema 1} nos ayuda a llegar a estos resultados y los corolarios \ref{cor1} y \ref{cor2} nos dan una idea de que tan bueno es el algoritmo. 
\begin{lem} 
\label{lema 1} \cite{Knuth} \\
Bajo el emparejamiento obtenido por Gale Shapley, solo un hombre puede terminar con la última mujer de su lista como pareja. 
\end{lem}

\begin{proof}

Supongamos que en el emparejamiento de Gale Shapley $m$ ($m\geq2$) hombres terminan con la última mujer de su lista como pareja, eso significa que cada uno de esos $m$ hombres invitó a salir a todas las mujeres. Entonces cada mujer fue invita a salir por lo menos $m$ veces, lo cual es una contradicción porque el algoritmo acaba cuando invitan a salir a la última mujer y a ésta solo la invitan a salir una vez. Por lo tanto, solo un hombre puede terminar con la última mujer de su lista como pareja. 
\end{proof}

Dos consecuencias casi inmediatas de esto son que bajo el algoritmo la gran mayoría de los hombres no terminan con su última opción y que el algoritmo converge relativamente rápido. 
\begin{cor}
\label{cor1}
Si en un emparejamiento estable por lo menos dos hombres están emparejados con la última mujer de sus respectivas listas, entonces existe dos o más emparejamientos estables en el problema.
\end{cor}

\begin{proof}
Llamemos $M$ al emparejamiento estable donde por lo menos dos hombres están emparejados con la última mujer de sus respectivas listas y llamemos $M'$ al emparejamiento de Gale Shapley.
Por el teorema \ref{teorema de Gale Shapley} sabemos que el algoritmo de Gale Shapley siempre acaba en un emparejamiento estable y además en ese emparejamiento estable solo un hombre puede terminar con la última mujer de su lista como pareja por el lema \ref{lema 1}.
Por lo tanto $m$ y $m'$ son diferentes y como ambos son emparejamientos estables entonces el número de emparejamientos estables en el problema es mayor o igual a dos. 
\end{proof}

\begin{cor}
\label{cor2}
El número máximo de propuestas en el algoritmo es $n^2-n+1$
\end{cor}

\begin{proof}
Por el lema \ref{lema 1} sabemos que a lo más un hombre acaba emparejado con la última mujer de su lista, por lo tanto, el peor emparejamiento posible para el algoritmo es uno donde $n-1$ hombres terminan con la penúltima mujer de sus respectivas listas y un hombre termina con la última. Para llegar a esta situación de acuerdo con el algoritmo, los $n-1$ hombres deben de realizar $n-1$ propuestas cada uno y el otro hombre debe de realizar $n$ propuestas. Esto es $(n-1)(n-1)+n$ propuestas que es igual a $n^2-n+1$ propuestas.
\end{proof}

\begin{obs}
\label{complejidad1}
La complejidad del algoritmo de Gale Shapley es del orden de $n^2$.
\end{obs}

\begin{eje}
Si retomamos el ejemplo \ref{ejemploGS} podemos ver que para 4 personas hubo $4^{2}-4+1=13$ propuestas que es el máximo número posible de éstas de acuerdo con el corolario \ref{cor2}.
\fin
\end{eje}

A partir de esto ya podemos resolver el problema en el que la cantidad de hombres y de mujeres es distinta. El siguiente resultado muestra que sin importar el número de hombres o de mujeres, el algoritmo también converge. 

\begin{teo} \cite{GaleShapley} \\
Dada una matriz de preferencias arbitraria con $n$ hombres y $m$ mujeres, el algoritmo de Gale Shapley converge a un emparejamiento estable
\end{teo}
\begin{proof}
Supongamos que la cantidad de hombres es más chica que la cantidad de mujeres ($n<m$), en este caso el algoritmo acaba cuando $n$ de las $m$ mujeres reciben una propuesta. Si suponemos que la cantidad de hombres es mayor a la cantidad de mujeres ($n>m$), en este caso el algoritmo acaba después de que $n-m$ hombres son rechazados por todas las mujeres y las propuestas de $m$ de los hombres son aceptadas. 

Además de forma análoga al teorema \ref{teorema de Gale Shapley} se puede ver que el emparejamiento producido es estable.
\end{proof}
A partir de esto podemos generalizar el algoritmo de Gale Shapley para que permita un número distinto de hombres y de mujeres. 

\IncMargin{1em}
\begin{Algoritmo}[H]
%\SetKwData{Left}{left}\SetKwData{This}{this}\SetKwData{Up}{up}
%\SetKwFunction{Union}{Union}\SetKwFunction{FindCompress}{FindCompress}
\SetKwInOut{Input}{input}\SetKwInOut{Output}{output}
\Input{Una matriz de preferencias para $n$ hombres y $m$ mujeres}
\Output{Un emparejamiento. }
\BlankLine
\emph{Cada hombre le propone a la primera mujer de su lista\; Cada mujer que recibe más de una propuesta acepta la que este más arriba en su lista y rechaza al resto \; }
\Repeat{hasta que todos los hombres tengan pareja o hasta que sean rechazados por todas las mujeres}{
	\emph{Los hombres no emparejados le proponen a la siguiente mujer en su lista\; } 
	\emph{Cada mujer que recibe una propuesta escoge la que está más arriba en su lista entre las propuestas que recibió y su pareja actual\;} 
	\emph{Las mujeres rechazan las propuestas que no aceptaron\;} 
}
\caption{Gale Shapley para un número distinto de hombres y mujeres}
\end{Algoritmo}
\DecMargin{1em}


El algoritmo de Gale Shapley no es el único algoritmo que existe para producir un emparejamiento estable, y si se cambia el algoritmo el emparejamiento producido podría ser totalmente diferente al producido por el primero. De forma inmediata se puede crear un algoritmo igual al de Gale Shapley con la única diferencia de que las mujeres les proponen a los hombres, esto está representado por el siguiente código.


%\begin{lstlisting}[style=R, escapeinside={(*}{*)},caption={Algoritmo de Gale Shapley para las mujeres}, captionpos=b, label=c1]
%Input : Una matriz de preferencias para (*$n$*) hombres y (*$n$*) mujeres 
%Output: Un emparejamiento. 
%Cada mujer le propone al primer hombre de su lista. Cada hombre que recibe más de una propuesta acepta la que este más arriba en su lista y rechaza al resto. 
%repeat{ #hasta que todas las mujeres tengan pareja
%Los hombres no emparejados le proponen al siguiente hombre en su lista. 
%Cada hombre que recibe una propuesta escoge la que está más arriba en su lista entre las propuestas que recibió y su pareja actual.
%Los hombres rechazan las propuestas que no aceptaron.
%}
%\end{lstlisting}

\IncMargin{1em}
\begin{Algoritmo}[H]
%\SetKwData{Left}{left}\SetKwData{This}{this}\SetKwData{Up}{up}
%\SetKwFunction{Union}{Union}\SetKwFunction{FindCompress}{FindCompress}
\SetKwInOut{Input}{input}\SetKwInOut{Output}{output}

\Input{Una matriz de preferencias para $n$ hombres y $m$ mujeres }
\Output{Un emparejamiento.}
\BlankLine
\emph{Cada mujer le propone al primer hombre de su lista\; Cada hombre que recibe más de una propuesta acepta la que este más arriba en su lista y rechaza al resto\; }
\Repeat{hasta que todas las mujeres tengan pareja o o hasta que sean rechazadas por todas los hombres}{
	\emph{Las mujeres no emparejadas le proponen al siguiente hombre en su lista\; } 
	\emph{Cada hombre que recibe una propuesta escoge la que está más arriba en su lista entre las propuestas que recibió y su pareja actual\;} 
	\emph{Los hombres rechazan las propuestas que no aceptaron\;} 
}
\caption{Gale Shapley para las mujeres}
\end{Algoritmo}
\DecMargin{1em}


El siguiente ejemplo ilustra que el emparejamiento obtenido es distinto en ciertas situaciones y que por lo tanto los emparejamientos estables no son únicos. 

\begin{eje}
\label{ejeunico}
Retomando el ejemplo \ref{ejemplo matrimonio 1}. \\
Si aplicamos el algoritmo de Gale Shapley para los hombres obtenemos en primera estancia el siguiente emparejamiento estable. 
\begin{figure}[H]\centering

\begin{tikzpicture}[ scale=0.8]
\tikzset{vertex/.style = {shape=circle,draw,minimum size=1.5em}}
\tikzset{edge/.style = {->,> = latex}}
\filldraw[color=blue!60, fill=blue!5, very thick](0,3) ellipse (.8 and 1.5);
\filldraw[color=red!60, fill=red!5, very thick](4,3) ellipse (.8 and 1.5);


% vertices
% 


\node[vertex] (a) at (0,4) {$\alpha$};
\node[vertex] (b) at (0,3) {$\beta$};
\node[vertex] (c) at (0,2) {$\gamma$};


\node[vertex] (e) at (4,4) {$A$};
\node[vertex] (f) at (4,3) {$B$};
\node[vertex] (g) at (4,2) {$C$};


\node (i) at (0,5) {Hombres};
\node (j) at (4,5) {Mujeres};

\path[-stealth] (a) edge (e);
\path[-stealth] (b) edge (f);
\path[-stealth] (c) edge (g);



%\draw (0.2,8)--(3.8,8);



\end{tikzpicture}

\caption{Resultado del algoritmo si los hombres proponen.}
\end{figure}

Si aplicamos el algoritmo de Gale Shapley para las mujeres obtenemos en primera estancia el siguiente emparejamiento estable. 
\begin{figure}[H]\centering

\begin{tikzpicture}[ scale=0.8]
\tikzset{vertex/.style = {shape=circle,draw,minimum size=1.5em}}
\tikzset{edge/.style = {->,> = latex}}
\filldraw[color=blue!60, fill=blue!5, very thick](0,3) ellipse (.8 and 1.5);
\filldraw[color=red!60, fill=red!5, very thick](4,3) ellipse (.8 and 1.5);


% vertices
% 


\node[vertex] (a) at (0,4) {$\alpha$};
\node[vertex] (b) at (0,3) {$\beta$};
\node[vertex] (c) at (0,2) {$\gamma$};


\node[vertex] (e) at (4,4) {$A$};
\node[vertex] (f) at (4,3) {$B$};
\node[vertex] (g) at (4,2) {$C$};


\node (i) at (0,5) {Hombres};
\node (j) at (4,5) {Mujeres};

\path[-stealth] (g) edge (a);
\path[-stealth] (f) edge (b);
\path[-stealth] (e) edge (c);



%\draw (0.2,8)--(3.8,8);



\end{tikzpicture}

\caption{Resultado del algoritmo si las mujeres proponen.}
\end{figure}

Es fácil ver que los dos emparejamientos estables son distintos.
\fin
\end{eje}

\begin{cor}
Dada una matriz de preferencias arbitraria, la cantidad de emparejamientos estables es mayor o igual a 1. 
\end{cor}
\begin{proof}
Con el corolario \ref{corexiste} podemos garantizar la existencia y con el ejemplo \ref{ejeunico} mostramos que no podemos garantizar la unicidad de éste.
\end{proof}


Para continuar, podemos generalizar el problema del matrimonio estable a uno que permite a los hombres casarse con varias mujeres cambiando su cota superior de uno a cualquier otro número. Este es conocido como el problema de admisión a universidades. 

\thispagestyle{empty}

\chapter{El problema de admisión a universidades}
Supongamos que en una ciudad hay $n$ personas que desean entrar a $m$ universidades, los solicitantes tienen una lista de preferencias en donde reflejan a qué universidades prefieren entrar y de forma análoga las universidades tienen una lista preferencias con la información de a quién prefieren admitir. Adicional a esto para cada universidad existe un número máximo de alumnos que pueden admitir, esta restricción es natural porque la cantidad de personal y de espacio en las universidades es limitado. 
En términos matemáticos, supongamos que tenemos $n$ solicitantes $\alpha_1,\alpha_2,\ldots,\alpha_n$ y $m$ universidades $a_1, a_2,\ldots,a_m$ con las restricciones:
\begin{enumerate}
\item \begin{equation} \label{2r1}
x_{i,j}= 
\begin{cases}
1 & \qquad \text{si $i$ entra a estudiar a $j$.} \\
0 &\qquad\text{en otro caso.}\ \\ 
\end{cases} \end{equation}
\item \begin{equation} \label{2r2}
\sum_{j=1}^{m}x_{i,j} \leq1 \ \text{ para toda $i=1,2,\ldots,n$. }
\end{equation} Cada solicitante entra solo a una universidad. 
\item \begin{equation} \label{2r3}
\sum_{i=1}^{n} x_{i,j} \leq M_j\ \text{ para toda $j=1,2,\dots,m$.} 
\end{equation}
Cada universidad tiene un límite de alumnos que puede admitir. 

\end{enumerate}

Es claro que esto es una generalización del problema del matrimonio estable y un caso particular del problema de admisión a universidades con cotas inferiores y comunes. Podemos retomar las definiciones de matriz de preferencias, emparejamiento estable y  emparejamiento óptimo como análogas a las definiciones \ref{matpref}, \ref{Estable} y \ref{optima}.

En el siguiente código exhibimos un análogo al algoritmo de Gale-Shapley para este problema, el cual encuentra un emparejamiento estable y que además lo hace relativamente rápido. Para el caso general este algoritmo es conocido como el de ``aceptación diferida''.
%\pagebreak
%\begin{lstlisting}[style=R, escapeinside={(*}{*)},caption={Algoritmo de Gale Shapley para admisión a universidades}, captionpos=b, label=c2]
%Input : Una matriz de preferencias para (*$n$*) solicitantes y (*$m$*) universidades, un vector (*$M$*) en donde la entrada (*$j$*) representa la cota superior de la universidad (*$j$*).
%Output: Un emparejamiento. 
%Cada solicitante aplica a la primera universidad de su lista. Cada universidad que recibe más de solicitudes a su cota superior acepta las que solicitudes que están más arriba en su lista y rechaza al resto. 
%repeat{ #hasta que cada uno de los solicitantes sea admitido por alguna universidad o rechazado por todas.
%Los solicitantes no emparejados solicitan entrar a la siguiente universidad en su lista. 
%Cada universidad que recibe alguna solicitud acepta hasta (*$M_j$*) solicitantes de los primeros de su lista entre los que aplicaron a ella y sus admitidos actuales.
%Las universidades rechazan a los alumnos no aceptados.
%}
%\end{lstlisting}


\IncMargin{1em}
\begin{Algoritmo}[H]
%\SetKwData{Left}{left}\SetKwData{This}{this}\SetKwData{Up}{up}
%\SetKwFunction{Union}{Union}\SetKwFunction{FindCompress}{FindCompress}
\SetKwInOut{Input}{input}\SetKwInOut{Output}{output}

\Input{Una matriz de preferencias para $n$ solicitantes y $m$ universidades, un vector $M$ en donde la entrada $j$ representa la cota superior de la universidad $j$.}
\Output{Un emparejamiento.}
\BlankLine
\emph{Cada solicitante aplica a la primera universidad de su lista\; Cada universidad que recibe más de solicitudes a su cota superior acepta las que solicitudes que están más arriba en su lista y rechaza al resto\; }
\Repeat{hasta que cada uno de los solicitantes sea admitido por alguna universidad o rechazado por todas}{
	\emph{Los solicitantes no emparejados solicitan entrar a la siguiente universidad en su lista\; } 
	\emph{Cada universidad que recibe alguna solicitud acepta hasta $M_j$ solicitantes de los primeros de su lista entre los que aplicaron a ella y sus admitidos actuales\;} 
	\emph{Las universidades rechazan a los alumnos no aceptados\;} 
}
\caption{Gale Shapley  para admisión a universidades}
\end{Algoritmo}
\DecMargin{1em}


Este algoritmo al igual que su análogo en el matrimonio estable encuentra siempre un emparejamiento estable.
\begin{cor}
\label{gsu}
Dada una matriz de preferencias arbitraria y un vector $M$ de cotas superiores arbitrario, el algoritmo de Gale Shapley siempre encuentra un emparejamiento estable.
\end{cor}
\begin{proof}
Al igual que en la demostración del teorema \ref{teorema de Gale Shapley}, supongamos que el emparejamiento producido por el algoritmo no es estable. 
Esto es, un solicitante $\alpha$ que esta admitido en una universidad $B$ (o en ninguna) prefiere estar en una universidad $A$ y simultáneamente una universidad $A$ tiene admitido a un alumno $\beta$ y preferiría tener a $\alpha$ que a $\beta$ como estudiante. 

Como $\alpha$ prefiere a $A$ más que a su universidad entonces, $\alpha$ solicitó entrar primero a $A$ que a su propia universidad. 
Además, como $A$ prefiere a $\alpha$ que a $\beta$ entonces de acuerdo con el algoritmo, $A$ hubiera rechazado a $\beta$ y se hubiera quedado con $\alpha$ como alumno lo cual es una contradicción (el argumento para cuando $\alpha$ no fue aceptado en ninguna universidad es el mismo). 

Por lo tanto, el algoritmo de Gale Shapley termina siempre en un emparejamiento estable. 
\end{proof}

Este algoritmo además de ser producir un emparejamiento estable también cada solicitante está mejor o igual en este emparejamiento que en cualquier otro emparejamiento estable. Para hacer la demostración primero introducimos una definición y un lema. 

\begin{dfn}{\cite{GaleShapley}}
\label{Posible}
Decimos que una universidad es \textbf{posible} para un aplicante si existe una asignación estable en la que esta persona asiste a esa universidad.
\end{dfn}

\begin{lem} 
\label{lema optimo} 
\cite{GaleShapley} \\
Supongamos que en un paso arbitrario del algoritmo ningún estudiante ha sido rechazado por una universidad posible para él, además supongamos que una universidad $A$ después de llenarse recibiendo a los estudiantes $\beta_1,\beta_2,\dots,\beta_q$ rechaza a $\alpha$, entonces $A$ no es posible para $\alpha$.
\end{lem}
\begin{proof}
Sabemos por hipótesis que para toda $i=1,\dots,q$, $\beta_i$ prefiere a $A$ que a todas las universidades que no lo han rechazado y además que cualquier universidad que lo rechazó previamente no es posible para él. Supongamos que existe un emparejamiento estable en el que $\alpha$ asiste a $A$, entonces alguna $\beta_i$ no asiste a $A$ porque $\alpha$ tomo su lugar. Este emparejamiento es inestable porque $\beta_i$ prefiere a $A$ que a su asignación actual porque $A$ es su mejor asignación posible y $A$ prefiere tener a $\beta_i$ que a $\alpha$, lo cual es claramente una contradicción y por lo tanto $A$ no es posible para $\alpha$.
\end{proof}

\begin{teo}
\label{optimo}
\cite{GaleShapley} \\
Dada una matriz de preferencias arbitraria y un vector de cotas superiores arbitrario, el emparejamiento producido por el algoritmo es óptimo para los solicitantes.
\end{teo}
\begin{proof}
La prueba es por inducción. Primero que nada, sabemos que si en el primer paso del algoritmo una universidad rechaza a un alumno es porque esta universidad no es posible para el aplicante, si suponemos que esta universidad es posible para él llegamos rápidamente a una contradicción porque esto quisiera decir que la universidad rechazó a un mejor estudiante que la tenía como primera opción para meterlo a él y por lo tanto el emparejamiento no sería estable. Luego por el lema \ref{lema optimo} sabemos que en los siguientes pasos del algoritmo ningún estudiante es rechazado por una universidad posible para él y por lo tanto el emparejamiento obtenido es óptimo. 
\end{proof}

El algoritmo mencionado anteriormente no es único, de cambiarlo el emparejamiento obtenido podría cambiar significativamente. En primera instancia el algoritmo puede ser modificado para que sea óptimo para las universidades en lugar de para los solicitantes. El siguiente algoritmo produce este resultado y las demostraciones se hacen de forma análoga a las mencionadas anteriormente, como propiedades importantes vale la pena mencionar que produce un emparejamiento estable y que cada universidad está mejor en este emparejamiento que en cualquier otro emparejamiento. 

%\begin{lstlisting}[style=R, escapeinside={(*}{*)},caption={Algoritmo de Gale Shapley para admisión a universidades modificado}, captionpos=b, label=c3]
%Input : Una matriz de preferencias para (*$n$*) solicitantes y (*$m$*) universidades, un vector (*$M$*) en donde la entrada (*$j$*) representa la cota superior de la universidad (*$j$*).
%Output: Un emparejamiento. 
%Cada universidad invita a los primeros (*$j_i$*) solicitantes en su lista a estudiar ahí. Cada alumno que recibe más de una solicitud acepta las más alta en su lista y rechaza el resto. 
%repeat{ #hasta que cada universidad este llene o cuando todos los estudiantes sean admitidos en una universidad.
%Las universidades que no alcanzaron su cota superior invitan a los siguientes alumnos de su lista de acuerdo con cuantos lugares tienen.
%Cada solicitante que recibe una solicitud acepta la más alta en su alta entre su actual universidad y las que lo invitaron. 
%El alumno rechaza el resto de las invitaciones.
%}
%\end{lstlisting}

\IncMargin{1em}
\begin{Algoritmo}[H]
%\SetKwData{Left}{left}\SetKwData{This}{this}\SetKwData{Up}{up}
%\SetKwFunction{Union}{Union}\SetKwFunction{FindCompress}{FindCompress}
\SetKwInOut{Input}{input}\SetKwInOut{Output}{output}

\Input{Una matriz de preferencias para $n$ solicitantes y $m$ universidades, un vector $M$ en donde la entrada $j$ representa la cota superior de la universidad $j$.}
\Output{Un emparejamiento.}
\BlankLine
\emph{Cada universidad invita a los primeros $j_i$ solicitantes en su lista a estudiar ahí \; Cada alumno que recibe más de una solicitud acepta las más alta en su lista y rechaza el resto\; }
\Repeat{hasta que cada universidad esté llena o cuando todos los estudiantes sean admitidos en una universidad.}{
	\emph{Las universidades que no alcanzaron su cota superior invitan a los siguientes alumnos de su lista de acuerdo con cuántos lugares tienen\; } 
	\emph{Cada solicitante que recibe una solicitud acepta la más alta entre su actual universidad y las que lo invitaron\;} 
	\emph{El alumno rechaza el resto de las invitaciones\;} 
}
\caption{Gale Shapley para admisión a universidades modificado}
\end{Algoritmo}
\DecMargin{1em}

\section{El teorema de los hospitales rurales}

Existe una extensión de este problema llamada el problema de hospitales y residentes médicos que es básicamente lo mismo a este problema con la excepción de que en vez de tener universidades tenemos hospitales y en vez de que existan estudiantes queriendo estudiar en ésta tenemos estudiantes médicos que quieren hacer su residencia en éste. Una pregunta interesante que se hizo fue ¿qué pasa con los hospitales no populares? Algunos hospitales por su naturaleza no llamaban la atención de los estudiantes de medicina, un ejemplo de esto eran los que estaban en una zona rural, los residentes preferían ir a hospitales en zonas urbanas más pobladas porque contaban con fama de ser mejores hospitales y además tenían más pacientes. La pregunta original puede ser modificada en ¿existe una modificación al algoritmo que produzca un emparejamiento estable y además asigne más residentes en los hospitales poco populares que el resultado del algoritmo Gale Shapley? A raíz de esto David Gale y Marilda Sotomayor en 1985 demostraron que la cantidad de gente que asiste a la universidad es la misma en cualquier emparejamiento estable. Este resultado se puede poner en términos de nuestro problema original, para llegar a éste es necesario primero demostrar un lema.

\begin{lem} 
\label{lema rural} 
\cite{Verde} \\
Dada una matriz de preferencias arbitraria y un vector de cotas superiores arbitrario, sea $M$ el emparejamiento obtenido por el algoritmo de Gale Shapley para admisión a universidades y sea $M'$ un emparejamiento estable arbitrario. Si una universidad $A$ no se llena en $M'$ entonces todo solicitante que entró a $A$ en $M$ también fue asignado a $A$ en $M'$. 
\end{lem}
\begin{proof}
Supongamos que $\alpha$ fue admitido en $A$ en la asignación $M$ pero no en $M'$, por hipótesis $A$ no está llena en $M'$ entonces tenemos una contradicción porque significaría que $\alpha$ en $M'$ está en una mejor universidad posible que $A$, lo cual no es posible porque la asignación en $M$ por el teorema \ref{optimo} es óptima. Por lo tanto, si una universidad $A$ no se llena en $M'$ entonces todo solicitante que entró a $A$ en $M$ también fue asignado a $A$ en $M'$. 
\end{proof}


\begin{teo}{Teorema de los hospitales rurales \\ }
\label{rural}
\cite{GaleSotomayor}\\
Dada una matriz de preferencias arbitraria y un vector de cotas superiores arbitrario:
\begin{enumerate}
\item Cada universidad recibe el mismo número de solicitantes en cada asignación estable.
\item Exactamente el mismo número de solicitantes quedan como no asignados en cada emparejamiento estable. 
\item Si una universidad no se llena en un emparejamiento estable recibe exactamente el mismo número de solicitantes en cualquier asignación estable. 
\end{enumerate}
\end{teo}
\begin{proof}
Sea M el emparejamiento obtenido por Gale Shapley y sea $M'$ un emparejamiento estable arbitrario. Primero notamos que si una persona no fue asignada a ninguna universidad en $M$ entonces tampoco en $M'$, esto es por el lema \ref{lema optimo} porque ninguna universidad es posible para él. Por lo tanto, el número de personas asignadas en $M'$ no puede exceder el de $M$. 

Ahora por el lema \ref{lema rural}, si una universidad se llena en $M$ entonces también está llena en $M'$. Simultáneamente por el lema \ref{lema rural}, si a una universidad le sobran lugares en $M'$ entonces en $M$ tiene mínimo la misma cantidad de lugares. Por lo tanto, el número de personas asignadas en $M$ no puede exceder el de $M'$. Luego, podemos concluir que para toda universidad tiene el mismo número de estudiantes asignados en $M$ y en $M'$ que es lo mismo a cada universidad recibe el mismo número de solicitantes en cada asignación estable. 

A partir de esto concluimos que exactamente el mismo número de solicitantes quedan como no asignados en cada emparejamiento estable dado que ningún estudiante que es no asignado en $M$ puede ser admitido en $M'$. También vemos (\ref{lema rural}) que si una universidad no se llena en un emparejamiento estable recibe exactamente el mismo número de solicitantes en cualquier asignación estable. 

Por lo tanto, concluimos que se cumplen 1, 2 y 3.
\end{proof}

Para seguir avanzando, plantearemos algunas variantes de este problema y veremos si su comportamiento es similar al antes visto. 

\thispagestyle{empty}

\chapter{Listas con empates}

Retomemos el problema planteado en el capítulo \ref{me}, recordemos que se planteó el problema del matrimonio estable y se demostró que sin importar si las listas de preferencia son completas o incompletas, existe un algoritmo polinomial que siempre encuentra un emparejamiento estable óptimo (en el sentido de los hombres o de las mujeres dependendiendo de qué algoritmo se use). \\

Ahora, consideremos el caso en el que la lista de preferencias es completa, es decir cada hombre y cada mujer tiene catalogado en su orden de preferencia a cada persona del género opuesto. Incluiremos la variante de que en la lista se pueden introducir empates, esto es muy común dada que en muchas ocasiones la diferencia entre escoger a dos personas puede ser no significativa. 

\begin{eje}{\cite{Verde}}
\label{ejemplo empates}
Supongamos que para 2 hombres y 2 mujeres tenemos la matriz de preferencias 
$$\begin{pmatrix}
& A & B \\
\alpha & 1,2 & 2,2 \\
\beta & 1,1 & 1,1\\
\end{pmatrix}.$$

En este caso el orden de preferencias $A$ y $B$ es $(\beta, \alpha)$, el orden de preferencia de $\alpha$ es $(A,B)$ y el orden de preferencia de $\beta$ es $([A,B])$\footnote{Los corchetes denotan que la lista contiene empates}.
\fin
\end{eje}

Para este caso es necesario cambiar un poco la definición de estabilidad y de inestabilidad. 

\begin{dfn}{\cite{Verde}}
\label{estricto}
Decimos que un emparejamiento es inestable si existen un hombre $\alpha$ y una mujer $\beta$, donde cada uno de ellos \textbf{estrictamente} prefiere al otro que a su respectiva pareja. \\
Alternativamente un emparejamiento es estable si no es inestable.
\end{dfn}

La única diferencia con la definición \ref{Estable} es la palabra estrictamente, este tipo de emparejamientos en muchas ocasiones se conoce como \textbf{estabilidad débil}\footnote{Si no se pide la preferencia estricta es posible ver que en muchas ocasiones no existe un emparejamiento estable para el problema (ver \cite{Verde}).}. En el caso de que no existan empates en las listas de preferencias las dos defunciones son idénticas. El siguiente resultado muestra que el algoritmo de Gale Shapley también sirve para encontrar un emparejamiento estable dadas estas condiciones.

\begin{cor}
Dada una matriz de preferencias arbitraria completa en la que se permiten empates con $n$ hombres y $m$ mujeres, el algoritmo de Gale Shapley converge a un emparejamiento estable.
\end{cor}
\begin{proof}
De la definición \ref{estricto} podemos ver que si convertimos los empates en órdenes estrictos de forma arbitraria llegamos a una instancia del problema planteado en el capítulo \ref{me}. Si aplicamos el resultado del teorema \ref{teorema de Gale Shapley} podemos ver que el algoritmo converge a un emparejamiento estable. \\
\end{proof}

Una nota importante de la demostración de arriba es que si cambiamos los empates en ordenes distintos podríamos llegar a emparejamientos estables diferentes. 





\begin{eje}
Consideremos la lista de preferencias del ejemplo \ref{ejemplo empates}, la modificamos para que permita un orden estricto obteniendo la siguiente matriz de preferencias
$$\begin{pmatrix}
& A & B \\
\alpha & 1,2 & 2,2 \\
\beta & 1,1 & 2,1\\
\end{pmatrix}.$$
En la primera iteración, $\alpha$ y $\beta$ le proponen a $A$.

\begin{figure}[H]
\centering

\begin{tikzpicture}[ scale=0.8]
\tikzset{vertex/.style = {shape=circle,draw,minimum size=1.5em}}
\tikzset{edge/.style = {->,> = latex}}
\filldraw[color=blue!60, fill=blue!5, very thick](0,2.5) ellipse (.8 and 2);
\filldraw[color=red!60, fill=red!5, very thick](4,2.5) ellipse (.8 and 2);


% vertices
% 


\node[vertex] (a) at (0,4) {$\alpha$};
\node[vertex] (b) at (0,2) {$\beta$};


\node[vertex] (e) at (4,4) {$A$};
\node[vertex] (f) at (4,2) {$B$};

%\draw (0,.5) node[cross,red] {};

\node (i) at (0,5) {Hombres};
\node (j) at (4,5) {Mujeres};

\path[-stealth] (a) edge (e);
\path[-stealth] (b) edge (e);




%\draw (0.2,8)--(3.8,8);



\end{tikzpicture}

\caption{Primera iteración.}
\end{figure}
En la segunda iteración, $A$ acepta la propuesta de $\beta$ y $\alpha$ le propone matrimonio a $B$.

\begin{figure}[H]
\centering

\begin{tikzpicture}[ scale=0.8]
\tikzset{vertex/.style = {shape=circle,draw,minimum size=1.5em}}
\tikzset{edge/.style = {->,> = latex}}
\filldraw[color=blue!60, fill=blue!5, very thick](0,2.5) ellipse (.8 and 2);
\filldraw[color=red!60, fill=red!5, very thick](4,2.5) ellipse (.8 and 2);


% vertices
% 


\node[vertex] (a) at (0,4) {$\alpha$};
\node[vertex] (b) at (0,2) {$\beta$};


\node[vertex] (e) at (4,4) {$A$};
\node[vertex] (f) at (4,2) {$B$};

%\draw (0,.5) node[cross,red] {};

\node (i) at (0,5) {Hombres};
\node (j) at (4,5) {Mujeres};

\path[-stealth] (a) edge (f);
\path[-stealth] (b) edge (e);




%\draw (0.2,8)--(3.8,8);



\end{tikzpicture}

\caption{Segunda iteración.}
\end{figure}

Es claro que el emparejamiento obtenido después de dos iteraciones es estable.
\fin
\end{eje}

Otro resultado importante de este problema es que sigue cumpliendo el teorema \ref{rural}, es decir dados dos emparejamientos estables la cantidad de hombres emparejados es la misma.

\begin{cor}
Dada una matriz de preferencias completa en la que se permiten empates con $n$ hombres y $m$ mujeres. Si tenemos dos emparejamientos estables $M_1$ y $M_2$, la cantidad de hombres emparejados es la misma.
\end{cor}
\begin{proof}
Dada que la lista de preferencias es completa, todos los hombres prefieren tener a una pareja (sin importar que mujer les toque) a quedarse solos, lo mismo es cierto para las mujeres. Por lo tanto, el número de hombres emparejados es el mínimo entre el número de hombres y el número de mujeres. 
\end{proof}

\section{Listas incompletas con empates}
Ahora, consideremos el problema del matrimonio estable considerando dos variantes para el problema; la lista de preferencias es incompleta, o sea para alguna persona en el problema existe otra persona del género opuesto de la cual prefiere quedarse no emparejado a terminar emparejado con ella y además se permiten empates en la lista del mismo modo que se planteó al principio del capítulo. 

\begin{eje}{\cite{empates}}
Supongamos que para 2 hombres y 2 mujeres tenemos la matriz de preferencias 
$$\begin{pmatrix}
& A & B \\
\alpha & 1,1 & , \\
\beta & 1,1 & 2,1\\
\end{pmatrix}.$$
En este caso el orden de preferencias de $\alpha$ es $(A)$, el orden de preferencias de $\beta$ es $(A,B)$, el orden de preferencias de $A$ es $([\alpha,\beta])$ y el orden de preferencias de $B$ es $(\alpha)$. \\
Para este problema existen dos emparejamientos estables:
\begin{figure}[H]
\centering

\begin{tikzpicture}[ scale=0.8]
\tikzset{vertex/.style = {shape=circle,draw,minimum size=1.5em}}
\tikzset{edge/.style = {->,> = latex}}
\filldraw[color=blue!60, fill=blue!5, very thick](0,2.5) ellipse (.8 and 2);
\filldraw[color=red!60, fill=red!5, very thick](4,2.5) ellipse (.8 and 2);


% vertices
% 


\node[vertex] (a) at (0,4) {$\alpha$};
\node[vertex] (b) at (0,2) {$\beta$};


\node[vertex] (e) at (4,4) {$A$};
\node[vertex] (f) at (4,2) {$B$};

%\draw (0,.5) node[cross,red] {};

\node (i) at (0,5) {Hombres};
\node (j) at (4,5) {Mujeres};

\path[-stealth] (b) edge (e);
%\path[-stealth] (b) edge (e);




%\draw (0.2,8)--(3.8,8);



\end{tikzpicture}

\caption{Primer emparejamiento estable.}
\end{figure}

\begin{figure}[H]
\centering

\begin{tikzpicture}[ scale=0.8]
\tikzset{vertex/.style = {shape=circle,draw,minimum size=1.5em}}
\tikzset{edge/.style = {->,> = latex}}
\filldraw[color=blue!60, fill=blue!5, very thick](0,2.5) ellipse (.8 and 2);
\filldraw[color=red!60, fill=red!5, very thick](4,2.5) ellipse (.8 and 2);


% vertices
% 


\node[vertex] (a) at (0,4) {$\alpha$};
\node[vertex] (b) at (0,2) {$\beta$};


\node[vertex] (e) at (4,4) {$A$};
\node[vertex] (f) at (4,2) {$B$};

%\draw (0,.5) node[cross,red] {};

\node (i) at (0,5) {Hombres};
\node (j) at (4,5) {Mujeres};

\path[-stealth] (a) edge (e);
\path[-stealth] (b) edge (f);




%\draw (0.2,8)--(3.8,8);



\end{tikzpicture}

\caption{Segundo emparejamiento estable.}
\end{figure}
\fin
\end{eje}

Un resultado directo mostrado en el ejemplo de arriba es el siguiente corolario.

\begin{cor}
Dada una matriz de preferencias incompleta en la que se permiten empates para $n$ hombres y $m$ mujeres. Si tenemos dos emparejamientos estables $M_1$ y $M_2$, la cantidad de hombres emparejados no necesariamente es la misma.
\end{cor}

Un problema interesante aquí es encontrar \textbf{un emparejamiento estable de máxima cardinalidad}, o sea encontrar algún emparejamiento estable $M_1$ con la propiedad de que no existe ningún otro emparejamiento $M_2$ estable en el problema en el cual exista una mayor cantidad de hombres emparejados que en $M_1$. El siguiente resultado muestra que encontrar una solución para este problema no es nada trivial.

\begin{teo} {\cite{empates}} \label{NPempates}
Dada una matriz de preferencias incompleta en la que se permiten empates para $n$ hombres y $n$ mujeres. El problema de encontrar un emparejamiento estable de máxima cardinalidad es NP-duro. Este resultado es correcto inclusive si solo existen empates en las listas de las mujeres (o de forma análoga en las listas de los hombres). 
\end{teo}
\begin{proof}
La demostración supera el alcance de esta tesis, pero se puede encontrar en \cite{empates}. 
\end{proof}
















\thispagestyle{empty}

\chapter{Cotas inferiores}

Retomamos el problema de admisión a universidades igual que como se plantea en el capíitulo 4, a cada universidad se le agrega la restricción de que necesita un número mínimo de alumnos asignados que necesitan para abrir. En particular se le agregan las siguientes tres ecuaciónes al problema:

\begin{enumerate}
\item \begin{equation} y_{j}= 
\begin{cases}
1 & \qquad \text{si la universidad $j$ abre.} \\
0 &\qquad\text{en otro caso.} \\ 
\end{cases} \end{equation} 
\item \begin{equation} \label{r6}
x_{i,j} \leq y_j \ \text{ para toda $i=1,2,\ldots,n$ y para toda $j=1,2,\ldots,m$.}
\end{equation}
Los solicitantes únicamente pueden asistir a universidades abiertas.
\item \begin{equation} \label{r4}
\sum_{i=1}^{n} x_{i,j} \geq\ m_j\times y_j 
\end{equation}
Cada universidad necesita admitir a una cantidad considerable de alumnos para abrir.
\end{enumerate}

Una pregunta inicial que surge es si todos los resultados obtenidos para el problema sin cotas inferiores o comunes se siguen cumpliendo, una vez que agregamos las cotas inferiores ¿el emparejamiento estable siempre existe? ¿De existir este es óptimo? ¿Se obtiene con facilidad? Para comenzar es posible mostrar que el emparejamiento estable no siempre existe.

\begin{eje}
\cite{Todo}\\
Supongamos que tenemos dos universidades $A$ y $B$, $A$ solo puede admitir a dos alumnos y necesita admitir a dos alumnos para entrar y $B$ solo puede admitir a un alumno y necesita a un alumno para abrir. Además, contamos con dos solicitantes $\alpha$ y $\beta$, la matriz de preferencias del problema está dada por 
$$\begin{pmatrix}
& A & B \\
\alpha & 1,1 & 2,1 \\
\beta & 2,2 & 1,2 \\ 
\end{pmatrix}.$$
Lo primero que notamos es que $A$ o $B$ tienen que estar cerradas porque la suma de las cotas inferiores es tres y solo contamos con dos estudiantes. Si $A$ cierra entonces tenemos dos emparejamientos posibles, en el primero $\alpha$ es admitido en $B$ y no es estable porque $\alpha$ prefiere estar en $A$, $\beta$ prefiere estar en $A$ a no estar en ningún lado y $A$ prefiere tener a los dos estudiantes que cerrar . 
\begin{figure}[H]\centering

\begin{tikzpicture}[ scale=0.8]
\tikzset{vertex/.style = {shape=circle,draw,minimum size=1.5em}}
\tikzset{edge/.style = {->,> = latex}}
\filldraw[color=blue!60, fill=blue!5, very thick](0,3) ellipse (.8 and 1.5);
\filldraw[color=green!60, fill=green!5, very thick](4,3) ellipse (.8 and 1.5);


% vertices
% 


\node[vertex] (a) at (0,4) {$\alpha$};
\node[vertex] (b) at (0,2) {$\beta$};


\node[vertex] (e) at (4,4) {$A$};
\node[vertex] (f) at (4,2) {$B$};


\node (i) at (0,5) {Solicitantes};
\node (j) at (4,5) {Universidades};

\path[-stealth] (a) edge (f);
%\path[-stealth] (b) edge (g);

\draw (4,4) node[cross=8pt,red] {};


%\draw (0.2,8)--(3.8,8);



\end{tikzpicture}

\caption{Primer caso.}
\end{figure}
En el segundo caso $\beta$ es admitido en $B$ y este emparejamiento no es estable porque $B$ prefiere tener a $\alpha$ que a $\beta$ y $\alpha$ prefiere estudiar que no estudiar. 
\begin{figure}[H]\centering

\begin{tikzpicture}[ scale=0.8]
\tikzset{vertex/.style = {shape=circle,draw,minimum size=1.5em}}
\tikzset{edge/.style = {->,> = latex}}
\filldraw[color=blue!60, fill=blue!5, very thick](0,3) ellipse (.8 and 1.5);
\filldraw[color=green!60, fill=green!5, very thick](4,3) ellipse (.8 and 1.5);


% vertices
% 


\node[vertex] (a) at (0,4) {$\alpha$};
\node[vertex] (b) at (0,2) {$\beta$};


\node[vertex] (e) at (4,4) {$A$};
\node[vertex] (f) at (4,2) {$B$};


\node (i) at (0,5) {Solicitantes};
\node (j) at (4,5) {Universidades};

\path[-stealth] (b) edge (f);
%\path[-stealth] (b) edge (g);

\draw (4,4) node[cross=8pt,red] {};


%\draw (0.2,8)--(3.8,8);



\end{tikzpicture}

\caption{Segundo caso.}
\end{figure}


Si $B$ cierra entonces solo tenemos un emparejamiento posible en el que $\alpha$ y $\beta$ son admitidos en $A$, este no es estable porque $B$ prefiere tener a $\beta$ que cerrar y $\beta$ prefiere estar en $A$ que en $B$.

\begin{figure}[H]\centering

\begin{tikzpicture}[ scale=0.8]
\tikzset{vertex/.style = {shape=circle,draw,minimum size=1.5em}}
\tikzset{edge/.style = {->,> = latex}}
\filldraw[color=blue!60, fill=blue!5, very thick](0,3) ellipse (.8 and 1.5);
\filldraw[color=green!60, fill=green!5, very thick](4,3) ellipse (.8 and 1.5);


% vertices
% 


\node[vertex] (a) at (0,4) {$\alpha$};
\node[vertex] (b) at (0,2) {$\beta$};


\node[vertex] (e) at (4,4) {$A$};
\node[vertex] (f) at (4,2) {$B$};


\node (i) at (0,5) {Solicitantes};
\node (j) at (4,5) {Universidades};

\path[-stealth] (a) edge (e);
\path[-stealth] (b) edge (e);

\draw (4,2) node[cross=8pt,red] {};


%\draw (0.2,8)--(3.8,8);



\end{tikzpicture}

\caption{Tercer caso.}
\end{figure}

\fin
\end{eje}

A partir de este ejemplo queda claro que si extendemos el problema agregando cotas inferiores perdemos la garantía de existencia de un emparejamiento estable en nuestro problema. Es posible además construir un algoritmo a partir del algoritmo de Gale Shapley para admisión a universidades que converja en caso de que exista a un emparejamiento estable o que saque como resultado si no existe un emparejamiento estable para este problema, el algoritmo esta dado por el siguiente código. Antes de mostrarlo vale la pena recordar que si tienes $m$ universidades entonces existen $2^{m}-1$ (corolario \ref{card3}) posibles combinaciones de qué universidades mantienen abiertas y cuales cierran. 
%\begin{lstlisting}[style=R, escapeinside={(*}{*)},caption={Algoritmo de Gale Shapley para admisión a universidades con cotas inferiores}, captionpos=b, label=c3]
%Input : Una matriz de preferencias para (*$n$*) solicitantes y (*$m$*) universidades, un vector (*$M$*) en donde la entrada (*$j$*) representa la cota superior de la universidad (*$j$*) y un vector (*$N$*) en donde la entrada (*$j$*) representa la cota inferior de la universidad (*$j$*).
%Output: Un emparejamiento o mensaje. 
%repeat{ #Hasta encontrar un emparejamiento estable factible.
%Se selecciona una de las (*$2^{m}-1$*) combinaciones, se escoge primero las opciones que cierren menos universidades de las todavía no escogidas. 
%Cada universidad abierta invita a los primeros (*$j_i$*) solicitantes en su lista a estudiar ahí. Cada alumno que recibe más de una solicitud acepta las más alta en su lista y rechaza el resto. 
%repeat{ #hasta que cada universidad abierta este llena o cuando todos los estudiantes sean admitidos en una universidad.
%Las universidades que no alcanzaron su cota superior invitan a los siguientes alumnos de su lista de acuerdo con cuantos lugares tienen.
%Cada solicitante que recibe una solicitud acepta la más alta en su alta entre su actual universidad y las que lo invitaron. 
%El alumno rechaza el resto de las invitaciones.
%}
%Si la asignación obtenida es factible, el algoritmo acaba y saca como output el emparejamiento. 
%}
%Si el algoritmo trato todas las combinaciones y no encontró un emparejamiento estable factible entonces saca un mensaje diciendo que este no existe.
%\end{lstlisting}
\vspace{1 cm}

\IncMargin{1em}
\begin{Algoritmo}[H]
%\SetKwData{Left}{left}\SetKwData{This}{this}\SetKwData{Up}{up}
%\SetKwFunction{Union}{Union}\SetKwFunction{FindCompress}{FindCompress}
\SetKwInOut{Input}{input}\SetKwInOut{Output}{output}

\Input{Una matriz de preferencias para $n$ solicitantes y $m$ universidades, un vector $M$ en donde la entrada $j$ representa la cota superior de la universidad $j$ y un vector $N$ en donde la entrada $j$ representa la cota inferior de la universidad $j$.}
\Output{Un emparejamiento o un mensaje.}
\BlankLine
\Repeat{Hasta encontrar un emparejamiento estable factible o hasta tratar todas las combinaciones}{
	\emph{Se selecciona una de las $2^{m}-1$ combinaciones, se escoge primero las opciones que cierren menos universidades de las todavía no escogidas\;}
	\emph{Se le aplica el algoritmo de Gale Shapley para admisión de universidades a la combinación de universidades abiertas y se obtiene una asiignación\;}
	\emph{Si la asignación obtenida es factible, el algoritmo acaba y saca como output el emparejamiento\;}
	}
	\emph{Si el algoritmo trato todas las combinaciones y no encontró un emparejamiento estable factible entonces saca un mensaje diciendo que este no existe\;}
\caption{Gale Shapley  para admisión a universidades con cotas inferiores.}
\end{Algoritmo}
\DecMargin{1em}

\begin{cor}
Dada una matriz de preferencias arbitraria, un vector de cotas superiores arbitrario y un vector de cotas inferiores arbitrario, el algoritmo de Gale Shapley para admisión a universidades con cotas inferiores converge.
\end{cor}
\begin{proof}
Supongamos que existe una combinación de universidades abiertas que cuenta con una asignación estable y el algoritmo no la encuentra. Esto es, existe una combinación de universidades en las que el emparejamiento estable producido por Gale Shapley no es factible, pero en la que si existe un emparejamiento estable factible, lo cual es una contradicción al teorema \ref{rural} porque todos los emparejamientos estables asignan exactamente el mismo número de personas a cada universidad y por lo tanto si el emparejamiento encontrado por el algoritmo no es factible eso implicaría que todas las asignaciones estables no son factibles. \\
Cómo se escoge las opciones que cierren menos universidades de las todavía no escogidas podemos garantizar que de encontrar un emparejamiento estable factible este no tiene una universidad bloqueadora. 
Además, porque el algoritmo trata todas las combinaciones y porque el corolario \ref{gsu} nos garantiza que dada una combinación de universidades abiertas el algoritmo siempre encuentra un emparejamiento estable, entonces podemos concluir que de existir un emparejamiento estable factible el algoritmo de Gale Shapley para admisión a universidades con cotas inferiores lo va a encontrar y por lo tanto el algoritmo converge a lo deseado. 
\end{proof}

Este algoritmo a diferencia de los anteriores en que tiene la enorme desventaja de que no es muy eficiente, si el número de universidades es muy grande el algoritmo podría no acabar porque el poder de computo necesario para correrlo es exponencial respecto al número de universidades, esto queda fundamentado por el siguiente resultado.
\begin{cor}
La complejidad del algoritmo de Gale Shapley para admisión a universidades con cotas inferiores es del orden de $2^{m}\cdot m^2$, donde $m$ es el número de universidades.
\end{cor}
\begin{proof}
En el peor escenario no existe una asignación estable factible para el problema, en este caso se realiza $2^{m}-1$ veces el algoritmo de Gale Shapley, que por la observación \ref{complejidad1} tiene como complejidad $m^2$. Por lo tanto, el número máximo de pasos que tiene el algoritmo es aproximadamente $(2^{m}-1)\cdot m^2=2^{m}\cdot m^2-m^2$ y podemos concluir que la complejidad del algoritmo es del orden de $2^{m}\cdot m^2$.
\end{proof}

A continuación mostramos un ejemplo de cómo funciona el algoritmo para dejarlo más claro. 
\begin{eje}
\cite{Todo}\\
Supongamos que tenemos dos universidades $A$ y $B$, $A$ tiene como cota inferior a dos estudiantes y solo puede admitir a dos solicitantes, $B$ necesita a tres estudiantes para abrir y tiene como cota superior a tres alumnos. Además, a esto contamos con tres solicitantes $\alpha$, $\beta$ y $\gamma$, la matriz de preferencias para este problema es 
$$\begin{pmatrix}
& A & B \\
\alpha & 1,1 & ,1 \\
\beta & 1,2 & 2,1 \\ 
\gamma & , & 1,2 \\ 
\end{pmatrix}$$
En el primer paso del algoritmo las dos universidades se mantienen abiertas, $\alpha$ aplica a $A$, $\beta$ aplica a $A$ y $\gamma$ aplica a $B$. $A$ y $B$ aceptan a los tres alumnos llegando a una asignación estable pero no factible. 
\begin{figure}[H]\centering

\begin{tikzpicture}[ scale=0.8]
\tikzset{vertex/.style = {shape=circle,draw,minimum size=1.5em}}
\tikzset{edge/.style = {->,> = latex}}
\filldraw[color=blue!60, fill=blue!5, very thick](0,3) ellipse (.8 and 1.5);
\filldraw[color=green!60, fill=green!5, very thick](4,3) ellipse (.8 and 1.5);


% vertices
% 


\node[vertex] (a) at (0,4) {$\alpha$};
\node[vertex] (b) at (0,3) {$\beta$};
\node[vertex] (c) at (0,2) {$\gamma$};


\node[vertex] (e) at (4,4) {$A$};
\node[vertex] (f) at (4,2) {$B$};


\node (i) at (0,5) {Solicitantes};
\node (j) at (4,5) {Universidades};

\path[-stealth] (a) edge (e);
\path[-stealth] (b) edge (e);
\path[-stealth] (c) edge (f);

%\draw (4,2) node[cross=8pt,red] {};


%\draw (0.2,8)--(3.8,8);



\end{tikzpicture}

\caption{Primera iteración.}
\end{figure}

En el segundo paso del algoritmo $A$ cierra y $B$ se mantiene abierto. $\alpha$ no aplica a ninguna universidad, $\beta$ aplica a $B$ y $\gamma$ aplica a $B$. $B$ acepta a los dos alumnos llegando a un emparejamiento estable pero no factible.
\begin{figure}[H]\centering

\begin{tikzpicture}[ scale=0.8]
\tikzset{vertex/.style = {shape=circle,draw,minimum size=1.5em}}
\tikzset{edge/.style = {->,> = latex}}
\filldraw[color=blue!60, fill=blue!5, very thick](0,3) ellipse (.8 and 1.5);
\filldraw[color=green!60, fill=green!5, very thick](4,3) ellipse (.8 and 1.5);


% vertices
% 


\node[vertex] (a) at (0,4) {$\alpha$};
\node[vertex] (b) at (0,3) {$\beta$};
\node[vertex] (c) at (0,2) {$\gamma$};


\node[vertex] (e) at (4,4) {$A$};
\node[vertex] (f) at (4,2) {$B$};


\node (i) at (0,5) {Solicitantes};
\node (j) at (4,5) {Universidades};

%\path[-stealth] (a) edge (e);
\path[-stealth] (b) edge (f);
\path[-stealth] (c) edge (f);

\draw (4,4) node[cross=8pt,red] {};


%\draw (0.2,8)--(3.8,8);



\end{tikzpicture}

\caption{Segunda iteración.}
\end{figure}

En el tercer paso del algoritmo $B$ cierra y $A$ se mantiene abierto. $\gamma$ no aplica a ninguna universidad, $\alpha$ aplica a $A$ y $\alpha$ aplica a $A$. $A$ acepta a los dos alumnos llegando a un emparejamiento estable factible. Después de 3 pasos en el
algoritmo, encontramos un emparejamiento estable factible, es decir, el algoritmo convergió.
\begin{figure}[H]\centering

\begin{tikzpicture}[ scale=0.8]
\tikzset{vertex/.style = {shape=circle,draw,minimum size=1.5em}}
\tikzset{edge/.style = {->,> = latex}}
\filldraw[color=blue!60, fill=blue!5, very thick](0,3) ellipse (.8 and 1.5);
\filldraw[color=green!60, fill=green!5, very thick](4,3) ellipse (.8 and 1.5);


% vertices
% 


\node[vertex] (a) at (0,4) {$\alpha$};
\node[vertex] (b) at (0,3) {$\beta$};
\node[vertex] (c) at (0,2) {$\gamma$};


\node[vertex] (e) at (4,4) {$A$};
\node[vertex] (f) at (4,2) {$B$};


\node (i) at (0,5) {Solicitantes};
\node (j) at (4,5) {Universidades};

\path[-stealth] (a) edge (e);
\path[-stealth] (b) edge (e);
%\path[-stealth] (c) edge (f);

\draw (4,2) node[cross=8pt,red] {};


%\draw (0.2,8)--(3.8,8);



\end{tikzpicture}

\caption{Tercera iteración.}
\end{figure}


\fin
\end{eje}


Al igual que en las secciones anteriores el algoritmo puede ser modificado para que sea optimo para las universidades, los resultados de convergencia y de complejidad son analogos a los mostrados.
Además a esto, \cite{Todo} propone una heuristica para resolver este problema de forma rapida. En general hay que tener cuidado al usarla porque el algoritmo ignora la existencia de universidades bloqueadoras y podria no llegar a una solución a pesar de que para ese problema en particular si exista una asignación estable.
\\
\IncMargin{1em}
\begin{Algoritmo}[H]
%\SetKwData{Left}{left}\SetKwData{This}{this}\SetKwData{Up}{up}
%\SetKwFunction{Union}{Union}\SetKwFunction{FindCompress}{FindCompress}
\SetKwInOut{Input}{input}\SetKwInOut{Output}{output}
\Input{Una matriz de preferencias para $n$ solicitantes y $m$ universidades, un vector $M$ en donde la entrada $j$ representa la cota superior de la universidad $j$ y un vector $N$ en donde la entrada $j$ representa la cota inferior de la universidad $j$.}
\Output{Un emparejamiento o un mensaje.}
\BlankLine
\emph{$U$ $\gets$ un vector de tamaño $m$ con los nombres de todas las universidades\;}
\Repeat{Hasta encontrar un emparejamiento estable factible o mientras $U$ sea no vacio}{
	\emph{Se aplica el algoritmo de Gale Shapley para admisión a universidades considerando que las escuelas que aparecen en $U$ estan abiertas y las que no estan cerradas obteniendo un emparejamiento $M$\;}
	\eIf{$M$ es factible}{
		El algoritmo acaba y saca como output el emparejamiento\;
		}
		{
		\emph{Sea $A$ la universidad con el cociente entre el número asignado de estudiantes y su cota inferior mínimo \;}
		\emph{$U$ $\gets$ $U \setminus A$ (Al vector $U$ le quitamos la universidad $A$) \;} 
		}
}
\emph{Si $U$ esta vacio, saca un mensaje diciendo que no se encontro el emparejamiento\;}
\caption{Heuristica  para admisión a universidades con cotas inferiores.}
\end{Algoritmo}
\DecMargin{1em}

%no siempre converge
%heuristicas
%es np completo


\thispagestyle{empty}

\chapter{Cotas comunes}

Retomamos el problema de admisión a universidades igual que como se plantea en el capítulo 4, se agrega la restricción de que varias universidades en conjunto pueden compartir cotas superiores, estas restricciones son de carácter natural porque las universidades pueden tener recursos compartidos como es el caso, por ejemplo, de becas gubernamentales . En particular se le agregan la siguiente restricción al problema:


%Para abreviar llamaremos a este problema \textbf{CA-CQ}.
\begin{equation}
\sum_{i=1}^{n} \sum_{j=1}^m w_{j,k} \cdot x_{i,j} \leq N_p %\cdot z_k
\end{equation} 
donde \begin{equation} w_{j,k}= 
\begin{cases}
1 & \qquad \text{si la universidad $j$ es parte de la restricción $k$.} \\
0 &\qquad\text{en otro caso.} \\ 
\end{cases} \end{equation} \\ para toda $k=1,2,\dots,p$ \footnote{Donde $p$ es el número de restricciones comunes.}. \\
Las universidades tienen cotas superiores comunes.
%\item \begin{equation} 
%z_{k}= 
%\begin{cases}
%1 & \qquad \text{si las universidades en la restricción $k$ abren.} \\
%0 &\qquad\text{en otro caso.}\ \\ 
%\end{cases} \end{equation}

%\item \begin{equation}
%\sum_{j=1}^{m} w_{j,k} y_{j} \leq z_{k} \text{ para toda $k=1,2,\dots,p$. }
%\end{equation} \\ Las universidades solo pueden abrir si cumplen su cota común. 

%\item \begin{equation} \label{r6}
%x_{i,j} \leq y_j \ \text{ para toda $i=1,2,\ldots,n$ y para toda $j=1,2,\ldots,m$.}
%\end{equation}
%Los solicitantes únicamente pueden asistir a universidades abiertas.


Para simplificar la notación del problema, en lugar de usar la matriz de preferencias definida por \ref{matpref} utilizaremos una lista de preferencias que tiene la ventaja de ser un poco más simple para expresar las cotas comunes. 

\begin{dfn}
\label{listpref}
Definimos la \textbf{lista de preferencias} para un problema con $n$ estudiantes, con $m$ universidades y $p$ restricciones
como una función $P$ de tal forma que si $\alpha$ es un estudiante $P(\alpha)$ es el orden de preferencia que le asigna a las universidades,
si $A$ es una universidad $P(A)$ es el orden de preferencia que le asigna a los estudiantes, salvo que el primer lugar denota la cota superior de $A$.
Además, si denotamos a $\mathcal{C}$ como el \textbf{sistema de conjuntos de universidades} %(ver \ref{conj1})
que está conformado por los conjuntos de universidades que forman cada restricción y denotamos a $C_k$ como el conjunto de universidades que forman parte de la restricción común $k$ para todo $k=1,2,\dots,p$.
Entonces $P(C_k)$ representa el orden de preferencias de la cota $C_k$ para todo $k=1,2,\dots,p$.

Es un poco contraintuitivo definir las preferencias en términos de conjuntos de universidades, esto es necesario, por ejemplo, en situaciones cuando se alcanza la cota común entre dos universidades, pero ninguna alcanza su cota superior. Si sólo nos basamos en las listas de preferencia de las dos universidades no es claro bajo qué criterio se aceptan o se rechazan nuevos estudiantes. Además, para que exista consistencia en los órdenes de preferencias hacemos los siguientes supuestos:
\begin{enumerate}
\item Si para alguna $k=1,2,\dots,p$ fija, existe una universidad $A$ que pertenece a $C_k$ y además existe un estudiante $\alpha$ que pertenece a $P(A)$. Entonces, $\alpha$ pertenece a $P(C_k)$.
\item Si para alguna $k=1,2,\dots,p$ fija, existe un estudiante $\alpha$ que pertenece a $P(C_k)$. Entonces, existe una universidad $A$ que pertenece a $C_k$ tal que $\alpha$ pertenece a $P(A)$.
\item Si para alguna $k=1,2,\dots,p$ fija, existe una universidad $A$ que pertenece a $C_K$ y además existen dos estudiantes $\alpha$ y $\beta$ de tal forma que $A$ prefiere a $\alpha$ que a $\beta$. Entonces, $\alpha$ aparece antes que $\beta$ en $P(C_K)$.
\item Si para alguna $k=1,2,\dots,p$ fija, existen dos estudiantes $\alpha$ y $\beta$, de tal forma que $\alpha$ aparece antes que $\beta$ en $P(C_k)$. Entonces, toda universidad $A$ que pertenece a $C_k$ prefiere tener a $\alpha$ que a $\beta$ en caso de que aparezcan en $P(A)$.
\end{enumerate}
\end{dfn}

Damos un ejemplo de una lista de preferencias para dejar todo un poco más claro, en general la idea es la misma que las matrices de preferencias.


\begin{eje} \cite{Todo}
\label{listprefeje}
Supongamos que tenemos tres solicitantes $\alpha,\beta, \gamma$, cuatro universidades $A,B,C,D$ y dos conjuntos de universidades $\{A,B\},\{A,C\}$ que forman cotas comunes. La lista de preferencias sigue la siguiente regla:

\noindent \begin{minipage}{.3\linewidth}
$$P(\alpha)=A,D$$ \\
$$P(\beta)=B$$ \\
$$P(\gamma)=D,C$$ 
\end{minipage}%
\begin{minipage}{.3\linewidth}
$$P(A)=1:\alpha$$ \\
$$P(B)=1:\beta$$ \\
$$P(C)=1: \gamma$$ \\
$$P(D)=1: \alpha, \gamma$$ 
\end{minipage}
\begin{minipage}{.4\linewidth}
$$P(\{A,B\})= 1:\beta,\alpha$$ \\
$$P(\{A,C\})=1:\gamma,\beta$$
\end{minipage}
Aquí (solo para dar algunos ejemplos) podemos ver que $\alpha$ tiene en su lista primero a $A$ y luego a $D$. $C$ únicamente quiere a $\gamma$ como su alumno y además tiene como cota superior al número uno. Las cotas comunes son consistentes de acuerdo con lo planteado anteriormente. 

\end{eje}

La primera pregunta que uno podría hacerse es si siempre existe una asignación estable. Recordemos que en el caso de cotas inferiores no siempre existe y por lo tanto podría aquí suceder lo mismo. El siguiente resultado muestra que de hecho no siempre existe una asignación estable para el problema.

\begin{teo} \cite{Todo}
\label{ejemplo teorema}
Dada una instancia del problema de admisión a universidades con cotas comunes podría no existir solución al problema de encontrar una asignación estable.
\end{teo}
\begin{proof}
Supongamos que para lista \ref{listprefeje} existe una asignación estable $M$. Si $\alpha$ no está emparejada llegamos a una contradicción porque $D$ tiene a $\alpha$ primero en su lista y no está bloqueada por ninguna restricción común. es decir, no existe razón para que $\alpha$ no esté asignada. 

Si $\alpha$ esta asignada a $A$ tenemos que entonces $\gamma$ esta asignada a $D$ y además por la restricción común de $\{A,B\}$, $\beta$ queda sin estar asignado lo cual contradice la hipótesis de estabilidad porque $\{A,B\}$ prefiere tener a $\beta$ que a $\alpha$. 

Si $\alpha$ este asignado a $D$ entonces $\gamma$ está asignado a $C$ lo que implica que $\beta$ queda sin estar asignado. $\{A,B\}$ y $A$ no cumplen con su cota superior (tienen espacio para más alumnos) y $\alpha$ prefiere estar en $A$ que en $D$, por lo tanto, esta asignación no es estable y de hecho concluimos que no existen asignaciones estables para este problema. 

Notamos que si $\beta$ no fuera parte del problema entonces si asignamos a $\alpha$ a $A$ y asignamos a $\gamma$ a $D$ tenemos una asignación estable, por lo tanto, concluimos que a veces sí existe una asignación estable. 
\end{proof}

%Además de no siempre existir una asignación estable sucede algo muy similar a lo planteado en el capítulo 5, en el que podría existir varios emparejamientos estables para el mismo problema de distinto tamaño. El siguiente ejemplo muestra que para este problema también sucede lo mismo.

A partir de este ejemplo queda claro que si extendemos el problema agregando cotas comunes perdemos la garantía de existencia de un emparejamiento estable en nuestro problema. La pregunta clave aquí es si existe alguna condición que garantice la existencia de un emparejamiento estable para cualquier problema de este tipo o si existe algún algoritmo para llegar a uno de forma eficiente, el siguiente resultado muestra que esta pregunta es mucho más complicada de lo que parece. 

\begin{teo} \cite{Todo}
El problema de ver si una instancia del problema de admisión a universidades con cotas comunes admite un emparejamiento estable es NP-completo. 
\end{teo}
\begin{proof}
\footnote{La demostración se parece bastante a la del teorema \ref{np1}.}
Dado que el problema claramente es de decisión y además dada una solución es fácil ver si está es correcta o no, este pertenece a la clase NP.

Para ver que el problema es NP-completo reduciremos el problema del matrimonio estable con listas incompletas con empates a esté\footnote{Recordemos que el problema aquí es ver si existe un emparejamiento estable con todos los hombres y todas las mujeres emparejadas.} (sabemos que este problema es NP-completo por el teorema \ref{NPempates}). Para este caso supondremos que sólo existen empates en las listas de las mujeres, estos son de a lo más longitud 2 y en caso de existir un empate para una mujer arbitraria $w_j$, esté ocupa toda la lista de $w_j$ y al menos uno de los dos hombres pertenecientes a esté tiene a $w_j$ hasta arriba de su lista\footnote{Dadas estas condiciones el problema sigue siendo NP-completo \cite{empates}}. 

Sea $I$ una instancia del problema del emparejamiento estable con listas incompletas con empates, sean $U=\{m_1,m_2,\dots,m_n\}$ la lista de hombres en el problema y sean $W=\{w_1,w_2,\dots,w_n\}$ las mujeres en el problema. Sea $W_0$ un subconjunto de $W$ que contiene a todas las mujeres con un empate de tamaño dos en su lista. Construiremos una instancia $I'$ del problema de admisión a universidades con cotas comunes a partir de $I$. 

Cada hombre en $I$ corresponde a un solicitante en $I'$ y sus listas de preferencia son idénticas en ambas instancias. Si las mujeres pertenecen a $W \setminus W_0$ entonces representan a una universidad en $I'$ y cuentan con una cota superior igual al número 1, sus listas de preferencia son idénticas en $I$ y en $I'$. 

El truco está con las mujeres en $W_0$, supongamos que tenemos una mujer $w_j$ en $W_0$ y supongamos que $m_{j,1}$ y $m_{j,2}$ son los dos hombres que conforman este empate. Creamos entonces dos solicitantes adicionales $b_j^1,b_j^2$, seis universidades $c_j^1,c_j^2,c_j^3,c_j^4,c_j^5,c_j^6$ y cuatro conjuntos de universidades $\{c_j^1,c_j^2\},\{c_j^2,c_j^3\},\{c_j^4,c_j^5\},\{c_j^5,c_j^6\}$ con cotas comunes y las siguientes preferencias: 

\begin{minipage}{.3\linewidth}
$$P(b_j^1)=c_j^5,c_j^2$$ \\
$$P(b_j^2)=c_j^3,c_j^6$$ 
\end{minipage}%
\begin{minipage}{.3\linewidth}
$$P(c_j^1)=1:m_{j,1}$$ \\
$$P(c_j^2)=1:b_j^1$$ \\
$$P(c_j^3)=1: b_j^2$$ \\
$$P(c_j^4)=1: m_{j,2}$$ \\
$$P(c_j^5)=1: b_j^1$$ \\
$$P(c_j^6)=1: b_j^2$$ 
\end{minipage}
\begin{minipage}{.4\linewidth}
$$P(\{c_j^1,c_j^2\})= 1:b_j^1,m_{j,1}$$ \\
$$P(\{c_j^2,c_j^3\})=1:b_j^1,b_j^2$$ \\
$$P(\{c_j^4,c_j^5\})= 1:b_j^1,m_{j,2}$$ \\
$$P(\{c_j^5,c_j^6\})=1:b_j^2,b_j^1$$
\end{minipage}
Para simplificar notación, sea $A_1$ los aplicantes de la forma $m_{j,k}$ y sea $A_2$ los aplicantes de la forma $b_j^k$. Finalmente, sustituimos a $w_j$ en la lista de $m_{j,1}$ por $c_{j,1}$ y sustituimos a $w_j$ en la lista de $m_{j,2}$ por $c_{j,4}$. Afirmamos que $I$ tiene un emparejamiento estable completo $M$ si y sólo si $I'$ tiene una asignación estable $M'$ en la que cada solicitante en $A_1$ está asignado a alguna universidad. Suponiendo que en $M$ tenemos que $m_i$ y $w_j$ están emparejados, la relación entre las dos instancias sigue las siguientes reglas:
\begin{itemize}
\item Si $w_j$ pertenece a $W \setminus W_0$, entonces $m_i$ está asignado a $w_j$ en $M'$.
\item Si $w_j$ pertenece a $W_0$, entonces $m_{j,1}$ está emparejado con $w_j$ si y sólo si en $M'$ tenemos que $m_{j,1}$ está asignado a $c_j^1$, $b_j^2$ está asignado a $c_j^3$ y $b_j^1$ está asignado a $c_j^5$.
\item Si $w_j$ pertenece a $W_0$, entonces $m_{j,2}$ está emparejado con $w_j$ si y sólo si en $M'$ tenemos que $m_{j,2}$ está asignado a $c_j^4$, $b_j^2$ está asignado a $c_j^6$ y $b_j^1$ está asignado a $c_j^2$.
\end{itemize}

Supongamos que existe un emparejamiento estable $M$ en $I$, es claro por construcción que la asignación $M'$ resultante es también estable y que además cada solicitante en $A_1$ es asignado a alguna universidad. 

Supongamos que existe una asignación estable $M'$ en $I'$ en el que todos los solicitantes en $A_1$ están emparejados.
% Me falta el párrafo que empieza in the other ddirection.


Para completar la reducción, es necesario construir una parte adicional de $I'$ como sigue: Para cada solicitante $a_i$ en $A_1$ construimos dos solicitantes adicionales $z_i^1,z_i^2$, cuatro universidades $d_i^1,d_i^2,d_i^3,d_i^4$ con cotas comunes $\{d_i^1,d_i^2\}$ y $\{d_i^2,d_i^3\}$. Las listas de preferencias y cotas son como siguen:
\noindent \begin{minipage}{.3\linewidth}
$$P(z_i^1)=d_i^1,d_i^4$$ \\
$$P(z_i^2)=d_i^4,d_i^3$$ 
\end{minipage}%
\begin{minipage}{.3\linewidth}
$$P(d_i^1)=1:z_i^1$$ \\
$$P(d_i^2)=1:a_i$$ \\
$$P(d_i^3)=1: z_i^2$$ \\
$$P(d_i^4)=1: z_i^1,z_i^2$$ 
\end{minipage}
\begin{minipage}{.4\linewidth}
$$P(\{d_i^1,d_i^2\})= 1:a_1,z_i^1$$ \\
$$P(\{d_i^2,d_i^3\})=1:z_i^2,a_i$$
\end{minipage}

Finalmente añadimos a $d_i^2$ al final de la lista de preferencias de $a_i$ y afirmamos que $I$ tiene un emparejamiento estable completo si y solo si $I'$ tiene emparejamiento estable. Lo primero que notamos es que la construcción es casi idéntica a la usada en la demostración de \ref{ejemplo teorema} entonces podemos usar argumentos similares. Notamos que en $I'$ cada solicitante en $A_i$ está emparejado a la parte no adicional de $I'$ porque si no la parte adicional bloquearía la estabilidad ({ejemplo teorema}) y además para garantizar estabilidad si cada universidad está emparejada con la parte no adicional de $I'$ agregamos las parejas $(z_i^1,d_i^1)$ y $(z_i^2,d_i^4)$ al emparejamiento (al igual que en \ref{ejemplo teorema}).

Por lo tanto, podemos concluir que el problema de ver si una instancia del problema de admisión a universidades con cotas comunes admite un emparejamiento estable es NP-completo.

\end{proof}

Además de no siempre existir una asignación estable sucede algo bastante problemático, para un mismo problema podrían existir dos emparejamientos estables en los que el conjunto de alumnos admitidos sea distinto en cada asignación. Esto tal vez quiere decir que es necesario introducir algún otro criterio para resolver el problema. El siguiente ejemplo demuestra esta afirmación. 

\begin{eje} \cite{Todo}
Supongamos que tenemos 4 estudiantes $\alpha,\beta,\gamma,\delta$, 6 universidades $A,B,C,D,E,F$ y cuatro conjuntos de universidades $\{A,B\},\{B,C\},\{D,E\},\{E,F\}$ con las siguientes listas de preferencias:
\noindent \begin{minipage}{.3\linewidth}
$$P(\alpha)=A$$ \\
$$P(\beta)=E,B$$ \\
$$P(\gamma)=C,F$$ \\
$$P(\delta)=D$$ \\
\end{minipage}%
\begin{minipage}{.3\linewidth}
$$P(A)=1:\alpha$$ \\
$$P(B)=1:\beta$$ \\
$$P(C)=1: \gamma$$ \\
$$P(D)=1: \delta$$ \\
$$P(E)=1: \beta$$ \\
$$P(F)=1: \gamma$$ 
\end{minipage}
\begin{minipage}{.4\linewidth}
$$P(\{A,B\})= 1:\beta,\alpha$$ \\
$$P(\{B,C\})=1:\beta,\gamma$$ \\
$$P(\{D,E\})= 1:\beta,\delta$$ \\
$$P(\{E,F\})=1:\gamma,\beta$$
\end{minipage}

Es fácil ver que las siguientes dos asignaciones son estables.\\

\begin{figure}[H]
\begin{minipage}{.5\linewidth}
\begin{figure}[H]\centering

\begin{tikzpicture}[ scale=0.8]
\tikzset{vertex/.style = {shape=circle,draw,minimum size=1.5em}}
\tikzset{edge/.style = {->,> = latex}}
\filldraw[color=blue!60, fill=blue!5, very thick](0,2.5) ellipse (.8 and 2);
\filldraw[color=green!60, fill=green!5, very thick](4,1.5) ellipse (.8 and 3);


% vertices
% 


\node[vertex] (a) at (0,4) {$\alpha$};
\node[vertex] (b) at (0,3) {$\beta$};
\node[vertex] (c) at (0,2) {$\gamma$};
\node[vertex] (d) at (0,1) {$\delta$};


\node[vertex] (e) at (4,4) {$A$};
\node[vertex] (f) at (4,3) {$B$};
\node[vertex] (g) at (4,2) {$C$};
\node[vertex] (h) at (4,1) {$D$};
\node[vertex] (i) at (4,0) {$E$};
\node[vertex] (j) at (4,-1) {$F$};

\node (k) at (0,5) {Solicitantes};
\node (l) at (4,5) {Universidades};

\path[-stealth] (a) edge (e);
\path[-stealth] (b) edge (i);
\path[-stealth] (c) edge (g);
%\path[-stealth] (b) edge (g);

%\draw (4,4) node[cross=8pt,red] {};


%\draw (0.2,8)--(3.8,8);



\end{tikzpicture}

\end{figure}
\end{minipage}%
\begin{minipage}{.5\linewidth}
\begin{figure}[H]\centering

\begin{tikzpicture}[ scale=0.8]
\tikzset{vertex/.style = {shape=circle,draw,minimum size=1.5em}}
\tikzset{edge/.style = {->,> = latex}}
\filldraw[color=blue!60, fill=blue!5, very thick](0,2.5) ellipse (.8 and 2);
\filldraw[color=green!60, fill=green!5, very thick](4,1.5) ellipse (.8 and 3);


% vertices
% 


\node[vertex] (a) at (0,4) {$\alpha$};
\node[vertex] (b) at (0,3) {$\beta$};
\node[vertex] (c) at (0,2) {$\gamma$};
\node[vertex] (d) at (0,1) {$\delta$};


\node[vertex] (e) at (4,4) {$A$};
\node[vertex] (f) at (4,3) {$B$};
\node[vertex] (g) at (4,2) {$C$};
\node[vertex] (h) at (4,1) {$D$};
\node[vertex] (i) at (4,0) {$E$};
\node[vertex] (j) at (4,-1) {$F$};

\node (k) at (0,5) {Solicitantes};
\node (l) at (4,5) {Universidades};

\path[-stealth] (b) edge (f);
\path[-stealth] (c) edge (j);
\path[-stealth] (d) edge (h);
%\path[-stealth] (b) edge (g);

%\draw (4,4) node[cross=8pt,red] {};


%\draw (0.2,8)--(3.8,8);



\end{tikzpicture}
\end{figure}
\end{minipage}
\caption{Asignaciones estables.}
\end{figure}
\fin
\end{eje}

En lo que sigue de la tesis analizaremos una variante del problema con cotas comunes en el que las restricciones tienen la propiedad de ser anidadas. Veremos que este problema resulta mucho más sencillo de resolver y que además tiene propiedades bastante interesantes.





%\begin{eje}
%Contraejemplo de que no siempre existe usando \ref{listprefeje}
%\end{eje}



\thispagestyle{empty}

\chapter{Matroides}

El objetivo de este capítulo es dar una breve introducción a qué una matroide, así como mostrar varios resultados interesantes sobre esta estructura y útiles para lo que resta de esta tesis \footnote{La gran mayoría de los resultados presentados en este capítulo fueron sacados de \cite{matroid}, otros resultados fueron sacados de \cite{CO1}, \cite{CO2} y \cite{CO3}}.

\subsection*{Un poco de independencia lineal}
Sea $V$ un espacio vectorial finito y sea $\mathcal{F}$ una colección de subconjuntos de $V$ con la propiedad que si $A$ pertenece a $\mathcal{F}$ es porque $A$ es linealmente independiente. Cualquier persona que ha tomado un curso en álgebra lineal puede ver que se cumplen las siguientes tres cosas: 
\begin{enumerate}
\item El conjunto vacío $\emptyset$ pertenece a $\mathcal{F}$. \label{axioma 1}
\item Si $X$ pertenece a $\mathcal{F}$ y $Y$ es un subconjunto de $X$, entonces $Y$ pertenece a $\mathcal{F}$. \label{axioma 2}
\item Si $X$ y $Y$ pertenecen a $\mathcal{F}$ y la cardinalidad de $X$ es uno más la cardinalidad de $Y$, entonces existe un elemento $v$ en $X \setminus Y$ tal que $Y \cup \{v\} $ pertenece a $\mathcal{F}$. \label{axioma 3}
\end{enumerate}

A partir de esto definimos una matroide de la siguiente forma:
\begin{dfn}
Una matroide $M=(S, \mathcal{F})$ es un conjunto $S$ de cardinalidad finita y una colección $\mathcal{F}$ de subconjuntos de $S$ (llamada \textbf{conjuntos independientes}) que cumplen los puntos \ref{axioma 1}, \ref{axioma 2} y \ref{axioma 3} mencionados arriba. 

Todo subconjunto de $S$ que no pertenece a $\mathcal{F}$ es llamado \textbf{conjunto dependendiente}.
\end{dfn}


%A partir de estos tres puntos podemos redefinir algunos conceptos básicos que se ven en álgebra lineal de una forma distinta pero equivalente. Definimos una \textbf{base} como un subconjunto de $V$ en $\mathcal{F}$ de cardinalidad máxima. Definimos el \textbf{rango} de un conjunto $A$ en $V$ como la cardinalidad del subconjunto de $A$ en $\mathcal{F}$ de mayor tamaño. 
Un primer resultado es el teorema del aumento, en términos simples es una forma de generalizar el punto \ref{axioma 3} para conjuntos independientes que difieren en tamaño más de una unidad. 

\begin{teo}{Teorema del aumento} \label{augmentation}
Sea $M=(S,\mathcal{F})$ una matroide y sean $X,Y$ en $\mathcal{F}$ con la propiedad que la cardinalidad de $Y$ es estrictamente mayor a la cardinalidad de $X$. Entonces, existe $Z$ subconjunto de $Y \setminus X$ de tal forma que la cardinalidad de $X \cup Z$ es igual a la cardinalidad de $Y$ y donde además $X \cup Z$ es independiente. 
\end{teo}
\begin{proof}
Sea Z subconjunto de $Y \setminus X$ tal que $X \cup Z$ es independiente y tal que $Z$ es máximal \footnote{En este contexto máximal se refiere a que Z es el conjunto en S con respecto a una propiedad que satisface la propiedad y no es subconjunto propio de otro conjunto que satisface la propiedad.}. 
 
Supongamos que la cardinalidad $X \cup Z$ es menor a la cardinalidad de $Y$, entonces existe $\{y_0\}$ subconjunto de $Y$ tal que la cardinalidad de $y_0$ es uno más la cardinalidad de $X \cup Z$, entonces por la propiedad \ref{axioma 3} de las matroides existe $y$ en $\{y_0\} \setminus (Y \setminus X)$ con la propiedad de que $X \cup Z \cup \{ y\}$ es independiente. Esto contradice la elección de $Z$ porque se pedía que ésta sea máxima, por lo tanto, la cardinalidad de $X \cup Z$ es igual a la cardinalidad de $Y$ y el teorema se cumple. 
\end{proof}

Mostraremos un ejemplo sencillo de una matroide, se recomienda al lector tratar de entender los resultados que siguen a partir de este ejemplo. En muchas ocasiones hacer esto ayuda a mantener claridad.

\begin{eje}
Sea $S$ un conjunto de cardinalidad $n$ y sea $\mathcal{F}$ todos los subconjuntos de $S$ de cardinalidad menor o igual que $k$. Entonces, $M=(S,\mathcal{F})$ es una matroide. 

En este caso se le conoce como  \textbf{Matroide uniforme} de orden $n,k$.
\fin
\end{eje}


A continuación, aprovecharemos la similitud que existe entre el álgebra lineal y la teoría de matroides para mostrar algunas definiciones y resultados interesantes. 

\section{Bases}

Las bases son una de las propiedades más importantes del álgebra lineal, aprovechando la intersección que existe entre independencia e independencia lineal decimos que para una matroide una base es: 

\begin{dfn}
Sea $M=(S,\mathcal{F})$ una matroide, decimos que ${B}$ es una \textbf{base} si:
\begin{enumerate}
\item ${B}$ es independiente.
\item ${B}$ es máximal, es decir, no existe ningún elemento $v$ en $S \setminus {B}$ tal que ${B} \cup \{v\}$ sea independiente.
\end{enumerate}
Al conjunto de todas las bases en M se le denota como $\mathcal{B}(M)$.
\end{dfn}

Es claro que esta definición es equivalente para espacios vectoriales a la de una base. El primer resultado que veremos es que todas las bases son del mismo tamaño, esto sale directo del teorema \ref{augmentation}. 

\begin{cor} \label{cor bases}
Sea $M=(S,\mathcal{F})$ una matroide, entonces todas las bases en M tienen la misma cardinalidad.
\end{cor}
\begin{proof}
Sean $B_1$ y $B_2$ dos bases en M y supongamos que la cardinalidad de $B_1$ es estrictamente menor a la cardinalidad de $B_2$, entonces por el teorema \ref{augmentation} existe $Z$ subconjunto de $B_2 \setminus B_1$ con la propiedad que $B_1 \cup Z$ es independiente, lo cual contradice la definición de qué es una base. Por lo tanto, la cardinalidad de $B_1$ es igual a la cardinalidad de $B_2$ y podemos concluir que todas las bases tienen el mismo número de elementos. 
\end{proof}

Una propiedad interesante de las bases es que es posible definir a una matroide de una forma alternativa haciendo uso de ellas, esto se muestra en el siguiente teorema. 

\begin{teo}{Axiomatización por bases} \label{abases}

Sea $S$ un conjunto finito, decimos que $\mathcal{B}$ es el conjunto de bases de una matroide si y solo si para cualquier par de conjuntos $B_1$ y $B_2$ en $\mathcal{B}$ se cumple que para toda $x$ en $B_1 \setminus B_2$ existe una $y$ con la propiedad que $(B_1 \cup \{y\}) \setminus \{x\}$ pertenece a $\mathcal{B}$. 
\end{teo}
\begin{proof}
Sea $M=(S,\mathcal{F})$ una matroide, como el conjunto vacío es independiente es claro que $\mathcal{B}(M)$ es diferente del vacío (existe por lo menos un elemento que es base en M). Al mismo tiempo, si tomamos 2 bases $B_1, B_2$ en $B(M)$ y consideramos una $x$ arbitraria en $B_1 \setminus B_2$ entonces por el teorema \ref{augmentation} existe $y$ en $B_2 \setminus (B_1 \setminus \{x\})$ de tal forma que la cardinalidad de $(B_1 \cup \{y\}) \setminus \{x\}$ es igual a la cardinalidad de $B_2$ y donde $(B_1 \cup \{y\}) \setminus \{x\}$ es independiente. Además, por el corolario \ref{cor bases} podemos concluir que $(B_1 \cup \{y\}) \setminus \{x\}$ es base. 

Por lo tanto para cualquier par de conjuntos $B_1$ y $B_2$ en $\mathcal{B}(M)$ se cumple que para toda $x$ en $B_1 \setminus B_2$ existe una $y$ con la propiedad que $(B_1 \cup \{y\}) \setminus \{x\}$ pertenece a $\mathcal{B}(M)$.

Ahora por el otro lado, supongamos que existe $\mathcal{B}$ una familia de subconjuntos de S con la propiedad que para cualquier par de conjuntos $B_1$ y $B_2$ en $\mathcal{B}$ se cumple que para toda $x$ en $B_1 \setminus B_2$ existe una $y$ con la propiedad que $(B_1 \cup \{y\}) \setminus \{x\}$ pertenece a $\mathcal{B}$. Decimos que un conjunto es independiente si es un subconjunto de cualquier elemento en $\mathcal{B}$. 

Claramente, el conjunto vacío es independiente y además se cumple que si $X$ es independiente y $Y$ es un subconjunto de $X$ entonces $Y$ también es independiente. Por lo tanto, las propiedades $\ref{axioma 1}$ y \ref{axioma 2} de la definición de matroides se cumplen. 

Ahora, sean $X$ y $Y$ dos subconjuntos independientes de $S$ en donde la cardinalidad de $X$ es igual a $k$ y la cardinalidad de $Y$ es igual a $k+1$. Como $X$ es independiente entonces existe $B_1$ en $\mathcal{B}$ de tal forma que $X$ es un subconjunto de $B_1$ y de forma análoga como $Y$ es independiente entonces existe $B_2$ en $\mathcal{B}$ de tal forma que $Y$ es un subconjunto de $B_2$. 

Para facilitar la notación sean 
$$X = \{ x_1, x_2, \dots, x_k\}$$
$$Y = \{ y_1, y_2, \dots, y_{k+1}\}$$
$$B_1 = \{x_1, x_2, \dots, x_k, b_1, \dots, b_q \}$$
$$B_2 = \{y_1, y_2, \dots, y_{k+1}, c_1, \dots, c_{q-1} \}.$$
Sea $W = B_1 \setminus\{ b_1 \} $, entonces existe z en $B_2$ tal que $W \cup \{z\}$ está en $\mathcal{B}$. Si $z$ está en $Y$ entonces $X \cup \{z\}$ es independiente y se cumple la propiedad \ref{axioma 3}. 

Si $z$ no es un elemento de $Y$, sea $W_1= (W \cup \{ z \}) \setminus \{ b_2 \}$. De nuevo, existe $z_1$ en $B_2$ tal que $W_1 \cup \{ z_1 \}$ está en $\mathcal{B}$. Si $z_1$ está en $Y$ entonces $X \cup \{z_1\}$ es independiente y se cumple la propiedad \ref{axioma 3}. 

Si $z_1$ no es un elemento de $Y$ entonces se repite el mismo procedimiento para $b_3, b_4, \dots b_q$, como la cardinalidad de $\{ b_1, b_2, \dots, b_q\}$ es mayor a la cardinalidad de $\{c_1,c_2, \dots, c_{q-1}\}$ eventualmente existe $z_j$ en $Y$, lo que implica que $X \cup \{z_j\}$ es independiente y se cumple la propiedad \ref{axioma 3} y por lo tanto $\mathcal{B}$ es el conjunto de bases de una matroide y el teorema se cumple. 
\end{proof}

\section{Función rango}
El rango de un conjunto es una propiedad clave del álgebra lineal, aprovechando la intersección que existe entre independencia e independencia lineal decimos que para una matroide la función rango queda definida como: 

\begin{dfn}
Sea $M=(S,\mathcal{F})$ una matroide, la \textbf{función rango} $\rho$ se define como una función que va del conjunto potencia de $S$ a los enteros no negativos, de tal forma que si $A$ es un subconjunto arbitrario de $S$ entonces $\rho(A)$ es igual a la cardinalidad del subconjunto independiente más grande de $A$.
\end{dfn}

Es claro que esta definición es equivalente para espacios vectoriales a la del rango de un conjunto de vectores. Algunas propiedades de la función rango que salen casi directo de la definición son:
\begin{cor}\label{R1}
Sea $M=(S,\mathcal{F})$ una matroide y $\rho$ su función rango entonces
el rango del conjunto vacío es igual a cero. 
\end{cor}
\begin{proof} Como el conjunto vacío es el único subconjunto de sí mismo, como éste es independiente y no contiene ningún elemento, entonces $\rho(\emptyset)=0$. \end{proof}

\begin{cor}\label{R2'}
Sea $M=(S,\mathcal{F})$ una matroide, $\rho$ su función rango y $A,B$ dos subconjuntos de $S$ tales que $A$ es un subconjunto de $B$ entonces $\rho(B) \geq \rho(A)$.
\end{cor}
\begin{proof}
Como todos los subconjuntos de $A$ son subconjuntos de $B$ no puede existir un subconjunto de $A$ independiente que sea más grande que todos los subconjuntos independientes de $B$. Por lo tanto $\rho(B) \geq \rho(A)$.
\end{proof}

\begin{cor} \label{R1'}
Sea $M=(S,\mathcal{F})$ una matroide y $\rho$ su función rango, entonces para todo subconjunto $A$ de $S$ se cumple que $\rho (A)$ es no negativo y que $\rho(A)$ es menor o igual a la cardinalidad de $A$.
\end{cor}

\begin{proof}
Sea $A$ un subconjunto de $S$ arbitrario. 
Como la cardinalidad de un conjunto es siempre no negativa eso implica que $\rho(A)$ es siempre no negativa. El subconjunto de $A$ de mayor cardinalidad es el mismo por lo tanto $\rho(A)$ es menor o igual a la cardinalidad de $A$
\end{proof}

\begin{cor} \label{R2}
Sea $M=(S,\mathcal{F})$ una matroide y $\rho$ su función rango, entonces para todo subconjunto $X$ de $S$ y para todo elemento $y$ de $S$ se cumple que
$$\rho(X) \leq \rho(X \cup \{y\}) \leq \rho(X)+1.$$
\end{cor} 

\begin{proof}
La desigualdad sale directo de que para todo conjunto $X$ se cumple que el agregar un elemento $y$ a éste puede incrementar la cardinalidad de su subconjunto independiente más grande en a lo más una unidad.
\end{proof}

\begin{cor} \label{R3}
Sea $M=(S,\mathcal{F})$ una matroide y $\rho$ su función rango, supongamos que para $X$ subconjunto de $S$ y para $x,y$ elementos de $S$ se cumple que 
$$\rho(X)=\rho(X \cup \{ x\}) =\rho(X \cup \{ y\}) $$ 
entonces 
$$\rho(X)=\rho(X \cup \{ x\} \cup \{ y\}).$$
\end{cor}

\begin{proof}
Supongamos que para $X$ subconjunto de $S$ y para $x,y$ elementos de $S$ se cumple que 
$$\rho(X)=\rho(X \cup \{ x\}) =\rho(X \cup \{ y\}). $$ 
Sea $A = X \cup \{ x\} \cup \{ y\}$, es claro por el corolario \ref{R1'} que $\rho(A) \geq \rho(X)$. Supongamos que $\rho(A) > \rho(X)$, sea $Y$ un subconjunto independiente de X máximal. Como $\rho(A) > \rho(X)$ eso implica que $Y \cup \{x\}$ es independiente o que $Y \cup \{y\}$ es independiente, ambas afirmaciones son falsas porque claramente contradicen que $\rho(X)=\rho(X \cup \{ x\}) =\rho(X \cup \{ y\})$. Por lo tanto, $\rho(A) = \rho(X)$ y el corolario se cumple. 
\end{proof}

\begin{cor} {Desigualdad submodular} \label {R3'}

Sea $M=(S,\mathcal{F})$ una matroide y $\rho$ su función rango, supongamos que $A,B$ son dos subconjuntos arbitrarios de $S$ entonces
$$\rho(A \cup B)+ \rho(A \cap B) \leq \rho(A) + \rho(B).$$
\end{cor}

\begin{proof}
Para facilitar la notación sea $\rho(A \cup B)=p$ y sea $\rho(A \cap B) =q$. Sea $X$ un subconjunto independiente de $A \cap B$ de cardinalidad $q$ entonces existe un conjunto $Y$ con las siguientes propiedades:
\begin{enumerate}
\item $Y$ es independiente.
\item $X$ es un subconjunto de $Y$.
\item $Y$ es un subconjunto de $A \cup B$.
\item La cardinalidad de $Y$ es $p$.
\end{enumerate}
A $Y$ lo podemos escribir como $Y=X \cup V \cup W$, donde $V$ es un subconjunto independiente de $B \setminus A$ y $W$ es un subconjunto independiente de $A \setminus B$. Además se cumple que $X \cup V$ es un subconjunto independiente de B y $X \cup W$ es un subconjunto independiente de A. Vale la pena mencionar que $X, V,W$ son tres conjuntos ajenos y por lo tanto la cardinalidad de $X \cup V$ más la cardinalidad de $X \cup W$ es igual a dos veces la cardinalidad de $X$ más la cardinalidad de $V$ más la cardinalidad de $W$ que ésta a su vez es igual a la cardinalidad de $X$ más la cardinalidad de $Y$ y por construcción esto es igual a $\rho(A \cup B)+ \rho(A \cap B)$. 

Como $\rho(B)$ es mayor igual que la cardinalidad de $X \cup V$ y $\rho(A)$ es mayor igual que la cardinalidad de $X \cup W$ esto implica que 
$$\rho(A \cup B)+ \rho(A \cap B) \leq \rho(A) + \rho(B).$$
\end{proof}

De igual forma que con las bases existe una forma de definir las matroides a través de su función rango, en particular si una función cumple \ref{R1}, \ref{R2} y \ref{R3} entonces es una función rango. El resultado se ve formalmente en el siguiente teorema:

\begin{teo}{Axiomas de rango 1} \label{rank 1}

Una función $\rho$ que va del conjunto potencia de $S$ a los enteros no negativos es la función rango de una matroide en $S$ si y solo si para todo $X$ subconjunto de $S$ y para todos $x,y$ elementos de $S$ se cumple que:
\begin{enumerate}
\item $\rho(\emptyset) =0.$
\item $\rho(X) \leq \rho(X \cup \{y\}) \leq \rho(X)+1.$
\item Si $\rho(X)=\rho(X \cup \{ x\}) =\rho(X \cup \{ y\}) $ entonces $\rho(X)=\rho(X \cup \{ x\} \cup \{ y\}).$
\end{enumerate}
\end{teo}

\begin{proof}
Si $\rho $ es la función rango de $M$ por los corolarios \ref{R1}, \ref{R2} y \ref{R3}, claramente se cumplen los puntos 1,2 y 3 del teorema. 

Supongamos que tenemos una función $\rho$ que va del conjunto potencia de $S$ a los enteros no negativos y que además para $\rho$ se cumplen los puntos 1,2 y 3 del teorema. 

Decimos que $X$ es un subconjunto independiente de $S$ si y solo si $\rho(X)$ es igual a la cardinalidad de $X$. Es claro a partir de esta definición que el conjunto vacío es independiente. 

Supongamos que $A$ subconjunto de $S$ es independiente y sea $B$ un subconjunto arbitrario de $A$, supongamos también que $B$ no es independiente. Como $B$ no es independiente eso implica que $\rho(B)$ es estrictamente menor a la cardinalidad de $B$, sea 
$$A \setminus B = \{ c_1, c_2, \dots c_k\},$$
por el punto 2 del teorema tenemos que $\rho(B) \leq \rho(B \cup \{c_1\}) \leq \rho(B)+1,$ si aplicamos esta propiedad $k$ veces obtenemos que $\rho(B) \leq \rho(B \cup \{ c_1, c_2, \dots c_k\}) = \rho(A) \leq \rho(B)+k.$ Por el otro lado, la cardinalidad de $A$ es igual a la cardinalidad de $B$ más $k$ y como $\rho(B)$ es estrictamente menor a la cardinalidad de $B$, entonces $\rho(A)$ es estrictamente menor que su cardinalidad lo cual claramente es una contradicción porque $A$ es independiente y por lo tanto $B$ es independiente. 

Ahora, sean $X,Y$ dos conjuntos independientes con la propiedad de que la cardinalidad de $Y$ es mayor a la cardinalidad de $X$ por una unidad, para facilitar la notación sean 
$$X = \{ x_1,x_2,\dots, x_q, y_{q+1}, \dots y_k\}$$
$$Y = \{ x_1,x_2,\dots, x_q, z_{q+1}, \dots z_{k+1}\}$$
donde $y_i$ es distinto a $z_j$ para toda $i$ y para toda $j$. 

Supongamos que para toda $j$ se tiene que $X \cup \{z_j\}$ no es independente, esto implica que $\rho(X) =\rho(X \cup \{z_j\})$ para toda $j$. Si aplicamos la propiedad 3 $k+1-q$ veces obtenemos que $\rho(X \cup \{ z_{q+1}, \dots z_{k+1} \} )= \rho(X)$. Además sabemos que $X \cup \{ z_{q+1}, \dots z_{k+1} \}= X \cup Y$, lo cual implicaría que $\rho(Y) = \rho (X)$, esto es claramente una contradicción porque esto implicaría que $Y$ no es independiente por lo tanto existe $z_j$ tal que $X \cup \{z_j\}$ es independiente y podemos concluir que el teorema se cumple.
\end{proof}

Además de esta forma de definir a las matroides usando la función rango existe otra, en particular si una función cumple \ref{R1'}, \ref{R2'} y \ref{R3'} entonces es una función rango para alguna matroide. El siguiente teorema muestra formalmente como sucede esto:

\begin{teo}{Axiomas de rango 2}

Una función $\rho$ que va del conjunto potencia de $S$ a los enteros no negativos es la función rango de una matroide en $S$ si y solo si para todo par $X,Y$ de subconjuntos de $S$ se cumple que: 
\begin{enumerate}
\item $\rho(X)$ toma un valor que se encuentra entre cero y la cardinalidad de $X$. 
\item Si $X$ es un subconjunto de $Y$ esto implica que $\rho(X) \leq \rho(Y)$.
\item $\rho(A \cup B)+ \rho(A \cap B) \leq \rho(A) + \rho(B).$
\end{enumerate}
\end{teo}

\begin{proof}
Si $\rho $ es la función rango de $M$ por los corolarios \ref{R1'}, \ref{R2'} y \ref{R3'}, claramente se cumplen los puntos 1,2 y 3 del teorema. 

Supongamos que tenemos una función $\rho$ que va del conjunto potencia de $S$ a los enteros no negativos y que además para $\rho$ se cumplen los puntos 1,2 y 3 del teorema. 

Al igual que en la última demostración, decimos que $X$ es un subconjunto independiente de $S$ si y solo si $\rho(X)$ es igual a la cardinalidad de $X$. Es claro a partir de esta definición que el conjunto vacío es independiente. 

Claramente, por el punto 1 se cumple que $\rho(\emptyset) =0.$

Sea $X$ un subconjunto arbitrario de $S$ y $y$ un elemento arbitrario de $S$, como la cardinalidad de $\{y\}$ es igual a uno es claro por el punto 1 que $\rho(\{y\})\leq 1$. Además, sabemos que 
$$\rho(X \cup \{y\}) \leq \rho(X \cup \{y\}) + \rho(X \cap \{y\}) \leq \rho(X) + \rho(\{y\}) \leq \rho(X) +1. $$
Si le agregamos la propiedad 2 del teorema obtenemos que $\rho(X) \leq \rho(X \cup \{y\}) \leq \rho(X)+1.$ 

Sea $A$ un subconjunto de $S$ y sean $x,y$ dos elementos de $S$ tales que $\rho(X)=\rho(X \cup \{ x\}) =\rho(X \cup \{ y\}) .$ Por el punto 3 del teorema tenemos que 
$$\rho(A \cup \{x\} \cup \{y\}) + \rho(A) \leq \rho(X \cup \{ x\} + \rho(X \cup \{ y\} = 2 \rho(A),$$
por lo tanto $\rho(A \cup \{x\} \cup \{y\}) = \rho(A)$. 

Como $\rho$ cumple que 
\begin{enumerate}
\item $\rho(\emptyset) =0.$
\item $\rho(X) \leq \rho(X \cup \{y\}) \leq \rho(X)+1.$
\item Si $\rho(X)=\rho(X \cup \{ x\}) =\rho(X \cup \{ y\}) $ entonces $\rho(X)=\rho(X \cup \{ x\} \cup \{ y\}).$
\end{enumerate}
Por el teorema \ref{rank 1} podemos concluir que $\rho$ es la función rango de una matroide en $S$ y por lo tanto se cumple el teorema. 
\end{proof}

Las matroides además de tener análogos con el álgebra lineal también tienen análogos con la teoría de gráficas, la siguiente sección muestra como es qué esto sucede. 

\section{Circuitos}

En teoría de gráficas los circuitos son de vital importancia, en particular sería imposible definir qué es un árbol sin el concepto de circuito. La siguiente definición viene de la intersección que existe entre la teoría de gráficas y las matroides \footnote{Si el lector no tiene familiaridad con la existencia de esta intersección, al finalizar esta sección tendrá claro cuál es la naturaleza de ésta.}, decimos que para una matroide los circuitos quedan definidos como: 

\begin{dfn} % \cite{tufte}
Sea $M=(S,\mathcal{F})$ una matroide, decimos que $C$ es un \textbf{circuito} si 
\begin{itemize}
\item C es un conjunto dependiente. 
\item C es mínimo, es decir, no existe ningún subconjunto $D$ de $C$ tal que $D$ es dependiente.
\end{itemize}
Al conjunto de todos los circuitos en M se le denota como $\mathcal{C}(M)$.
\end{dfn}

Algunas propiedades de los circuitos que salen casi directo de la definición son: 

\begin{cor} \ref{circ1}
Sea $M=(S,\mathcal{F})$ una matroide, $C$ un circuito en M y $\rho$ su función rango entonces, $\rho(C)$ es igual a su cardinalidad menos uno. 
\end{cor}
\begin{proof}
Por definición la cardinalidad del subconjunto independiente más grande de $C$ es igual a la cardinalidad de $C$ menos 1, por lo tanto $\rho(C)$ es igual a su cardinalidad menos uno.
\end{proof}

\begin{cor}
Sea $M=(S,\mathcal{F})$ una matroide y $C$ un circuito en M. Entonces la cardinalidad de $C$ es menor igual que $\rho(S) + 1$.
\end{cor}

\begin{proof}
Si $\rho(S)=k$ esto implica que el subconjunto independiente de $S$ más grande es de cardinalidad $k$ si a este conjunto le agregamos un elemento obtenemos un circuito de cardinalidad $k +1$, es claro que este circuito es el de mayor cardinalidad de todos los elementos de $\mathcal{C}(M)$ y por lo tanto la cardinalidad de $C$ es menor igual que $\rho(S) + 1$ para todo circuito $C$.
\end{proof}

\begin{cor}
Sea $M=(S,\mathcal{F})$ una matroide, si $\mathcal{C}(M)=\emptyset$ entonces $\mathcal{B}(M)=\{S\}$.
\end{cor}

\begin{proof}
Supongamos que $\mathcal{C}(M)=\emptyset$, esto implicaría que no existe ningún subconjunto dependiente de $S$ y por lo tanto $S$ es un subconjunto independiente de $S$ máximal. Por lo tanto, $\mathcal{B}(M)=\{S\}$.
\end{proof}

\begin{cor}
Sea $M=(S,\mathcal{F})$ una matroide, $C$ un circuito en M y $A$ un subconjunto propio de $C$. Entonces $A$ es un subconjunto independiente de $S$.
\end{cor}

\begin{proof}
Si $A$ fuera dependiente esto implicaría que $C$ no es mínimo como se pide en la definición, por lo tanto $A$ es un subconjunto independiente de $S$.
\end{proof}

\begin{cor} \label{C1}
Sea $M=(S,\mathcal{F})$ una matroide y sean $C_1, C_2$ dos circuitos distintos en M. Entonces, $C_1$ no es un subconjunto de $C_2$. 
\end{cor}

\begin{proof}
Supongamos que $C_1$ es un subconjunto de $C_2$, entonces $C_2=C_1 \cup (C_2 \setminus C_1)$. Como $(C_2 \setminus C_1)$ no es vacío entonces $C_1$ es un subconjunto propio de $C_2$ lo que implica que $C_1$ es un subconjunto independiente en $S$. Esto es claramente una contradicción porque $C_1$ es un circuito, por lo tanto $C_1$ no es un subconjunto de $C_2$.
\end{proof}

\begin{cor} \label{C2}
Sea $M=(S,\mathcal{F})$ una matroide, sean $C_1, C_2$ dos circuitos distintos en M y sea $z$ un elemento en $C_1 \cap C_2$. Entonces, existe un circuito $C_3$ subconjunto de $(C_1 \cup C_2)\setminus \{z\}$.
\end{cor}

\begin{proof}
Supongamos que no existe un circuito $C_3$ subconjunto de $(C_1 \cup C_2)\setminus \{z\}$. Esto implica directamente que $(C_1 \cup C_2)\setminus \{z\}$ es un subconjunto independiente de $S$, lo cual implica que $\rho((C_1 \cup C_2)\setminus \{z\}) = \rho((C_1 \setminus \{z\}) + (C_2 \setminus \{z\}) )$ es igual a la cardinalidad de $C_1 \cap C_2$ menos uno. 

Además, por el corolario \label{circ1} tenemos que $\rho(C_1)$ es igual a la cardinalidad de $C_1$ menos uno y que $\rho(C_2)$ es igual a la cardinalidad de $C_2$ menos uno. Si aplicamos la desigualdad de submodularidad obtenemos que $\rho(C_1 \cup C_2)+ \rho(C_1 \cap C_2 )$ es menor o igual que la cardinalidad de $C_1$ más la cardinalidad de $C_2$ menos dos lo cual es igual a la cardinalidad de $C_1 \cup C_2$ más la cardinalidad de $C_1 \cap C_2$ menos dos. 

Ahora, sabemos que $C_1 \cap C_2$ es un subconjunto independiente de $S$ y por lo tanto $\rho(C_1 \cap C_2)$ es igual a la cardinalidad de $C_1 \cap C_2$. Por lo tanto, $\rho(C_1 \cup C_2)$ es menor o igual que la cardinalidad de $C_1 \cup C_2$ menos dos. 

Además, $\rho(C_1 \cup C_2) \geq \rho((C_1 \cup C_2)\setminus \{z\})$, donde $\rho((C_1 \cup C_2)\setminus \{z\})$ es igual a la cardinalidad de $C_1 \cup C_2$ menos uno. 

Por lo tanto, la cardinalidad de $C_1 \cup C_2$ menos uno es menor o igual a la cardinalidad de $C_1 \cup C_2$ menos dos, esto implicaría que $1>2$, lo cual claramente es una contradicción y podemos concluir que existe un circuito $C_3$ subconjunto de $(C_1 \cup C_2)\setminus \{z\}$.
\end{proof}

De igual forma que con las bases y con la función rango existe una forma de definir las matroides a través de sus circuitos, en particular si una familia $\mathcal{C}$ de subconjuntos de $S$ cumple \ref{C1} y \ref{C2} entonces son los circuitos de una matroide. Antes de enunciar y demostrar esto se requiere de un lema y dos teoremas. 

\begin{lem} \label{lem circuitos}
Sea $M=(S,\mathcal{F})$ una matroide, $A$ un subconjunto independiente de $S$ y $x$ un elemento de $x$. Entonces $A \cup \{x\}$ contiene a lo más un circuito.
\end{lem}

\begin{proof}
Supongamos que existen dos circuitos $C_1,C_2$ subconjuntos de $A \cup \{x\}$. Como $A$ es un subconjunto independiente de $S$, tenemos que $x$ está contenida en $C_1 \cap C_2$. Por \ref{C2} existe un circuito $C_3$ subconjunto de $(C_1 \cup C_2)\setminus \{x\}$, lo que directamente implica que $C_3$ está contenido en $A$, lo cual es claramente una contradicción porque $A$ es un subconjunto independiente de $S$. Por lo tanto $A \cup \{x\}$ contiene a lo más un circuito.
\end{proof}

Un corolario directo de este resultado que vale la pena mencionar el siguiente:

\begin{cor} \label{circuitos bases}
Sean $M=(S,\mathcal{F})$ una matroide, $B$ una base en $M$ y $x$ un elemento en $S \setminus B$. Entonces, existe un único circuito $C$ tal que $C$ es un subconjunto de $B \cup \{x\}$. 
\end{cor}

\begin{proof}
La demostración es directa del lema \ref{lem circuitos}, del hecho de que todas las bases en $M$ son subconjuntos independientes de $S$ y que éstas son subconjuntos independientes máximales (al unirles un elemento se vuelven dependientes). 
\end{proof}

A este circuito $C=C(x,B)$ se le llama el \textbf{circuito fundamental} de $x$ en la base $B$. Un resultado importante de esta definición es el siguiente:

\begin{teo}
Sean $M=(S,\mathcal{F})$ una matroide y $B$ una base en $M$, entonces para toda $x$ en $S \setminus B$ $(B \setminus \{ y\}) \cup \{x\}$ es una base en $M$ si y solo si $y$ pertenece a $C(x,B)$ o si $y=x$.
\end{teo}

\begin{proof}
Si $y=x$ el resultado es directo porque por hipótesis $B=$$(B \setminus \{ x\}) \cup \{x\}$ es una base. 
 
Supongamos que $(B \setminus \{ y\}) \cup \{x\}$ es una base en $M$, supongamos además que $y$ no pertenece a $C(x,B)$. Como $y$ no pertenece a $C(x,B)$, tenemos que $C(x,B)$ es un subconjunto de la base $(B \setminus \{ y\}) \cup \{x\}$ y por lo tanto que $C(x,B)$ es un subconjunto independiente de $S$, lo cual es claramente una contradicción porque $C(x,B)$ es un subconjunto dependiente de $S$ y podemos entonces concluir que $y$ pertenece a $C(x,B)$ . 

Por el otro lado, supongamos que $y$ pertenece a $C(x,B)$ y además supongamos que $(B \setminus \{ y\}) \cup \{x\}$ no es una base. Como $(B \setminus \{ y\}) \cup \{x\}$ tiene la misma cardinalidad que $B$ y no es base, como todas las bases tienen el mismo tamaño podemos ver que $(B \setminus \{ y\}) \cup \{x\}$ es un subconjunto dependiente de $S$. Ahora, como$(B \setminus \{ y\}) \cup \{x\}$ es dependiente entonces existe un circuito $C'$ (distinto de $C(x,B)$ subconjunto de $(B \setminus \{ y\}) \cup \{x\}$, lo que implica que $B \cup \{x\}$ contiene dos circuitos distintos, lo cual es una contradicción al lema \ref{lem circuitos} y por lo tanto $(B \setminus \{ y\}) \cup \{x\}$ es una base y el teorema es cierto. 
\end{proof}

Antes de mostrar la equivalencia entre circuitos y matroides es necesario mostrar el siguiente resultado, este teorema puede ser interpretado como un resultado más fuerte que el corolario \ref{C2}.

\begin{teo} \label{C3}
Sean $M=(S,\mathcal{F})$ una matroide, $C_1, C_2$ dos circuitos en $M$ y $x$ un elemento en $C_1 \cap C_2$, entonces para todo elemento $y$ en $C_1 \setminus C_2$ existe un circuito $C$ tal que 
\begin{enumerate}
\item $y$ pertenece a $C$.
\item $C$ es un subconjunto de $(C_1 \cup C_2) \setminus \{x\}$.
\end{enumerate}
\end{teo}

\begin{proof}
Supongamos que existen dos circuitos $C_1,C_2$, un elemento $x$ en $C_1 \cap C_2$ y un elemento $y$ en $C_1 \setminus C_2$ que muestran que el teorema es falso, además supongamos la cardinalidad de $C_1 \cup C_2$ es mínima con esta propiedad, es decir no existen otros dos circuitos con la cardinalidad de su unión menor a la de $C_1$ y $C_2$ que no cumplen esta propiedad. 

Por el corolario \ref{C2}, existe un circuito $C_3$ subconjunto de $(C_1 \cup C_2)\setminus \{x\}$ (no tenemos garantía alguna de que $y$ pertenezca a $C_3$). 

Como $C_3$ no es un subconjunto de $C_1$ (corolario \ref{C1}), tenemos que $C_3 \cap (C_2\setminus C_1)$ no es vacío. Sea $z$ en $C_3 \cap (C_2\setminus C_1)$ arbitraria, podemos ver que $z$ pertenece a $C_3 \cap C_2$ y que $x$ pertenece a $C_2 \setminus C_3$. Como $y$ no pertenece a $C_2 \cup C_3$ podemos ver que $C_2 \cup C_3$ es un subconjunto propio de $C_1 \cup C_2$. Por la minimalidad de $C_1 \cup C_2$ existe $C_4$ tal que $x$ pertenece a $C_4$ y $C_4$ es un subconjunto de $(C_2 \cup C_3) \setminus \{z\}$. 

Ahora, tenemos que $x$ pertenece a $C_1 \cap C_4$ y que $y$ no pertenece a $C_2 \cup C_3$. Como $y$ no pertenece a $C_2 \cup C_3$ tenemos que $y$ pertenece a $C_1 \setminus C_4$. Además tenemos que $C_1 \cup C_4$ es un subconjunto de $C_1 \cup C_2$\footnote{Es un subconjunto propio porque $C_1 \cup C_4$ no contiene a $z$ y $C_1 \cup C_2$ si la contiene.}, de nuevo por la minimalidad de $C_1 \cup C_2$ tenemos que existe un circuito $C_5$ tal que $y$ pertenece a $C_5$ y donde $C_5$ es un subconjunto de $(C_1 \cup C_4) \setminus \{x\}$. Como $C_1 \cup C_4$ es un subconjunto de $C_1 \cup C_2$ tenemos que $C_5$ es un subconjunto de $(C_1 \cup C_2) \setminus \{x\}$, lo cual claramente es una contradicción porque viola directamente la hipótesis de que para $C_1,C_2,x$ y $y$ el teorema es falso y por lo tanto el teorema es cierto. 
\end{proof}

Ahora sí, podemos ver la equivalencia que existen entre circuitos y matroides. Una nota importante es que en la última demostración únicamente se utilizan los corolarios \ref{C1} y \ref{C2}, entonces el siguiente teorema en vez de pedir las propiedades $C_1$ y $C_2$ podría haber pedido en su lugar las propiedades \ref{C1} y \ref{C3}.

\begin{teo} {Axiomatización por circuitos}\label{circuitos}

Una colección $\mathcal{C}$ de subconjuntos de $S$ es el conjunto de circuitos de una matroide si y solo si se cumple que 
\begin{enumerate}
\item Si dos conjuntos distintos $X,Y$ pertenecen a $\mathcal{C}$, entonces $X$ no es un subconjunto de $Y$.
\item Si $C_1,C_2$ son dos subconjuntos distintos de $\mathcal{C}$ y $z$ es un elemento en $C_1 \cap C_2$, entonces existe $C_3$ en $\mathcal{C}$ tal que $C_3$ es un subconjunto de $(C_1 \cup C_2)\setminus \{z\}$.
\end{enumerate}
\end{teo}

\begin{proof}
Si $\mathcal{C}$ es la colección de circuitos de una matroide sabemos por los corolarios \ref{C1} y \ref{C2} que las propiedades 1 y 2 se cumplen. 
Ahora, sea $\{x_1,x_2,\dots,x_q\}$ una familia ordenada arbitraria de objetos en $S$ y sea $\theta$ de la siguiente manera 
$$\theta_{i}= 
\begin{cases}
0 & \qquad \text{si $\{x_1,x_2,\dots,x_i\}$ contiene a un elemento $C$ de $\mathcal{C}$} \\
& \qquad \text{tal que $x_i$ pertenece a $C$} \\
1 &\qquad\text{en otro caso.} 
\end{cases} $$
Sea $t$ una función que va de subconjuntos ordenados de $S$ a los enteros no negativos definida como 
\begin{equation} \label{funcic}
t(\{x_1,x_2,\dots,x_q\}) = \sum_{i=1}^{q}\theta_i,
\end{equation}
Afirmamos \footnote{La demostración de la afirmación se muestra más adelante.} que para toda permutación $\pi$ se cumple que 
\begin{equation} \label{afirmacion}
t(\{x_1,x_2,\dots,x_q\}) = t(\{x_{\pi(1)},x_{\pi(2)},\dots,x_{\pi(q)}\}).
\end{equation}


A partir de esto definimos $r$ como una función que va de los subconjuntos de $S$ a los enteros no negativos y que sigue la regla de correspondencia $r(A) = t(\{x_1,x_2,\dots,x_q\})$, donde $x_1,x_2,\dots,x_q\}$ es una representación ordenada de los elementos de $A$. Por la afirmación mencionada en la ecuación \ref{afirmacion} sabemos que $r$ está bien definida. 

Ahora, es claro que $r(\emptyset)=0$ porque la suma sobre ningún elemento siempre es cero. Además, podemos ver que para cualquier subconjunto $X$ de $S$ y para cualquier elemento $y$ de $S$ claramente se cumple que $r(X)\leq r(X \cup \{y\}) \leq r(X) + 1$ porque al agregar un elemento al conjunto la función no puede decrecer por propiedades de la suma y ésta a lo más puede crecer en una unidad. 

Por el otro lado, sea $A$ un subconjunto de $S$ y sean $x,y$ dos elementos de $S$ y supongamos que 
$$r(A)= r(X \cup \{x\}) = r(A \cup \{y\}),$$
entonces por definición se cumple que existen $C_1,C_2$ en $\mathcal{C}$ tales que $x$ pertenece a $C_1$, con $C_1$ subconjunto de $X \cup \{x\}$ y de forma análoga $y$ pertenece a $C_2$, con $C_2$ subconjunto de $X \cup \{y\}$. Por lo tanto, podemos concluir que $r(A)= r(X \cup \{x\} \cup \{y\})$ y por el teorema \ref{rank 1} $r$ es la función rango de una matroide y podemos concluir que $\mathcal{C}$ es la familia de circuitos de esa matroide.
\end{proof}

La demostración de arriba no está completa, hace falta la demostración de la afirmación mencionada en \ref{afirmacion}. Esta se muestra a continuación.

\begin{lem}
Para la función $t$ definida \ref{funcic} se cumple que para cualquier familia ordenada $\{x_1,x_2,\dots,x_q\}$ de objetos de $S$ y para cualquier permutación $\pi$ que 
$$t(\{x_1,x_2,\dots,x_q\}) = t(\{x_{\pi(1)},x_{\pi(2)},\dots,x_{\pi(q)}\}).$$
\end{lem}

\begin{proof}
Para demostrar el lema solo es necesario mostrar que 
$$t(\{x_1,x_2,\dots x_{q-1},x_q\}) = t(\{x_1,x_2,\dots,x_q,x_{q-1}\}), $$
esto es porque toda permutación puede ser representada como un producto de ciclos de orden dos (ver \cite{moderna}). Para facilitar la notación sean 
$$Y = t(\{x_1,x_2,\dots x_{q-1},x_{q-2}\}),$$
$$t(Y)=a,$$
$$t(Y,x_{q-1}) = a_1,$$
$$t(Y,x_{q}) = a_2,$$
$$t(Y,x_{q-1}) = a_1,$$
$$t(Y,x_{q-1},x_q) = a_{1,2},$$
$$t(Y,x_{q},x_{q-1}) = a_{2,1}.$$
Para demostrar lo deseado es necesario suponer cuatro casos distintos:
\begin{enumerate}
\item Supongamos que no existe $C$ en $\mathcal{C}$ subconjunto de $Y \cup {x_{q-1}}$ tal que $x_{q-1}$ es un elemento de $C$ y supongamos que no existe $C$ en $\mathcal{C}$ subconjunto de $Y \cup {x_{q}}$ tal que $x_{q}$ es un elemento de $C$. Entonces podemos ver que $a_1 = a_2 = a$, entonces existen dos escenarios posibles: 
\begin{enumerate}
\item Existe $C$ en $\mathcal{C}$ subconjunto de $Y \cup \{ x_{q-1}\} \cup \{x_q\}$ tal que contiene a $x_{q-1}$ y a $x_q$ lo que implicaría que $$a_{1,2}=a_1 =a_2 = a_{2,1}.$$
\item No existe $C$ en $\mathcal{C}$ subconjunto de $Y \cup \{ x_{q-1}\} \cup \{x_q\}$ tal que contiene a $x_{q-1}$ y a $x_q$ lo que implicaría que $$a_{1,2}=a_1 +1 =a_2 +1 = a_{2,1}.$$
\end{enumerate}
En los dos escenarios se cumple que $a_{1,2}=a_{2,1}$ como deseamos. 
\item Supongamos que existe $C_2$ en $\mathcal{C}$ contenido en $Y \cup \{ x_{q-1}\}$ tal que contiene a $x_{q-1}$ y además supongamos que existe $C_1$ en $\mathcal{C}$ contenido en $Y \cup \{ x_{q-1}\} \cup \{x_q\}$ tal que contiene a $x_{q-1}$ y a $x_q$. Entonces por hipótesis sabemos que existe $C_3$ en $\mathcal{C}$ subconjunto de $Y \cup \{ x_{q}\}$ tal que contiene a $x_{q}$ y podemos concluir que 
$$a_{1,2}=a_1 =a=a_2 = a_{2,1}.$$
\item Supongamos que existe $C_2$ en $\mathcal{C}$ contenido en $Y \cup \{ x_{q-1}\}$ tal que contiene a $x_{q-1}$ y además supongamos que no existe $C_1$ en $\mathcal{C}$ contenido en $Y \cup \{ x_{q-1}\} \cup \{x_q\}$ tal que contiene a $x_{q-1}$ y a $x_q$. A partir de esto podemos ver que existen dos escenarios posibles: 
\begin{enumerate}
\item Existe $C_3$ en $\mathcal{C}$ subconjunto de $Y \cup \{ x_{q}\}$ tal que contiene a $x_{q}$ y podemos concluir que 
$$a_{1,2}=a_1 =a=a_2 = a_{2,1}.$$
\item No existe $C_3$ en $\mathcal{C}$ subconjunto de $Y \cup \{ x_{q}\}$ tal que contiene a $x_{q}$ y podemos concluir que 
$$a_{1,2}=a_1 +1 = a + 1= a_2 = a_{2,1}.$$
\end{enumerate}
En los dos escenarios se cumple que $a_{1,2}=a_{2,1}$ como deseamos. 
\item Existe $C$ en $\mathcal{C}$ subconjunto de $Y \cup \{x_q\}$ tal que contiene a $x_q$. Este caso es análogo a los casos 2 y 3, podemos concluir que $a_{1,2}=a_{2,1}$. 
\end{enumerate}
Después de plantear todos los casos podemos concluir que $$t(\{x_1,x_2,\dots x_{q-1},x_q\}) = t(\{x_1,x_2,\dots,x_q,x_{q-1}\}) $$ y que por lo tanto el lema es cierto.
\end{proof}


El siguiente resultado muestra de donde viene la conexión entre las matroides y las gráficas, además muestra de donde viene la definición de circuitos.

\begin{teo}
Sea $G=(V.E)$ una gráfica entonces los ciclos en G forman los circuitos de una matroide. 
\end{teo}
\begin{proof}
Claramente ningún ciclo contiene a otro ciclo porque se pide minimalidad en la definición de estos. Sean $C_1$ y $C_2$ dos ciclos en G y sea $e$ una arista en $C_1 \cap C_2$ \footnote{En este caso representamos a un ciclo por el conjunto de aristas que lo forman}. Para facilitar la notación sean 
$$C_1 = \{ e, a_1, \dots, a_p\} $$
$$C_2 = \{e,b_1,\dots, b_q \},$$
Es claro que el conjunto $(C_1 \cup C_2) \setminus \{e\}= \{ a_1, \dots, a_p, b_1,\dots, b_q\}$ contiene un ciclo porque $a_1$ y $b_q$ inciden en el mismo vértice. 

Por lo tanto, por el teorema \ref{circuitos}, los ciclos en G forman los circuitos de una matroide.
\end{proof}

La siguiente sección muestra cómo toda esta teoría se aplica en particular a los problemas de optimización y mostraremos un algoritmo basado en esta teoría muy usado en la investigación de operaciones. 

\section{El algoritmo glotón}

Esta sección muestra el resultado más importante sobre matroides para este trabajo y ejemplifica por qué éstas son muy importantes en la investigación de operaciones. Para definir el algoritmo glotón es necesario antes definir qué es una matroide ponderada y cuál es el problema de optimización asociada a este. 

\begin{dfn}
Decimos que $(M,\omega)$ es una \textbf{matroide ponderada} si $M=(S,\mathcal{F})$ es una matroide y $\omega$ es una función que va del conjunto potencia de $S$ a los reales positivos, con la propiedad de que si $A=\{e_1,e_2,\dots e_p\}$ es un subconjunto de $S$ entonces se cumple que $$\omega(A) = \sum_{k=0}^p \omega(e_k).$$
\end{dfn}

De forma casi directa se define el algoritmo de optimización de la siguiente manera:

\begin{dfn}
Sea $(M,\omega)$ una matroide, si $M=(S,\mathcal{F})$ definimos el problema de optimización $(\mathcal{F},\omega)$ como encontrar el conjunto $A$ en $\mathcal{F}$ con la propiedad de que $\omega(A)$ sea máximal, es decir no existe $B$ en $\mathcal{F}$ tal que $\omega(B) > \omega(A)$.
\end{dfn}

Ahora sí ya encaminados podemos definir el algoritmo glotón, más adelante veremos que este además de ser sencillo y fácil de entender, converge para el problema $(\mathcal{F},\omega)$. El seudocódigo del algoritmo es el siguiente:

\IncMargin{1em}
\begin{Algoritmo}[H]
%\SetKwData{Left}{left}\SetKwData{This}{this}\SetKwData{Up}{up}
%\SetKwFunction{Union}{Union}\SetKwFunction{FindCompress}{FindCompress}
\SetKwInOut{Input}{input}\SetKwInOut{Output}{output}
\Input{Una matroide ponderada $(M,\omega)$.}
\Output{Un conjunto $A$ en $\mathcal{F}$.}
\BlankLine
\emph{Sea $J=\emptyset$ \; }
\Repeat{hasta que no existe $e$ no en $J$ con la propiedad de que $J \cup \{e\}$ pertenece a $\mathcal{F}$}{
\emph{Escogemos $e^*$ en $S$ tal que $\omega(J \cup \{e^*\}) \geq \omega(J \cup \{e\}) $ para toda e tal que $J \cup \{e\}$ pertenece a $\mathcal{F}$ (con $J \cup \{e^*\}$ también en $\mathcal{F}$) y con $\omega(J \cup \{e^*\}) > \omega(J)$ \; } 
\emph{ Actualizamos $J=J\cup \{e^*\}$\;} 
}
\caption{Glotón}
\end{Algoritmo}
\DecMargin{1em}
El siguiente resultado muestra que el algoritmo converge a pesar de ser relativamente sencillo.

\begin{teo} \label{glot}
Sean $M=(S,\mathcal{F})$ y $\omega$ una función de pesos para $M$, entonces el algoritmo glotón converge para el problema de optimización $(\mathcal{F}, \omega)$. 
\end{teo}
\begin{proof}
Sea $B$ la base en $M$ elegida por el algoritmo glotón \footnote{El algoritmo glotón siempre converge a una base porque frena cuando encuentra un subconjunto independiente de $S$ máximal.}, supongamos que existe otra base $T$ tal que $\omega(T)> \omega(B)$ y con la propiedad de que la cardinalidad de $T\cap B$ es máxima en el sentido de que no existe $T'$ tal que $\omega(T')> \omega(B)$ y la cardinalidad de $T'\cap B$ es estrictamente mayor a la cardinalidad de $T\cap B$ . 

Sea $x_k$ el primer elemento que el glotón escogió en $B \setminus T$, por el corolario \ref{circuitos bases} existe un circuito $C(x_k,T)$ tal que $x_k$ pertenece a $C(x_k,T)$ y que $C(x_k,T)$ es un subconjunto de $T \cup \{x_k\}$. 

Sea $y$ en $C(x_k,T) \setminus B$ de tal forma que $(T \cup \{x_k\}) \setminus \{y\}$ es una base (teorema \ref{abases}). 
 
Ahora, sabemos que $\omega(y)\leq \omega(x_k)$ (porque el glotón lo escogió), como $T$ tiene peso máximo eso implica que $\omega(y)=\omega(x_k)$ lo que también implica que $\omega(T)=\omega((T \cup \{x_k\}) \setminus \{y\})$, lo cual es claramente una contradicción porque la cardinalidad de $T\cap B$ es máxima. Por lo tanto, el algoritmo glotón converge para el problema de optimización $(\mathcal{F}, \omega)$.
\end{proof}

Algo muy interesante sobre el algoritmo es que al igual que en las secciones anteriores es posible definir a las matroides por medio del algoritmo glotón. 

\begin{teo} \label{axiomas gloton}
Sea $S$ un conjunto finito y sea $\mathcal{F}$ una familia de subconjuntos (no vacía) de $S$ con las siguientes propiedades:
\begin{enumerate}
\item Si $A$ pertenece a $\mathcal{F}$ y $B$ es un subconjunto de $A$, entonces $B$ pertenece a $\mathcal{F}$.
\item El algoritmo glotón converge para el problema de optimización $(\mathcal{F},\omega)$ para toda función que va del conjunto potencia de $S$ a los reales positivos.
\end{enumerate}
Entonces, $\mathcal{F}$ es la familia de conjuntos independientes de una matroide. 
\end{teo}

\begin{proof}
Para demostrar que $\mathcal{F}$ es la familia de conjuntos independientes de una matroide solo es necesario ver que si $X$ y $Y$ pertenecen a $\mathcal{F}$ y la cardinalidad de $X$ es uno más la cardinalidad de $Y$, entonces existe un elemento $v$ en $X \setminus Y$ tal que $Y \cup \{v\} $ pertenece a $\mathcal{F}$ (ver \ref{axioma 3}). Para facilitar la notación sean 
$$X = \{x_1,x_2,\dots,x_n \}$$ $$Y = \{y_1,y_2,\dots,y_{n+1}\}.$$
Definimos $\omega$ una función que va del conjunto potencia de $S$ a los reales positivos según la siguiente regla:
$$\omega(x_i)=u \text{ para toda $i$ entre $0$ y $n$,}$$
$$\omega(y_j)=v \text{ para toda $y_j$ en $X \setminus Y$,}$$
$$\omega(e)=0 \text{ para toda $e$ en $S \setminus(X \cup Y)$. }$$
Supongamos que $u>v$ y que queremos resolver el problema de optimización $(\mathcal{F}, \omega)$, por hipótesis sabemos que el algoritmo glotón converge para este problema. El algoritmo en sus primeras $n$ iteraciones escoge a $J={x_1,x_2,\dots,x_n}=X$ (sabemos esto porque $X$ está en $\mathcal{F}$ y porque todo subconjunto de $X$ está en $\mathcal{F}$). Ahora tenemos dos escenarios posibles, el algoritmo ya convergió después de $n$ pasos o todavía no converge y realiza por lo menos otra iteración. 
Supongamos que el algoritmo convergió. Sea $t$ igual a la cardinalidad de $X \cap Y$, notamos que se cumple lo siguiente:
$$\omega(X)= n\cdot u$$
$$\omega(Y) = t \cdot u + (n+1-t)v.$$
Si $u$ es igual a $\left(1+\dfrac{1}{2(n-t)}\right)v$\footnote{Con este supuesto se sigue cumpliendo que $u>v$.}, tenemos que $ \omega(Y) > \omega(X)$. Por lo tanto, podemos concluir que el algoritmo no convergió porque claramente no encontró al óptimo. Como el algoritmo no convergió, este realiza por lo menos un paso más en el que selecciona $y_j$ en $X \setminus Y$ de tal forma que $X \cup \{y_j\}$ pertenece a $\mathcal{F}$. Concluimos que se cumple \ref{axioma 3} y por lo tanto $\mathcal{F}$ es la familia de conjuntos independientes de una matroide. 
\end{proof}

Este teorema se puede reescribir de la siguiente forma para encontrar una forma de definir las matroides:

\begin{cor}{Axiomatización por algoritmo glotón.}

Una colección $\mathcal{F}$ no vacía de subconjuntos de $S$ es la familia de conjuntos independientes de $S$ si y solo si 
\begin{enumerate}
\item Si $A$ pertenece a $\mathcal{F}$ y $B$ es un subconjunto de $A$, entonces $B$ pertenece a $\mathcal{F}$.
\item Para toda función $\omega$ que va de $S$ a los reales el algoritmo glotón escoge al conjunto $A$ en $\mathcal{F}$ que maximiza la función 
$$f(A) = \sum_{k=1}^{n}\omega(a_k) \text{ con } A=\{a_1,\dots,a_n\}$$ es decir, no existe $B$ en $\mathcal{F}$ tal que $f(B)>f(A)$.
\end{enumerate}
\end{cor}

\begin{proof}
El corolario es un caso particular del teorema \ref{axiomas gloton}.
\end{proof}




\thispagestyle{empty}

\chapter{Conclusiones}

En esta tesis se construyó el camino para explicar los problemas de admisión a universidades con cotas inferiores o con cotas comunes. Estos a pesar de su facilidad al momento de explicarlos tienen la propiedad de ser NP-Completos y por lo tanto no son fáciles de resolver. Al mismo tiempo se vio que si se hace una reducción al problema de cotas comunes, esté es fácil de resolver y simultáneamente notamos que cuenta con una estructura algebraica muy poderosa. 

Al principio de la tesis se habló de como estos problemas se pueden plantear usando restricciones enteras, esto fue con el fin de plantear todo en el mundo de investigación de operaciones en el que estamos todos acostumbrados. Además, se aprovechó para dar una muy breve introducción a las clases de complejidad por si el lector no se encontraba familiarizado con el tema. 

Posteriormente, se dio una introducción a dos problemas realmente simples: el del matrimonio estable y el de admisión de universidades. Estos a pesar de su simplicidad son el primer paso para entender la situación completa de la tesis y nos dieron herramientas para estudiar el resto de los resultados. 

Aprovechando los resultados anteriores se plantearon tres problemas y se demostró que dos de ellos son NP-Completos (el tercero también cumple esta propiedad, pero se decidió que la demostración no iba con los objetivos de la tesis). Estos problemas resultan interesantes porque con solo incluir una restricción a los problemas originales estos pasan de ser muy fáciles de resolver a muy difíciles. 

Al final se vio que uno de los tres problemas de los planteados (el de cotas comunes) si le agregamos la propiedad de que estas son anidadas, el problema pasa de regreso a ser muy simple de resolver. Además, usando una estructura algebraica llamada matroides tuvimos la oportunidad de reevaluar muchos de los resultados de la tesis en un contexto diferente. 

\thispagestyle{empty}


%----------------------------------------------------------------------------------------
%	APÉNDICES
%----------------------------------------------------------------------------------------

\addtocontents{toc}{\vspace{2em}} % Agrega espacios en la toc

\appendix % Los siguientes capítulos son apéndices

%  Incluye los apéndices en el folder de apéndices

\chapter{Teoría de Conjuntos}

\begin{dfn}
\label{conjunto} \cite{Schaums} \\
Un \textbf{conjunto} es una colección bien definida de objetos; los objetos son llamados \textbf{elementos} del conjunto.
\end{dfn}

\begin{dfn}
La \textbf{cardinalidad} de un conjunto $\Omega$ es la cantidad de elementos que tiene.
\end{dfn}

\begin{dfn}
Definimos al \textbf{conjunto vacio} $\emptyset$ como aquel que no tiene ningun elemento.
\end{dfn}

\begin{dfn}
Dado un conjunto $\Omega$ arbitrario, decimos que un conjunto $A$ es \textbf{subconjunto} de $\Omega$  si todos los elementos de $A$ son elementos de $\Omega$. Esto se denota como $A 	\subseteq \Omega$.
\end{dfn}

\begin{obs}
Dado un conjunto $\Omega$ arbitrario, $\emptyset$ 	$\subseteq$ $\Omega$. 
\end{obs}

\begin{dfn}
Definimos la \textbf{union} entre dos conjuntos $A$ y $B$ como el conjunto $A \cup B$ que contiene a todos los elementos que estan en $A$ y a todos los elementos que estan en $B$.
\end{dfn}

\begin{dfn}
Definimos la \textbf{intersección} entre dos conjuntos $A$ y $B$ como el conjunto $A \cap B$ que contiene a todos los elementos que estan en $A$ y  en $B$. Es decir, si un elemento esta en $A$ y no esta en $B$, entonces no esta en $A \cap B$.
\end{dfn}

\begin{dfn}
Decimos que dos conjuntos son \textbf{ajenos} si la interesección entre ellos es vacia. 
\end{dfn}

\begin{dfn}
Definimos la \textbf{resta} entre dos conjuntos $A$ y $B$ como el conjunto $A \setminus B$ que contiene a todos los elementos que estan en A pero no estan en $B$.
\end{dfn}

\begin{dfn}
\label{conj1}
Decimos que $\mathcal{C}$ es un \textbf{sistema de conjuntos} de $\Omega$ si cada elemento de $\mathcal{C}$ es un conjunto de elementos de $\Omega$.  Es decir, $\mathcal{C}$ es un conjunto cuyos elementos son subconjutos de $\Omega$.
\end{dfn}

\begin{dfn}
Decimos que $\mathcal{P}(\Omega)$ es el \textbf{conjunto potencia} de un conjunto $\Omega$ si $\mathcal{P}(\Omega)$ es un sistema de conjuntos que contiene a todos los subconjuntos de $\Omega$.
\end{dfn}

\begin{teo}
\label{card2}
Si la cardinalidad de $\Omega$ es $n$, entonces la cardinalidad de su conjunto potencia es $2^n$.
\end{teo}
\begin{proof}
La prueba se hace por inducción matemática. \\
Supongamos que $n=0$, entonces $\Omega=\emptyset$. El conjunto vacio unicamente se tiene como subconjunto a si mismo y como $2^0=1$, el teorema es cierto para $n=0$. \\
Supongamos que $n=1$, entonces $\Omega$ tiene dos posibles subconjuntos: $\Omega$ y $\emptyset$. Como $2^1=2$, el teorema es cierto para $n=1$. \\
Supongamos que para alguna $n$ arbitraria el teorema es cierto y supongamos que la cardinalidad de $\Omega$ es $n+1$. Sea $\omega$ un elemento arbitrario de $\Omega$ y consideremos a $\Omega^*$ como $\Omega \setminus \omega$. Ahora, como la cardinalidad de $\Omega^*$ es $n$ sabemos por hipotesis que la cardinalidad de su conjunto potencia es $2^n$. \\
Luego, notamos que los subconjuntos de $\Omega$ son unicamente de dos formas: los que contienen a $\omega$ y los que no. El grupo de los que no contienen a $\Omega$ son el conjunto potencia de $\Omega^*$ y el grupo de los que contienen a $\omega$ puede ser visto como los elementos del conjunto potencia de $\Omega^*$ unidos a $\omega$. Por lo tanto la cardinalidad de $\mathcal{P}(\Omega)$ es el doble a la cardinalidad $\mathcal{P}(\Omega^*)$ que por hipotesis es $2^n$ lo que implica que la cardinalidad de $\mathcal{P}(\Omega)$  es $2 \cdot 2^n=2^{n+1}$ y por lo tanto, el teorema es cierto para toda $n$.
\end{proof}

\begin{cor}
\label{card3}
Si la caridinalidad de $\Omega$ es $n$, entonces la cantidad de subconjuntos no vacios que tiene es $2^{n}-1$.
\end{cor}
\begin{proof}
Por el teorema \ref{card2} sabemos que la cantidad de subconjuntos que tiene $\Omega$ es $2^{n}$, si quitamos el conjunto  vacio nos quedamos con $2^{n}-1$ subconjuntos.
\end{proof}

\begin{dfn} \label{conj2} \cite{Todo} \\
Decimos que $\mathcal{C}$ es un \textbf{sistema anidado de conjuntos} si para todo par de conjuntos no ajenos, $S$ y $S'$, 
se tiene que $S$ es un subconjunto de $S'$ o que $S'$ es un subconjunto de $S$.
\end{dfn}

\begin{obs}
Si la cardinalidad de $\Omega$ es mayor a 2, su conjunto potencia no es sistema anidado de conjuntos.
\end{obs}

\begin{dfn}
Decimos que una colección de conjuntos $a_1,a_2,\dots,a_n$ es una \textbf{partición} de un conjunto $A$, si $$\bigcup\limits_{i=1}^{n}a_i=A.$$ 
\end{dfn}
\thispagestyle{empty}
\chapter{Teoría de Gráficas}

\begin{dfn} \cite{Yo} \\
Una \textbf{gráfica} es una pareja $G = (V, E)$ donde $V$ es un conjunto finito no vacío
cuyos elementos se llaman \textbf{vértices} y $E$ es un conjunto cuyos elementos son subconjuntos de
cardinalidad 2 de V y estos son llamados \textbf{aristas}.
\end{dfn}

\begin{dfn} \cite{Yo} \\ 
El \textbf{grado} de un vértice $v$ es el número de aristas que inciden en $v$, se denota $d(v)$.
\end{dfn}

\begin{dfn} \cite{Ramon}
Se dice que una gráfica es \textbf{bipartita}, si su conjunto de vertices puede ser partido en dos subconjuntos ajenos $X$ y $Y$, de tal forma que cada arista tiene un extremo en $X$ y otro en $Y$. Esta partición tiene el nombre de $\textbf{bipartición}$ de la gráfica.
\end{dfn}


%\include{Apendices/AppendixC}

\addtocontents{toc}{\vspace{2em}} % Agrega espacio en la toc


%----------------------------------------------------------------------------------------
%	BIBLIOGRAFÍA
%----------------------------------------------------------------------------------------

\backmatter
\nocite{*}
\bibliographystyle{apacite}
\bibliography{tesis_itam_bibliografia} %Aquí ponen el nombre del archivo .bib windows



\end{document}